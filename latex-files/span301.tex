% Gregory Ecklund
% October 2024
% SPAN-301 Spanish Conversation & Composition I Notes

\documentclass[12pt]{article}
\usepackage{xcolor}
\usepackage{hyperref}

\title{Spanish Conversation \& Composition I Notes}
\author{Gregory Ecklund}
\date{\today}

\begin{document}
    \maketitle
    \tableofcontents
    \newpage

    \section{Nacionalidades}
        All of the following are written as feminine adjectives. Changing them to masculine should be trivial.
        One thing to note should be that all of the nationalities are lowercase in spanish.
        \begin{itemize}
            \item \textbf{afgana} - Afghan
            \item \textbf{alemana} - German
            \item \textbf{argelina} - Algerian
            \item \textbf{argentina} - Argentinian
            \item \textbf{austriaca} - Austrian
            \item \textbf{belga} - Belgian
            \item \textbf{boliviana} - Bolivian
            \item \textbf{británica} - British
            \item \textbf{búlgara} - Bulgarian
            \item \textbf{chilena} - Chilean
            \item \textbf{china} - Chinese
            \item \textbf{colombiana} - Colombian
            \item \textbf{costarricense} - Costa Rican
            \item \textbf{cubana} - Cuban
            \item \textbf{danesa} - Danish
            \item \textbf{ecuatoriana} - Ecuadorian
            \item \textbf{española} - Spanish
            \item \textbf{etíope} - Ethiopian
            \item \textbf{estadounidense} - American (from U.S.)
            \item \textbf{francesa} - French
            \item \textbf{filipina} - Philipina
            \item \textbf{greiga} - Greek
            \item \textbf{holandesa} - Dutch
            \item \textbf{húngara} - Hungarian
            \item \textbf{india} - Indian
            \item \textbf{iraquí} - Iraqi
            \item \textbf{iraní} - Iranian
            \item \textbf{irlandesa} - Irish
            \item \textbf{israelí} - Israeli
            \item \textbf{italiana} - Italian
            \item \textbf{japonesa} - Japanese
            \item \textbf{jordana} - Jordanian
            \item \textbf{libanesa} - Lebanese
            \item \textbf{libia} - Libyan
            \item \textbf{malaya} - Malaysian
            \item \textbf{maltesa} - Maltese
            \item \textbf{marroquí} - Moroccan
            \item \textbf{mauritania} - Mauritanian
            \item \textbf{mongol} - Mongolian
            \item \textbf{nigeriana} - Nigerian
            \item \textbf{norcoreana} - North Korean
            \item \textbf{norurga} - Norwegian
            \item \textbf{pakistaní} - Pakistani
            \item \textbf{palestina} - Palestinian
            \item \textbf{panameña} - Panamanian
            \item \textbf{paraguaya} - Paraguayan
            \item \textbf{peruana} - Peruvian
            \item \textbf{polaca} - Polish
            \item \textbf{portuguesa} - Portuguese
            \item \textbf{rumana} - Romanian
            \item \textbf{rusa} - Russian
            \item \textbf{saharaui} - Sahrawi (west part of Sahara)
            \item \textbf{salvadoreña} - Salvadorian
            \item \textbf{senegalesa} - Senegalese
            \item \textbf{siria} - Syrian
            \item \textbf{somalí} - Somalian
            \item \textbf{sudafricana} - South African
            \item \textbf{sudanesa} - Sudanese
            \item \textbf{sueca} - Swedish
            \item \textbf{surcoreano} - South Korean
            \item \textbf{ucraniana} - Ukranian
            \item \textbf{tailandesa} - Thai
            \item \textbf{tunecina} - Tunisian
            \item \textbf{turca} - Turkish
            \item \textbf{venezolana} - Venezuelan
            \item \textbf{vietnamita} - Vietnamese
            \item \textbf{yemení} - Yemeni
        \end{itemize}

    \section{Palabras para describir}
        All of the following words are adjectives that can be used to describe a person. If the word ends
        with "(a)", that means that in order to change it to feminine, there are 2 possibilities. If the word
        ends in '-o', change it to '-a', otherwise, if the word ends in '-r', change it to '-ra'.
        Adjectives ending with "(gn)" are gender-neutral.
        \begin{itemize}
            \item desconfiado(a) - distrustful, suspicious
            \item malo(a) - bad, ill, sick, poor (in quality)
            \item bueno(a) - good, ok
            \item amable(gn) - kind, nice
            \item inconformista(gn) - nonconformist, nonconforming
            \item sencillo(a) - simple
            \item flexible(gn) - flexible, supple, pliable
            \item trabajador(a) - hardworking, worker, hardworker
            \item perezoso(a) - lazy
            \item simpático(a) - nice, likeable, pleasant, friendly, kind
            \item generoso(a) - generous, magnanimous
            \item vago(a) - lazy, slacker
            \item desordenado(a) - messy, untidy, disorganized
            \item antipático(a) - unfriendly, unpleasant
            \item comprensible(gn) - understandable, comprehensible
            \item intransigente(gn) - intransigent, die-hard, unyielding
            \item apático(a) - apathetic
            \item abierto(a) - open, open-minded
            \item versado(a) - well versed
            \item aburrido(a) - boring, bored
            \item divertido(a) - fun, entertaining, enjoyable
            \item alegre(gn) - happy, cheerful, bright
            \item ordenado(a) - tidy, well-organized, orderly
            \item apasionado(a) - passionate, impassioned
            \item obstinado(a) - obstinate, stubborn
            \item temperamental(gn) - tempermental
            \item cerrado(a) - closed, closed-minded
            \item sincero(a) - sincere, honest
            \item disponible(gn) - available, free
            \item hablador(a) - talkative, chatty
            \item interesante(gn) - interesting
            \item docto(a) - well versed, erudite
            \item respetuoso(a) - respectful
            \item pesimista(gn) - pessimistic
            \item creativo(a) - creative
            \item radical(gn) - radical, extreme
        \end{itemize}
        
    \section{Las profesiones}
        All of the following are jobs or professions. If the word ends with "(a)", that means that
        in order to change it to feminine, there are 2 possibilities. If the word ends in '-o' or 'e'
        change it to '-a, otherwise, if the word ends in '-r' or '-z', change it to '-ra'. Then change the 'el' to 'la'.
        \begin{itemize}
            \item (el/la) socorrista - the lifeguard
            \item (el/la) astronauta - the astronaut
            \item el vendedor(a) - the salesperson, the salesman, the saleswoman
            \item el cocinero(a) - the cook, the chef
            \item (el/la) cantante - the singer
            \item (el/la) asistente de vuelo - the flight attendant, the steward
            \item (el/la) agente immobiliario - the real estate agent, the realtor
            \item el político(a) - the politician
            \item el bombero(a) - the firefighter, the fireman
            \item (el/la) soldado - the soldier
            \item el médico(a) - the medic, the doctor
            \item el costurero(a) - the seamster, the seamstress
            \item el informático(a) - the IT guy, the computer expert
            \item el fontanero(a) - the plumber
            \item (el/la) piloto - the pilot
            \item (el/la) barman - the bartender
            \item (el/la) taxista - the taxi driver
            \item (el/la) botones - the bellboy, the bellhop
            \item el pintor(a) - the painter
            \item el dependiente(a) - the salesperson, the salesclerk
            \item el juez(a) - the judge
            \item (el/la) policía - the police, the cop
            \item el enfermero(a) - the nurse
            \item el maestro(a) - the teacher
            \item el arquitecto(a) - the architect
            \item el ejecutivo(a) - the executive
            \item (el/la) ama de casa - the housewife, the homemaker
        \end{itemize}
\end{document}