% Gregory Ecklund
% September 2023
% English 101 Notes

\documentclass[12pt]{article}
\usepackage{xcolor}
\usepackage{hyperref}

\title{English 101 Notes}
\author{Gregory Ecklund}
\date{\today}

\begin{document}
    \maketitle
    \tableofcontents
    \newpage

    \section{Proofreading VS Revision}
    What is the difference?

    \subsection{What is Revision?}
    \begin{itemize}
        \item Re-seeing
	    \item Changing perspective
	    \item Getting to the potatoes, not uprooting flowers
    \end{itemize}
	
    \subsection{What is Proofreading?}
    \begin{itemize}
        \item Finding syntax and grammar errors
	    \item Making sure everything makes sense
	    \item Weeding the garden so the flowers can breathe
    \end{itemize}

    \section{Organization}
    How to get your thoughts in order

    \subsection{Organization: How should you do it?}
    \begin{itemize}
        \item What kind of writing are you doing?
	\item What will help your audience?
	\item What is the most important element/takeaway?
    \end{itemize}
    Answering these questions will tell you which organizational form to use.
   
    \subsection{Organization Style : Description}
    \begin{itemize}
        \item Goal - to produce an image in the reader's mind.
        \item Sensory detail
	\item Details ordered spatially, chronologically, or by importance
    \end{itemize}

    \subsection{Organization Style : Narration}
    \begin{itemize}
        \item Goal - to tell a story so that the reader understands the sequence of events.
        \item Relates a story with characters, setting, plot
	\item Details ordered chronologically, using time transitions
    \end{itemize}
	
    \subsection{Organization Style : Illustration/Example}
    \begin{itemize}
        \item Goal - to clarify an abstract or unfamiliar idea for a reader.
        \item Concrete and specific
	\item Provide enough and a sufficient variety
    \end{itemize}
	
    \subsection{Organization Style : Process}
    \begin{itemize}
        \item Goal - to explain how a reader should do something.
	\item "How-to" writing
	\item Specific, step-by-step instructions
    \end{itemize}

    \subsection{Organization Style : Definition}
    \begin{itemize}
        \item Goal - to help a reader understand the meaning of a term.
	\item Three components:
	\begin{enumerate}
	    \item The term being defined
            \item The category it belongs to
	    \item Distinguishing factors within the category
        \end{enumerate}
    \end{itemize}

    \subsection{Organization Style : Comparison/Contrast}
    \begin{itemize}
        \item Goal - to show a reader the similarities and differences between two or more entities/ideas.
	\item Compare similarities
	\item Contrast differences
	\item Can be organize point-by-point or subject-by-subject
    \end{itemize}

    \subsection{Organization Style : Cause/Effect}
    \begin{itemize}
        \item Goal - to show a reader the causal relationship between two or more entities/ideas.
	\item Determine the focus of your topic sentence or thesis statement by asking questions about an event or action:
        \begin{itemize}
	    \item "Why did this happen?"
            \item "What caused it?"
            \item "What happened because of this event?"
	    \item "What was the effect?"
	\end{itemize}
    \end{itemize}

    \subsection{Organization Style : Argument/Persuasion}
    \begin{itemize}
        \item Goal - to get a reader to agree with your point of view.
	\item Pick a subject that you feel strongly about and can argue for.
	\item Build your case with reasoning and sources.
	\item Avoid logical errors.
	\item Arrange your points in logical order.
    \end{itemize}

    \newpage
    \section{Reflection}
    There are two types of reflection:

    \subsection{Reflection-in-Action}
    \begin{itemize}
        \item Circle back to earlier understandings and observations.
	\item Re-think your writing from the present to think about how they will look in the future.
	\item Explain to others so that we explain to ourselves, re-understanding.
	\item The goal:
	\begin{itemize}
	    \item Re-understand your work for yourself.
            \item Re-understand yourself as a writer.
	    \item Re-understand how your work can reach an audience.
        \end{itemize}
    \end{itemize}

    \subsection{Constructive Reflection}
    This is where it gets meta.
    \begin{itemize}
        \item Reflecting on your writing - generalization across rhetorical situations.
        \item Constructive reflection - seeing yourself do that generalization.
	\item Forming your identity as a writer.
	\item Seeing how you've changed approach.
	\item The goal;
	\begin{itemize}
	    \item Build your knowledge of your own writing process.
	    \item Build confidence in your ability as a writer.
	    \item Build up who you are as a writer.
        \end{itemize}
    \end{itemize}
\end{document}
