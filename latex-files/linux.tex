% Gregory Ecklund
% September 2023
% CS462 Linux Tools Notes

\documentclass[12pt]{article}
\usepackage{xcolor}
\usepackage{hyperref}
\usepackage[T1]{fontenc} % For some font enconding in the table?

\title{CS462 Linux Tools Notes}
\author{Gregory Ecklund}
\date{\today}

\begin{document}
    \maketitle

    \tableofcontents
    \newpage

    \section{Sed}
    The sed utility is a stream editor that reads one or more files, makes changes 
    according to a script of editing commands and writes the results to standard output. 
    Sed performs many ed or ex commands. ed and ex are line editors, which were the original 
    type of editors. Line editing commands are easier for scripting.

    \subsection{Line Addressing}
    \begin{tabular}{|l|l|}
        \hline
        n & absolute line number n \\
        \hline
        i, j & range of lines from line i to line j \\
        \hline
        + & go forward one line \\
        \hline
        - & go backwards one line \\
        \hline
        +n & go forward n lines \\
        \hline
        -n & go backward n lines \\
        \hline
    \end{tabular}

    \subsection{Line Editing Commands}
    \begin{tabular}{|l|l|}
        \hline
        addr & goto addr \\
        \hline
        [addr] 1 & print line(s) to standard output \\
        \hline
        /<pattern> & goto first line that matches pattern \\
        \hline
        ?<pattern> & goto last line that matches the pattern \\
        \hline
        [addr] a text . & append text (line by line) after addr until a '.' is typed in the first column \\
        \hline
        [addr] i text . & insert text (line by line) before addr until a '.' is typed in the first column \\
        \hline
        [addr] d & delete specified lines(s) \\
        \hline
        [addr] c text . & replace specified line(s) with text until a '.' is typed in the first column \\
        \hline
        [addr] s/pat1/pat2/ & substitute pat2 for pat1 at addr \\
        \hline
    \end{tabular}

    \subsection{Using Editing Commands in Sed}
    Sed editing commands are implicitly global (applied to every line in the file).
    Example sed scripts:
    \begin{itemize}
        \item \begin{verbatim}50d\end{verbatim}
        \item \begin{verbatim}/^#/d\end{verbatim}
        \item \begin{verbatim}/Smith/,$d\end{verbatim}
        \item \begin{verbatim}s/old/new/\end{verbatim}
        \item \begin{verbatim}s/old/new/g\end{verbatim}
        \item \begin{verbatim}s/old/new/2\end{verbatim}
    \end{itemize}

    \subsection{Performing Multiple Commands on a Match}
    Can use curly braces to perform multiple commands on a single match
    Example: Make each list element within an ordered list a paragraph and change the 
    ordered list to an unordered list in an html file.
    \begin{verbatim}
        /^<ol>/,/^<\/ol>/{
        s/^<ol>/<ul>/
        s/^<li>/<p><li>/
        s/^<\/ol>/<\/ul>/
        }
    \end{verbatim}

    \subsection{Referencing the Search String}
    The ampersand (\&) represents the extent of the pattern was matched and can be
    referenced in the replacement string.
    Examples:
    \begin{itemize}
        \item s/[A-Z][A-Z][A-Z]*/"\&"/g --- quote all acronyms
        \item s/[Uu]nix/"\&"/g --- quote all references to Unix or unix
    \end{itemize}

    \subsection{Referencing Portions of a Search String}
    \begin{verbatim}
    Can reference portions of a search string by enclosing them in escaped
    parentheses, "\(" and "\)", and referencing them by \<num> in the 
    replacement string. Example: Reversing the order of the first two 
    integer values separated by a colon in a file.
    \end{verbatim}
    \begin{itemize}
        \item \begin{verbatim}s/^\([0-9]*\):\([0-9]*\)/|2:\1/\end{verbatim}
    \end{itemize}
    



\end{document}
