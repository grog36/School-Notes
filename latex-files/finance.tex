% Gregory Ecklund
% August 2024
% FIN-210: Managing Money for Financial Success Notes

\documentclass[12pt]{article}
\usepackage{xcolor}
\usepackage{hyperref}

\title{Managing Money for Financial Success Notes}
\author{Gregory Ecklund}
\date{\today}

\begin{document}
    \maketitle
    \tableofcontents
    \newpage

    \section{Unit 1: Money, Happiness, Financial Planning, and the Time Value of Money}

        \subsection{Consumer Theory and  Financial Planning}
            \subsubsection{DOES MONEY = HAPPINESS}
                Key word: \textbf{\underline{Indirectly!}}
                Basically 3 types of resources:
                \begin{itemize}
                    \item Human Capital - Knowledge, skills, abilities
                    \item Financial Capital - \$\$
                    \item Durable Goods (lasting stuff) - Cars, houses, etc.
                \end{itemize}
            \subsubsection{WHAT IS FINANCIAL CAPITAL?}
                \begin{itemize}
                    \item \underline{Money}
                        \begin{itemize}
                            \item Liquid Assets
                        \end{itemize}
                    \item \underline{Resources}
                        \begin{itemize}
                            \item Unearned Assets/Income
                            \item Future Cashflows
                        \end{itemize}
                    \item \underline{Wealth}
                        \begin{itemize}
                            \item Investment Assets
                            \item Personal Property
                            \item Retirement Assets
                        \end{itemize}
                \end{itemize}
            \subsubsection{WHAT IS HUMAN CAPITAL?}
                "It is the stock of knowledge, habits, social and personality attributes, including creativity, embodied
                in the ability to perform labor so as to *produce economic value*"
            \subsubsection{MAXIMIZING HAPPINESS}
                The goal of the financial planning process is \textbf{NOT} to get the most money. The goal is to get the most
                \textbf{HAPPINESS} (Maximize total utility over the life-cycle).
                \newline
                \textbf{"Using money, not as an end in itself, but as a tool to help achieve what is most important in life."}
                \newline
                \textbf{--Klontz, Kahler, and Klontz (2008)}
            \subsubsection{THE FINANCIAL PLANNING PROCESS - THE 6 STEPS}
                Step 0: First Meeting - Define Scope of Relationship (Advice vs Planning)
                \begin{enumerate}
                    \item Determine Life Goals, THEN Financial Goals
                    \item Evaluate Your Financial Health
                    \item Develop a Plan of Action
                    \item Implement Your Plan
                    \item Track Progress Using Financial Ratios
                    \item Review Your Progress, Reevaluate, and Revise Your Plan
                \end{enumerate} 
            \subsubsection{SET SMART GOALS}
                \begin{itemize}
                    \item \textbf{S}pecific
                        \begin{itemize}
                            \item What exactly do you want to achieve?
                        \end{itemize}
                    \item \textbf{M}easurable
                        \begin{itemize}
                            \item How much is it going to cost?
                        \end{itemize}
                    \item \textbf{A}chievable
                        \begin{itemize}
                            \item Based on your financial situation, can you achieve the goal?
                        \end{itemize}
                    \item \textbf{R}ealistic
                        \begin{itemize}
                            \item Even if you can achieve the goal, will you?
                        \end{itemize}
                    \item \textbf{T}imely
                        \begin{itemize}
                            \item What is the timeframe for achieving the goal?
                        \end{itemize}
                \end{itemize}
            \subsubsection{EVALUATE YOUR FINANCIAL HEALTH}
                \begin{itemize}
                    \item Examples of Qualitative or Subjective Information
                        \begin{itemize}
                            \item Health
                            \item Life expectancy
                            \item Family circumstances
                            \item Values
                            \item Attitudes
                            \item Expectations
                            \item Earnings potential
                            \item Risk tolerance
                            \item Goals
                            \item Needs
                            \item Priorities
                            \item Current course of action
                        \end{itemize}
                    \item Examples of Quantitative or Objective Information
                        \begin{itemize}
                            \item Age
                            \item Dependents
                            \item Other professional advisors
                            \item Income
                            \item Expenses
                            \item Cash flow
                            \item Savings
                            \item Assets
                            \item Liabilities
                            \item Available resources
                            \item Liquidity
                            \item Taxes
                            \item Employee benefits
                            \item Government benefits
                            \item Insurance coverage
                            \item Estate plans
                            \item Capacity for risk
                            \item Education and retirement accounts and benefits
                        \end{itemize}
                    \item What resources do you have available?
                    \item How much \textbf{human capital} do you have?
                        \begin{itemize}
                            \item Do you need more to reach your goals?
                            \item Should you adjust your goals?
                        \end{itemize}
                    \item How much \textbf{financial capital} do you have?
                        \begin{itemize}
                            \item Check financial statement
                            \item Check ratios
                        \end{itemize}       
                \end{itemize}
            \subsubsection{DEVELOP PLAN}
                \begin{itemize}
                    \item Relect \textbf{goals}
                    \item Informed and controlled \textbf{budget}
                    \item Determine \textbf{investment strategy}
                    \item Key Factors to Consider:
                        \begin{itemize}
                            \item Flexibility
                            \item Liquidity
                            \item Protection
                            \item Minimization of Taxes
                        \end{itemize}
                \end{itemize}
            \subsubsection{IMPLEMENT}
                \begin{itemize}
                    \item Implementing the plan means putting the plan to work
                    \item Although you have the plan developed, it takes discipline and desire to put it into action
                    \item Create a table of who is responsible for what action and by when.
                        \begin{itemize}
                            \item Follow up with each party to ensure implementation is successful!
                        \end{itemize}
                \end{itemize}
            \subsubsection{THE FINANCIAL PLANNING PROCESS}
                Three ways...
                \begin{enumerate}
                    \item Do it yourself
                    \item Have someone do it for you
                    \item Or... have it done to you
                \end{enumerate}

        \subsection{Interest and the Time Value of Money}
            \subsubsection{VALUE OF CONSUMPTION}
                When cash flow is greater than total expenses there is a \textbf{\underline{surplus.}}
                \newline What three things can you do with a surplus?
                \begin{enumerate}
                    \item Consume more today
                    \item Pay back borrowed money used to increase consumption in the past
                    \item Invest money to pay for more consumption in the future
                \end{enumerate}
                Factors that influence this decision:
                    \begin{itemize}
                        \item Value of current consumption to you
                        \item My ability to pay you back (perception thereof)
                            \begin{itemize}
                                \item YOUR Risk Tolerance
                            \end{itemize}
                        \item Your time horizon
                        \item Opportunity Cost
                        \item Increase in Future Consumption - INTEREST
                    \end{itemize}
            \subsubsection{BENEFITS OF COMPOUNDING}
                \$1,000 investment earning 12\% compounded annually = \textbf{\$120 in interest}    
                \newline
                \newline \$1,000 investment earning 12\% compounded monthly (12\%/12 months = 1\% each month).
                \newline 1\% * \$1,000 = \$10 of interest in month 1
                \newline 1\% * \$1,010 = \$10.10 of interest in month 2
                \newline 1\% * \$1,020.10 = \$10.20 of interest in month 3
                \newline ...
                \newline After 12 months = \textbf{\$126.83 in interest}
            \subsubsection{YOUR FRIEND WANTS TO RETIRE A MILLIONAIRE...}
                Your 20-year-old friend confided in you that they want to retire a millionaire. They told you they
                want to retire at full retirement age (67) and they want to make monthly deposits into their checking
                account with their bank.
                \newline 47 * 12 = 564 months
                \newline \$1,000,000 / 564 = \$1,773.05 (a month)
                \newline Income needed to meet \textbf{Savings Ratio} $\approx$ \$180,000
                \newline
                \newline You tell them about compound interest and how it can be used to reach a greater sum with less
                of a strain on cashflows. You recalculate their monthly savings need.
                \newline 47 * 12 = 564 months
                \newline 7\% APY on portfolio
                \newline Calculated Monthly Savings need: \$227.98
                \newline Income needed to meet \textbf{Savings Ratio} $\approx$ \$25,000
            \subsubsection{TIME VALUE OF MONEY}
                \begin{itemize}
                    \item Future Value - How much it would be worth
                    \item Present Value - How much I should play for that investment
                \end{itemize}
                But you can also compute:
                \begin{itemize}
                    \item Payment
                    \item Number of periods
                    \item Interest Rate per period
                \end{itemize}
                Simple and Compound Interest Formula
                \newline Simple interest = $PRT$
                \newline Compound interest = $P(1+R)^T - P$
                \newline P = Principal
                \newline T = Term           
                \newline R = Rate
            \subsubsection{THE PUZZLE PIECES OF TVM}
                \begin{itemize}
                    \item FV - Future Value
                    \item PMT - Payment
                    \item PV - Present Value
                    \item I/Y - Interest (per period)
                    \item N - Term (number of periods)
                \end{itemize}
                \textbf{You will be given 3 or 4 pieces and need to find the 5th or 6th peice in all TVM problems}
            \subsubsection{INFLOWS VS. OUTFLOWS}
                Funds moving AWAY from you are recorded as (-) and are reffered to as cash OUTFLOWS
                \begin{itemize}
                    \item Cash into an investment
                    \item Money to pay down a loan
                    \item Monthly savings
                \end{itemize}
                Funds coming TO you are entered as (+) and are referred to as cash INFLOWS
                \begin{itemize}
                    \item Cash from an investment
                    \item Cash from a loan
                    \item Cashing out your savings
                \end{itemize}
            \subsubsection{WHEN TO USE BOTH PV AND FV}
                \begin{itemize}
                    \item With saving calculations where there is a starting balance
                    \item Figuring out what interest rate is needed to grow an investment to a certain amount
                    \item Figuring out what term (N) is needed to grow an investment to a certain amount
                \end{itemize}
            \subsubsection{KNOWING THE TIME PERIOD}
                You may need to adjust your time period (N) and your rate (I/Y) depending on the frequency of compounding. This is because the standard
                is annual (once a year) but often you will be dealing with other time periods.
                \newline Semi-annual is compounded \textbf{twice} a year.
                \newline Quarterly is compounded \textbf{four} times a year.
                \newline Monthly is compounded \textbf{twelve} times a year.
            \subsubsection{ADJUSTING THE TIME PERIOD}
                \textbf{Step 1: Adjust the time period}
                \begin{itemize}
                    \item To adjust your time period (N) you need to multiply N by the number of periods.
                \end{itemize}
                Ex. N*2, N*4, N*12
                \begin{itemize}
                    \item This will ensure all of the extra compounding periods are accounted for.
                \end{itemize}
                \textbf{Step 2: Adjust the rate}
                \begin{itemize}
                    \item To adjust your rate (I/Y) you need to divide I/Y by the number of periods.
                \end{itemize}
                Ex. (I/Y)/2, (I/Y)/4, (I/Y)/12
                \begin{itemize}
                    \item This will ensure all of the extra compounding periods are accounted for.
                \end{itemize}
                \textbf{HELPFUL TIP:}
                \begin{itemize}
                    \item If the information provided does not explicitly state the compounding period for the interest, but does have a PMT cashflow,
                    \textbf{adjust the N \& I/Y to match the PMT cashflow}   
                \end{itemize}
        \subsection{Net Present Value}
            \subsubsection{CAPITAL BUDGETING}
                \textbf{Capital Budgeting} is the process of analyzing projects and deciding which are acceptable investments and which should be purchased/undertaken.
                \begin{itemize}
                    \item If a firm identifies an investment opportunity with a present value that is greater than its cost, the value of the firm will
                    increase if the investment is purchased.
                \end{itemize}
                \underline{Steps in Capital Budgeting}
                \begin{enumerate}
                    \item Estimate the cash flows expected to be generated.
                    \item Evaluate the riskiness of the projected cash flows to determine the appropriate rate of return to use for computing present value
                    (PV) of cash flows. (Cost of Capital)
                    \item Compute the PV of the expected cash flows.
                    \item \textbf{Compare the PV of the future expected cashflows with the initial investment}
                        \begin{itemize}
                            \item \textbf{Net Present Value (NPV)} is derived by subtracting the purchase price of the asset from the present value of its expected cash
                            flows, the result is the net dollar value of making the purchase or otherwise engaging in the opportunity.
                                \begin{itemize}
                                    \item A project is acceptable if \textbf{NPV} > \$0
                                \end{itemize}
                            \item CF = after-tax Cash Flow "bottom line"
                            \item r = rate of return required to invest in the project (discount rate)
                            \item n = final cash flows
                            \item t = cumulative time periods (cash flows)
                        \end{itemize}                            
                \end{enumerate}
        
    \section{Unit 2: Financial Statements, Ratios, (Income) Taxation, and Money Psychology}
    
        \subsection{Financial Statements and Ratios}
            \subsubsection{EVALUATE YOUR FINANCIAL HEALTH}
                \begin{itemize}
                    \item \textbf{What resources do you have available}
                    \item How much human capital do you have?
                        \begin{itemize}
                            \item Do you need more to reach your goals?
                            \item Should you adjust your goals?
                        \end{itemize}
                    \item How much financial capital do you have?
                        \begin{itemize}
                            \item Check financial statements
                            \item Check ratios
                        \end{itemize}
                \end{itemize}
            \subsubsection{ASSETS}
                \begin{itemize}
                    \item Monetary assets - extremely liquid
                    \item Investment assets - Stocks, bonds, mutual funds
                    \item Retirement assets - IRAs, 401(k)s, 403(b), SEP
                    \item Houses and autos - Valued at fair market value
                \end{itemize}
            \subsubsection{BOOK VALUE VS. FAIR MARKET VALUE}
                \begin{itemize}
                    \item \textbf{Book value} - what you originally bought the asset for, recorded in the accounting ledger
                    \item \textbf{Fair market value} - what it would take to replace the asset in its most recent condition
                \end{itemize}
            \subsubsection{LIABILITIES}
                \begin{itemize}
                    \item Current Liabilities (< 1 year):
                        \begin{itemize}
                            \item Past-due bills
                            \item Credit card debt
                            \item Title/Payday loan
                        \end{itemize}
                    \item Long-Term Liabilities (> 1 year):
                        \begin{itemize}
                            \item Student loan debt
                            \item Mortgage
                            \item Auto loan (not leased autos)
                            \item Personal loan
                        \end{itemize}
                \end{itemize}
            \subsubsection{WHAT QUESTIONS ARE ANSWERED USING A BALANCE SHEET?}
                \begin{itemize}
                    \item How am I handling my finances right now?
                    \item Do I have an debt?
                        \begin{itemize}
                            \item If so, how much?
                            \item Can I take on more debt?
                        \end{itemize}
                    \item How much cash and cash equivalents do I have on hand?
                        \begin{itemize}
                            \item What do I have set aside for emergencies?
                            \item What do I have to use as collateral?
                        \end{itemize}
                \end{itemize}
            \subsubsection{INCOME AND EXPENSE STATEMENT}
                Sometimes called a cash flow statment, an income and expense statement is a statement of your financial
                position over a specified period of time, usually a month or a year.
                \newline What it tells you:
                    \begin{itemize}
                        \item Am I spending within my means?
                        \item Am I spending too much \textbf{disposable income}
                        \item Did I save enough last year?
                    \end{itemize}
            \subsubsection{INCOME}
                \begin{itemize}
                    \item Wages and salaries
                    \item Bonuses
                    \item Interest and dividends
                    \item Child support
                    \item Tax refunds
                    \item Gifts
                    \item Government transfer payments
                \end{itemize}
                Types of Income:
                    \begin{itemize}
                        \item Gross Income (Everything)
                            \begin{itemize}
                                \item Before taxes
                            \end{itemize}
                        \item Net Income (Take Home)
                            \begin{itemize}
                                \item Also known as \textbf{Disposable Income}
                            \end{itemize}
                        \item \textbf{Discretionary Income} (Choice)
                            \begin{itemize}
                                \item What's left to spend on variable expenses
                                \item Calculated differently for Gov. Benefits
                            \end{itemize}
                    \end{itemize}
            \subsubsection{FIXED OR VARIABLE}
                \begin{itemize}
                    \item Mortgage Payment - Fixed
                    \item Groceries - Variable
                    \item Car Loan Payment - Fixed
                    \item Utilities - Variable
                    \item Entertainment - Variable
                    \item Credit Card Payment - Variable
                    \item Auto Insurance Payment - Fixed
                    \item Past Due Bill (Paid in the last period) - Fixed
                \end{itemize}
            \subsubsection{WHY DO I NEED A BUDGET?}
                \begin{itemize}
                    \item Personal control over YOUR OWN finances!
                    \item Keeps your future on track!
                    \item Easy to identify fraudulent charges!
                    \item Holds you accountable - \textbf{Behavioral}
                    \item Makes your spending more transparent
                \end{itemize}
                The budget controls the I\&E. Stmt., Balance Sheet, and our ability to accomplish our goals
            \subsubsection{HOW DO I BUDGET?}
                \begin{itemize}
                    \item Figure out your fixed expenses versus your variable!
                        \begin{itemize}
                            \item Tip: make the budget last, IE. Stmt. First
                        \end{itemize}
                    \item Track the amount of cash you spend!
                    \item Access your spending through online bill pay (utilities, phone, internet, etc.)
                    \item Keep up with credit card statements
                    \item Keep up with bank statements
                \end{itemize}
            \subsubsection{USING THE BUDGET IN PLANNING}
                \begin{itemize}
                    \item Standard of Living Based: Less Painful
                        \begin{itemize}
                            \item Start with previous I\&E Stmt.
                            \item \textbf{Surplus}: Use Surplus to Fund Financial Goals and look for ways to increase DI.
                            \item \textbf{Deficit}: Look for ways to increase DI, how long has there been a deficit?
                            Check solvency. Get Financially Well before continuing Goal Funding.
                        \end{itemize}
                    \item Goals Based: More Focused
                        \begin{itemize}
                            \item Start with fixed expenses
                            \item Add cost of goals
                            \item If surplus, add in Discretionary Spending
                        \end{itemize}
                    \item Only three things can be done with a Surplus:
                        \begin{enumerate}
                            \item Consume more today
                            \item Payoff previous borrowed consumption
                            \item Save to increase future consumption
                        \end{enumerate}
                    \item All budgets should be "Zero-Sum"
                        \begin{itemize}
                            \item Allocate every dollar! (Surplus should be "0")
                        \end{itemize}
                \end{itemize}
            \subsubsection{WHAT IF I NEED HELP BUDGETING?}
                \begin{itemize}
                    \item Download a template for excel
                    \item Make your own template
                    \item Help from bank?
                    \item Websites
                    \item Phone apps
                \end{itemize}

        \subsection{Ratios}
            \subsubsection{The 5 Basic Financial Ratios}
                \begin{enumerate}
                    \item Living Expenses Covered Ratio
                    \item Debt Ratio
                    \item Debt Service to Income Ratio
                    \item Saving Ratio
                    \item Investment Assets to Total Assets
                \end{enumerate}
            \subsubsection{Living Expenses Covered Ratio}
                (aka Basic Liquidity or Emergency Fund Ratio)
                \begin{itemize}
                    \item Monetary Assets / Monthly Living Expenditures
                    \item withstand income shock (recommended 3-6 months)
                \end{itemize}
            \subsubsection{Debt Ratio}
                \begin{itemize}
                    \item Total Liabilities / Total Assets
                    \item \underline{Solvency} (more than 1 means insolvent, ratio should go down with age)
                \end{itemize}
            \subsubsection{Debt Service to Income Ratio}
                \begin{itemize}
                    \item Annual Debt PMTs / Gross Annual Income
                    \item debt load (recommended: less than 36\%)
                \end{itemize}
            \subsubsection{Saving Ratio}
                \begin{itemize}
                    \item Annual Savings / Annual Living Expenses
                    \item degree of saving
                    \item ~10\% in 20s, 11-20\% in 30s, more than 30\% in 40s
                    \item Should go up with age
                \end{itemize}
            \subsubsection{Investment Assets to Total Assets Ratio}
                \begin{itemize}
                    \item (Investment Assets + Retirement Assets) / Total Assets
                    \item degree of wealth building
                    \item ~10\% in 20s, 11-20\% in 30s, more than 30\% in 40s
                    \item Should go up with age
                \end{itemize}
        
        \subsection{Taxation}
            \subsubsection{TAXES AND DECISION MAKING}
                \begin{itemize}
                    \item \textbf{Income Tax}
                        \begin{itemize}
                            \item U.S. Tax Structure
                            \item Effective (Average) Tax Rate
                            \item Marginal Tax Rate
                            \item 1040 Breakdown
                            \item Tax Credits vs Tax Deductions
                        \end{itemize}
                    \item \textbf{Capital Gains Tax}
                        \begin{itemize}
                            \item Short vs. Long
                        \end{itemize}
                    \item \textbf{Estate Tax}
                        \begin{itemize}
                            \item Lifetime "Bucket"
                        \end{itemize}
                    \item \textbf{Gift Tax}
                        \begin{itemize}
                            \item Annual Exclusion
                            \item Lifetime "Bucket"
                        \end{itemize}
                \end{itemize}
            \subsubsection{U.S. - PROGRESSIVE TAX SYSTEM}
                \begin{itemize}
                    \item Taxed at increasing rates
                    \item The wealthy can afford to pay more
                    \item The more you earn, the higher your marginal tax bracket
                \end{itemize}
            \subsubsection{EFFECTIVE OR AVERAGE TAX RATE}
                \begin{itemize}
                    \item \textbf{Taxes paid divided by Gross Income}
                    \item Example: John is single and made \$82,450 this year. He paid \$14,078.50 in taxes
                    \item 14,078.50/82,450 = 17\% effective or average rate.
                    \item Notice his top marginal rate is 22\%. He paid 22\% on all his income over \$38,700. On his first \$9,525
                        he only paid 10\%, then from \$9,526-\$38,700 he paid 12\%.
                    \item In the end he averaged 17\%
                    \item A person's average income tax rate is ALWAYS lower than their marginal income tax rate
                \end{itemize}
            \subsubsection{MARGINAL TAX}
                \begin{itemize}
                    \item This is the tax paid on the next dollar made
                    \item Useful when choosing between investments
                    \item Useful when trying to reduce taxes
                \end{itemize}
            \subsubsection{WHO MUST FILE FORM 709}
                \begin{itemize}
                    \item Everyone who makes a gift, unless gifts are:
                        \begin{itemize}
                            \item Under the annual Exclusion
                            \item Qualified transfers (hospitals, educational institutions)
                            \item Transfers to spouses (generally)
                            \item Transfers to charities
                        \end{itemize}
                    \item Remember - If gifts are split between spouses, there must be a tax return even if less than the annual exclusion
                \end{itemize}
            \subsubsection{INCOME TAX - THE RUNDOWN}
                \begin{itemize}
                    \item W4 selection determines your withholdings
                        \begin{itemize}
                            \item Your withholdings are your estimated annual tax per pay-period
                        \end{itemize}
                    \item When you file, you estimate your tax due \& weigh it against your withholdings
                        \begin{itemize}
                            \item More withholdings than taxes due = refund
                            \item More taxes due than withheld = tax obligation
                        \end{itemize}
                \end{itemize}
            \subsubsection{1040: BASIC INFORMATION \& FILING STATUS}
                \begin{enumerate}
                    \item \textbf{STEP 1: DETERMINE GROSS INCOME}
                        \begin{itemize}
                            \item Taxable income from all sources
                                \begin{itemize}
                                    \item Active income
                                    \item Portfolio or investment income
                                    \item Passive income
                                \end{itemize}
                        \end{itemize}
                    \item \textbf{STEP 2: CALCULATE AGI (ADJUSTED GROSS INCOME)}
                        \begin{itemize}
                            \item The government allows for some adjustments that you won't have to take into income
                            \item You subtract these adjustments from gross income to arrive at adjusted gross income
                        \end{itemize}
                    \item \textbf{STEP 3: SUBTRACT DEDUCTIONS}
                        WHAT CAN I ITEMIZE?
                        \begin{itemize}
                            \item Medical and dental expense (above 7.5\% AGI)
                            \item Tax expenses (real estate, state/local tax (limited to \$10,000 collectively))
                            \item Home interest expense (but not loans against equity)
                            \item Gifts to charity
                        \end{itemize}
                    \item \textbf{STEP 4: CALCULATE TAXABLE INCOME}
                        \begin{itemize}
                            \item Subtract your deductions from AGI to arrive at taxable
                        \end{itemize}
                    \item \textbf{STEP 5: SUBTRACT YOUR CREDIT/DETERMINE TAX}
                        \begin{itemize}
                            \item Credits reduce dollar for dollar
                            \item (Tax Credits are \underline{ALWAYS} more favorable than Tax deductions!)
                            \item Once you have subtracted credits you use this amount to determine your tax using the tax brackets
                            \item Some common tax credits for individuals include:
                                \begin{itemize}
                                    \item Child Tax Credit
                                    \item Earned Income Tax Credit
                                    \item Credit for Other Dependents
                                    \item Adoption Credit
                                    \item Low-Income Housing Credit
                                    \item Premium Tax Credit (Affordable Care Act)
                                    \item American Opportunity Credit
                                    \item Lifetime Learning Credit
                                \end{itemize}
                        \end{itemize}
                    \item \textbf{STEP 6: PAYA OR GET YOUR REFUND!}
                        \begin{itemize}
                            \item April 15th is tax day (unless on a weekend then it's the next Monday)
                            \item Extensions are available (you must file and pay for them)
                        \end{itemize}
                \end{enumerate}
                    

    \section {Unit 3: Credit Worthiness and Navigating Credit}

    \section{Unit 4: Borrowing for Large Purchases}

    \section{Unit 5: Investing and Human Capital}
    
    \section{Unit 6: Investment Products and Investment Strategy}

    \section{Unit 7: Risk Management and Estate Planning}
    
\end{document}