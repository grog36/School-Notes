% Gregory Ecklund
% September 2024
% CS-466: Information Security I

\documentclass[12pt]{article}
\usepackage{xcolor}
\usepackage{hyperref}

\title{Information Securit I Notes}
\author{Gregory Ecklund}
\date{\today}

\begin{document}
    \maketitle
    \newpage

    \section{Fundamental Security Design Principles}
        \begin{itemize}
            \item Economy of mechanism
                \begin{itemize}
                    \item Simple and as small as possible
                    \item Complex designs increase cost, time, testing needs and the likelihood
                    of an exploit or bug
                \end{itemize}
            \item Fail-safe defaults
                \begin{itemize}
                    \item Access decisions should be based on permission rather than exclusion
                    \item The default is lack of access with permissions added
                    \item This approach exhibits better fail-safe characteristics
                \end{itemize}
            \item Complete mediation
                \begin{itemize}
                    \item Every access must be checked against access control mechanism
                    \item Access permissions should not be derived from a cache
                    \item If the system depends on cached information it changes in authority will
                    be less immediate
                \end{itemize}
            \item Open design
                \begin{itemize}
                    \item An open design has a higher degree of professional scrutiny
                    \item Standardization also promotes system compatibility
                    \item Tried and tested is better than bespoke and unknown
                \end{itemize}
            \item Separation of privilege
                \begin{itemize}
                    \item Multiple privilege attributes are required to access a restricted resource
                    \item Also, high privilege operations in separate processes with a higher level of
                    privilege needed
                        \begin{itemize}
                            \item User
                            \item Super User
                            \item Administrator
                        \end{itemize}
                \end{itemize}
            \item Least privilege
                \begin{itemize}
                    \item Every process and user should operate using the least set of privileges to perform the task
                    \item Roles and associated privileges should be defined in security policies
                    \item Administrators with special privileges should be granted those 'special' privileges
                    only when needed
                        \begin{itemize}
                            \item Leaving special privileges on all the time increases risk
                        \end{itemize}
                \end{itemize}
            \item Least common mechanism
                \begin{itemize}
                    \item Minimize the amount of mechanism common to more than one user and depended on by all users
                    \item A shared mechanism is a potential information path between users and should be designed
                    with care
                    \item For example, the same authentication method should not be used to authenticate employers
                    of a company and non-employee users of its popular website
                \end{itemize}
            \item Psychological acceptability
                \begin{itemize}
                    \item Security methods must
                        \begin{itemize}
                            \item Be user friendly
                            \item Not interfere with work of users
                            \item Meet the needs of those who authorize access
                        \end{itemize}
                \end{itemize}
            \item Isolation
                \begin{itemize}
                    \item Access systems should be isolated (physically and logically) from critical resources to
                    prevent disclosure or harm caused by tampering
                    \item Processes and files of users should be isolated from each other
                \end{itemize}
            \item Encapsulation
                \begin{itemize}
                    \item A basic form of isolation rooted in object oriented programming practices
                        \begin{itemize}
                            \item Keeping class and object members private and hidden unless there is a reason not to
                            \item Use setter methods (mutators) to ensure fields set within valid, safe values
                            \item Each class having one well defined purpose
                        \end{itemize}
                    \item All of this prevents accidental or miss intended use of an application and its functions
                \end{itemize}
            \item Modularity
                \begin{itemize}
                    \item Security functions should embody a modular architecture for mechanism implementation
                    \item Common security modules such as cryptographic algorithms developed for reuse
                    \item Coding to an interface allows easy substitution for other implementations
                    \item Code for reuse. Less code repetition leads to a safer system, that is less prone to error
                    and is easier to test and maintain
                \end{itemize}
            \item Layering
                \begin{itemize}
                    \item Layering refers to multiple overlapping security protections
                    \item The failure or circumnavigation of one system will not leave the rest unprotected
                    \item \textbf{Layers:}
                        \begin{enumerate}
                            \item Physical Security
                            \item Identity and Access
                            \item Perimeter
                            \item Network
                            \item Compute
                            \item Application
                            \item Data
                        \end{enumerate}
                \end{itemize}
            \item Least astonishment
                \begin{itemize}
                    \item A program or user interface should always respond in a way least likely to astonish the user
                    \item For example, it should be transparent enough for the user to see how the security goals
                    map to the provided security mechanism
                    \item This principle can also be applied to code modules used by other developers
                \end{itemize}
        \end{itemize}
\end{document}