% Gregory Ecklund
% January 2024
% ND-190 Personal Nutrition Notes

\documentclass[12pt]{article}
\usepackage{xcolor}
\usepackage{hyperref}

\title{Personal Nutrition Notes}
\author{Gregory Ecklund}
\date{\today}

\begin{document}
    \maketitle
    \tableofcontents
    \newpage
    
    \section{Chapter 1: What is Nutrition?}

        \subsection{What Drives Our Food Choices?}
            \begin{itemize}
                \item We need to eat and drink to obtain:
                \begin{itemize}
                    \item \textbf{Nutrients:} chemical compounds in foods to provide fuel for energy, growth, and maintenance, and to regulate body processes
                        \begin{itemize}
                            \item Six classes:
                                \begin{itemize}
                                    \item Carbohydrates, fats, protein: provide energy in the form of \textbf{kilocalories}. (\textbf{Kilocalories = calories})
                                    \item Vitamins, minerals, water: help regulate many body processes, including \textbf{metabolism}
                                \end{itemize}
                        \end{itemize}
                    \item Food also provides nonnutrient compounds that contribute to health and may play a role in fighting chronic diseases
                \end{itemize}
                \item We choose foods for many other reasons beyond the basic need to obtain nutrients:
                    \begin{itemize}
                        \item Taste and Culture
                        \item Social reasons and trends
                        \item Cost, time, and convenience
                        \item Habits and emotions
                    \end{itemize}
            \end{itemize}

        \subsection{What is Nutrition and Why is Good Nutrition So Important?}
            \begin{itemize}
                \item \textbf{Nutrition:} the science that studies how nutrients and compounds in foods nourish and affect body functions and health
                \item Chronic deficiencies, excesses, and imbalances of nutrients can affect health
                \item Good nutrition plays a role in reducing the risk of many chronic diseases and conditions, including heart disease, cancer, and stroke
            \end{itemize}

        \subsection{What Are the Essential Nutrients and Why Do You Need Them?}
            \begin{itemize}
                \item The size classes of nutrients are all essential in the diet to maintain bodily functions
                \item \textbf{Macronutrients:} energy-yielding nutrients needed in higher amounts
                    \begin{itemize}
                        \item Carbohydrates, lipids (fats), and proteins
                    \end{itemize}
                \item \textbf{Micronutrients:} needed in smaller amounts
                    \begin{itemize}
                        \item Vitamins and minerals
                    \end{itemize}
                \item Water: copious amounts needed daily for hydration
                \item Carbohydrates, fats, and proteins
                    \begin{itemize}
                        \item Provide energy
                        \item One kilocalorie equals the amount of energy needed to raise the temperature of 1 kilogram of water 1 degree Celsius
                            \begin{itemize}
                                \item Carbohydrates and protein provide 4kcal/gram
                                \item Fats provide 9kcal/gram
                            \end{itemize}
                        \item Are \textbf{organic} compounds (contain Carbon atoms)
                        \item Also contain Hydrogen and Oxygen atoms
                        \item Proteins also contain Nitrogen atoms (unlike carbs and fats)
                    \end{itemize}
                \item Carbohydrates supply glucose, a major energy source
                \item Fats are another major fuel source and also:
                    \begin{itemize}
                        \item Cushion organs
                        \item Insulate body to maintain body temperature
                    \end{itemize}
                \item Proteins can provide energy but are better suited for:
                    \begin{itemize}
                        \item Growth and maintenance of muscle, tissues, organs
                        \item Making hormones, enzymes, healthy immune system
                        \item Transporting other nutrients
                    \end{itemize}
                \item To calculate the amount of energy a food provides:
                    \begin{itemize}
                        \item Multiply the total grams of a nutrient by the number of calories per gram
                            \begin{itemize}
                                \item 1 gram of carbohydrate or protein = 4 calories
                                \item 1 gram of fat = 9 calories
                            \end{itemize}
                    \end{itemize}
                \item Vitamins and minerals are essential for metabolism
                    \begin{itemize}
                        \item Many assist \textbf{enzymes} in speeding up chemical reactions in the body
                            \begin{itemize}
                                \item Example: B vitamins are coenzymes in carbohydrate and fat metabolism
                            \end{itemize}
                        \item Vitamins are organic compounds that usually have to be obtained from food
                        \item Minerals are inorganic substances
                            \begin{itemize}
                                \item Key roles in body processes and structures
                            \end{itemize}
                    \end{itemize}
                \item Water is vital for many processes in your body
                    \begin{itemize}
                        \item Part of fluid medium inside and outside of cells
                        \item Helps chemical reactions, such as those involved in energy production
                        \item Helps maintain body temperature
                        \item Key role in transporting nutrients and oxygen to cells and removing waste products
                        \item Lubricant for joints, eyes, mouth, intestinal tract
                        \item Protective cushion for organs
                    \end{itemize}
            \end{itemize}

        \subsection{How Should You Get These Important Nutrients?}
            \begin{itemize}
                \item The best way to meet your nutrient needs is with a well-balanced diet that includes essential nutrients from all six classes
                \item A well balanced diet will also include \textbf{fiber} and \textbf{phytochemicals}, which have been shown to help fight many diseases
                    \begin{itemize}
                        \item \textbf{Phytochemicals}
                            \begin{itemize}
                                \item Nonnutritive compounds in plants foods that may play a role in fighting chronic diseases
                            \end{itemize}
                        \item \textbf{Fiber}
                            \begin{itemize}
                                \item The portion of plant foods that isn't digested in the small intestine
                            \end{itemize}
                        \item Whole grains, fruits, and vegetables are rich sources
                    \end{itemize}
                \item A supplement can be beneficical:
                    \begin{itemize}
                        \item When nutrient needs are higher
                            \begin{itemize}
                                \item Example: pregnant women need an iron supplement to meet increased needs
                            \end{itemize}
                        \item When diet restrictions exist
                            \begin{itemize}
                                \item Example: lactose-intolerant individuals (difficulty digesting milk products) may choose a calcium supplement to help meet needs
                            \end{itemize}
                    \end{itemize}
                \item Well-balanced diet and supplements are not mutually exclusive; they can be partnered for good health
            \end{itemize}

        \subsection{How Does the Average American Diet Stack Up?}
            \begin{itemize}
                \item High in:
                    \begin{itemize}
                        \item Added sugar, sodium, saturated fat, calories
                            \begin{itemize}
                                \item Americans consume ~ 17 teaspoons of added sugar
                            \end{itemize}
                    \end{itemize}
                \item Low in:
                    \begin{itemize}
                        \item Vitamin D, calcium, potassium, fiber
                            \begin{itemize}
                                \item Iron: women fall short
                            \end{itemize}
                    \end{itemize}
                \item Lack of healthy diet may also be due to where we eat - Americans currently spend 40 percent of their food budget consuming food outside the home
                \item Incidence of overweight and obesity is on the rise
                \item Adults
                    \begin{itemize}
                        \item 70 percent are overweight and of those, approximately 40 percent are obese
                    \end{itemize}
                \item Children
                    \begin{itemize}
                        \item 16 percent of children ages 2-19 are overweight
                        \item 17 percent are considered obese
                    \end{itemize}
                \item High rates of overweight and obesity
                    \begin {itemize}
                        \item Obesity is ubiquitous, non-discriminatory
                    \end{itemize}
                \item Causes
                    \begin{itemize}
                        \item Consume more calories than needed
                        \item Burn fewer calories due to sedentary lifestyles
                    \end{itemize}
                \item Effects
                    \begin{itemize}
                        \item Increased rate of type 2 diabetes (especially children), heart disease, cancer, and stroke
                    \end{itemize}
                \item Improving Americans' diets is one goal of \textbf{Healthy People 2020}
                    \begin{itemize}
                        \item Disease prevention and health promotion objectives for Americans to meet in the second decade of twenty-first century
                        \item Focuses on several overarching goals:
                            \begin{itemize}
                                \item Attain high-quality, longer lives free of preventable disease, disability, injury, and premature death
                                \item Achieve healthy equity, eliminate disparities, and improve the health of all groups
                                \item Create social and physical environments that promote good health for all
                                \item Promote quality of life, healthy development, and healthy behaviors across every stage of life
                            \end{itemize}
                    \end{itemize}
            \end{itemize}

        \subsection{Poor, Obese, and Malnourished: A Troubling Paradox}
            \begin{itemize}
                \item Americans living near or below the poverty level have much higher rates of obesity than affluent Americans
                \item Children who are food insecure are more likely to be deficient in iron, have colds and headaches, have delayed cognitive development, and be at risk for behavioral problems
                \item Some factors that lead to obesity:
                    \begin{itemize}
                        \item Inconsistent meal patterns
                        \item Household stress
                        \item Limited access to supermarkets
                        \item Convenience stores and fast-food restaurants
                    \end{itemize}
            \end{itemize}

        \subsection{What's the Real Deal When It Comes to Nutrition Research and Advice}
            \begin{itemize}
                \item Newspaper headlines and television news items that report results of a single research study can be misleading
                \item In contrast, advice from authoritative health and nutrition organizations is based on:
                    \begin{itemize}
                        \item \textbf{Consensus:} the opinion of a group of experts based on collective information
                    \end{itemize}
            \end{itemize}

        \subsection{How Can I Evaluate Nutrition News?}
            \begin{itemize}
                \item Before making dietary and lifestyle changes based on media reports, read with a critical eye and ask:
                    \begin{itemize}
                        \item Was the research finding published in a peer-reviewed journal?
                        \item Was the study done using animals or humans?
                        \item Do the study participants resemble me?
                        \item Is this the first time I've heard about this?
                    \end{itemize}
                \item Wait until research findings are confirmed and consensus is reached by reputable health organizations before making changes
            \end{itemize}

        \subsection{Sound Nutrition Research Begins with the Scientific Method}
            \begin{itemize}
                \item \textbf{Scientific Method:} process used by scientists to generate sound research findings
                    \begin{enumerate}
                        \item Observe, ask questions, and formulate a \textbf{hypothesis} (idea based on observation)
                        \item Conduct an experiment to test the hypothesis
                        \item Share findings in a \textbf{peer-reviewed journal} (research publication for scientists)
                    \end{enumerate}
            \end{itemize}

        \subsection{Research Studies and Experiments Confirm Hypotheses}
            \begin{itemize}
                \item \textbf{Observational research:} involves looking at factors in two or more groups of subjects to see if there is a relationship to certain health outcomes
                    \begin{itemize}
                        \item \textbf{Epidemiological research:} study of populations of people
                            \begin{itemize}
                                \item Example: Relationship of sun exposure and incidence of rickets in Norway compared with Australia
                                    \begin{itemize}
                                        \item May be due to other unidentified diet or lifestyle factors
                                    \end{itemize}
                            \end{itemize}
                    \end{itemize}
                \item \textbf{Experimental research:} involves at least two groups of subjects
                    \begin{itemize}
                        \item \textbf{Experimental group:} given a specific treatment
                        \item \textbf{Control group:} given a placebo ("sugar pill")
                        \item Double-blind placebo-controlled experiments is "gold standard"
                            \begin{itemize}
                                \item Neither scientists nor subjects know which group is receiving which treatment
                                \item All variables held the same and controlled for both groups
                            \end{itemize}
                    \end{itemize}
            \end{itemize}

        \subsection{What is Nutritional Genomics?}
            \begin{itemize}
                \item Genes determine your inherited, specific traits
                    \begin{itemize}
                        \item With the completion of the \textbf{Human Genome Project}, the complete sequencing of \textbf{deoxyribonucleic acid (DNA)} in your cells is now known.
                        \item Your DNA contains the genetic instructions for making proteins that direct activities in the body
                    \end{itemize}
                \item \textbf{Nutritional genomics:} a field of study that researches the relationship between nutrition and genomics (the study of genes and \textbf{gene expression})
                    \begin{itemize}
                        \item Example: certain dietary components in foods can cause a different response in another person
                    \end{itemize}
            \end{itemize}

        \subsection{You Can Trust the Advice of Nutrition Experts}
            \begin{itemize}
                \item \textbf{Registered dietitian nutritionist (RDN):} completed at least a bachelor's degree at an accredited U.S. college or university and a supervised practice, and passed a national exam administered by the Academy of Nutrition and Dietetics
                    \begin{itemize}
                        \item They have an understanding of \textbf{medical nutrition therapy}, which is an integration of nutrition counseling and dietary changes based on an individual's medical history and current health needs to improve that person's health.
                    \end{itemize}
                \item \textbf{Public health nutritionist:} has a degree in nutrition but is not an RDN (if s/he didn't cokplete supervised practice, s/he is not eligible to take the AND exam)
                \item \textbf{Licensed dietitian nutritionist(LDN):} licensed by state licensing agencies
                \item Be wary of anyone who calls him/herself a \textbf{nutritionist}, as this is a generic term, which means s/he may have taken few or no accredited courses in nutrition
            \end{itemize}

        \subsection{You Can Obtain Accurate Nutrition Information on the Internet}
            \begin{itemize}
                \item National Institutes of Health (NIH) 10 questions to consider when viewing a health-related website:
                    \begin{enumerate}
                        \item Who runs the site?
                        \item Who pays for the site?
                        \item What is the purpose of the site?
                        \item Where does the information come from?
                        \item What is the basis of the information?
                        \item Is the information reviewed by experts?
                        \item How current is the information?
                        \item How does the site choose links to other sites?
                        \item How does the site collect and handle personal information? Is the site secure?
                        \item Can you communicate with the owner of the website?
                        \item Is it safe to link to Twitter or Facebook through a website?
                    \end{enumerate}
            \end{itemize}

        \subsection{Nutrion in the Real World: Don't Be Scammed!}
            \begin{itemize}
                \item Beware of health quackery and fraud
                    \begin{itemize}
                        \item To avoid falling for scams, watch for "red falgs" that try to convince you that:
                            \begin{itemize}
                                \item There is a quick fix for what ails you
                                \item Their product causes miraculous cures
                                \item One product does it all
                                \item You can lose weight in a short amount of time without dieting or exercise
                                \item Other folks are claiming that the product worked for them
                            \end{itemize}
                    \end{itemize}
                \item The FDA's health fraud website helps consumers identify scams and fraud (\href{https://www.consumer.ftc.gov/scams}{https://www.consumer.ftc.gov/scams})
            \end{itemize}

    \section{Chapter 2: Tools for Healthy Eating}

        \subsection{What is Healthy Eating and What Tools Can Help?}
            \begin{itemize}
                \item Key principles of healthy eating:
                    \begin{itemize}
                        \item Balance
                        \item Variety
                        \item Moderation
                    \end{itemize}
                \item \textbf{Undernutrition:} not meeting nutrient needs
                \item \textbf{Overnutrition:} excess nutrients and/or calories in diet
                \item \textbf{Malnourished:} long-term outcome of consuming a diet that doesn't meet nutrient needs
                    \begin{itemize}
                        \item Can result fromm both under and over nutrition
                    \end{itemize}
                \item Tools to help avoid under and over nutrition
                    \begin{enumerate}
                        \item Dietary Reference Intakes (DRIs)
                            \begin{itemize}
                                \item Nutrient recommendations
                            \end{itemize}
                        \item \textbf{Dietary Guidelines for Americans}
                            \begin{itemize}
                                \item General dietary and lifestyle advice
                            \end{itemize}
                        \item MyPlate
                            \begin{itemize}
                                \item Food recommendations based on DRIs and the advice from the \textbf{Dietary Guidelines}
                            \end{itemize}
                    \end{enumerate}
                    \begin{itemize}
                        \item \textbf{Nutrition Facts Panel} on food labels
                            \begin{itemize}
                                \item Contain Daily Values which can help you decide which foods to buy
                            \end{itemize}
                    \end{itemize}
            \end{itemize}
        
        \subsection{What Are the Dietary Reference Intakes?}
            \begin{itemize}
                \item DRIs tell you how much of each nutrient you need to consume to:
                    \begin{itemize}
                        \item Maintain good health
                        \item Prevent chronic diseases
                        \item Avoid unhealthy excesses
                    \end{itemize}
                \item Issued by U.S. National Academy of Sciences' Institute of Medicine
                \item Updated periodically based on latest scientific research
            \end{itemize}
        
        \subsection{DRIs Encompass Several Reference Values}
            \begin{itemize}
                \item \textbf{Estimated Average Requirement (EAR)}
                    \begin{itemize}
                        \item Average amount of a nutrient known to meet the needs of \textbf{50 percent of individuals} of same age and gender
                        \item Starting point for determining the other values
                    \end{itemize}
                \item \textbf{Recommended Dietary Allowance (RDA)}
                    \begin{itemize}
                        \item Based on the EAR, but set higher
                        \item Average amount of a nutrient that meets the needs of \textbf{nearly all individuals (97 to 98 percent)}
                    \end{itemize}
                \item \textbf{Adequate Intake (AI)}
                    \begin{itemize}
                        \item Used if scientific data to determine EAR and RDA are insufficient
                        \item Next best estimate of amount of nutrients needed to maintain good health
                    \end{itemize}
                \item \textbf{Tolerable Upper Intake Level (UL)}
                    \begin{itemize}
                        \item Highest amount of nutrient that is unlikely to cause harm if consumed daily
                        \item Consuming amount higher than the UL daily may cause \textbf{toxicity}
                    \end{itemize}
                \item \textbf{Acceptable Macronutrient Distribution Range (AMDR)}
                    \begin{itemize}
                        \item Recommended range of intake for energy-containing nutrients
                            \begin{itemize}
                                \item Carbohydrates: 45 to 65 percent of daily caloric intake
                                \item Fat: 20 to 35 percent of daily caloric intake
                                \item Proteins: 10 to 35 percent of daily caloric intake
                            \end{itemize}
                    \end{itemize}
                \item \textbf{Estimated Energy Requirement (EER)}
                    \begin{itemize}
                        \item Amount of daily energy needed to maintain healthy body weight and meet energy needs
                        \item Different approach than RDAs or AIs
                        \item Takes into account age, gender, height, weight, and activity level
                    \end{itemize}
            \end{itemize}
        
        \subsection {Table 2.1 How Many Calories Do You Need Daily? TODO}
        %TODO
        \subsection{How to Use the DRIs}
            \begin{itemize}
                \item To plan a quality diet and make healthy food choices
                    \begin{itemize}
                        \item Goal:
                            \begin{itemize}
                                \item Meet the RDA or AI for all nutrients
                                \item Do not exceed the UL
                                \item Consume the energy-yielding nutrients within the ranges of the AMDR
                            \end{itemize}
                    \end{itemize}
            \end{itemize}
        
        \subsection{What Are the Dietary Guidelines for Americans}
            \begin{itemize}
                \item \textbf{The Dietary Guidelines for Americans reflect the most current} nutrition and physical activity recommendations.
                    \begin{itemize}
                        \item Set by the USDA and Department of Health and Human Services
                        \item Updated every five years
                        \item Allow healthy individuals over the age of 2 to maintain good health and prevent chronic disease
                    \end{itemize}
            \end{itemize}

        \subsection{Dietary Guidelines for Americans at a Glance}
            \begin{itemize}
                \item Five overarching guidelines
                    \begin{enumerate}
                        \item Follow a healthy eating pattern across the lifespan
                            \begin{itemize}
                                \item An eating pattern is the combination of foods and beverages that constitutes an individual's complete dietary intake over time
                            \end{itemize}
                        \item Focus on variety, nutrient density, and amount
                        \item Limit calories from added sugars and saturated fats and reduce sodium intake
                        \item Shift to healthier food and beverage choices
                        \item Support healthy eating patterns for all
                    \end{enumerate}
            \end{itemize}

        \subsection{What Are MyPlate and myplate.gov}
            \begin{itemize}
                \item \textbf{MyPlate} is the most recent food guidance system for Americans, released by the USDA in 2011
                    \begin{itemize}
                        \item Part of Web-based initiative, \href{https://www.myplate.gov}{https://www.myplate.gov}
                        \item Shows variety of food groups
                    \end{itemize}
                \item Food \textbf{guidance systems:} visual diagrams providing variety of food recommendations to help create a well-balanced diet
                    \begin{itemize}
                        \item Many countries create these based on their food supply, cultural food preferences, and the nutritional needs of their population
                    \end{itemize}
            \end{itemize}
        
        \subsection{MyPlate Emphasizes Changes in Diet, Eating Behaviors, and Phyiscal Activity}
            \begin{itemize}
                \item Promotes the concept of meal planning, healthful choices, proportionality, and moderation when planning a healthful diet
                \item \textbf{Proportionality:} the relationship of one entity to another
                    \begin{itemize}
                        \item Shown by the five food groups and choices should be nutrient-dense
                        \item Half your plate should be vegetables and fruits
                        \item Smaller portion for grains (preferably whole grains)
                        \item Lean protein foods
                        \item Fat-free and low-fat dairy foods
                    \end{itemize}
                \item \textbf{Nutrient density} refers to the amount of nutrients a food contains in relationship to the number of calories it contains
                    \begin{itemize}
                        \item Provide more nutrients per calorie
                            \begin{itemize}
                                \item More nutrients per bite
                            \end{itemize}
                        \item Little solid fats and added sugars
                    \end{itemize}
                \item \textbf{Energy density} refers to foods that are high in energy (calories) but low in weight or volume; more calories per gram
                    \begin{itemize}
                        \item Example: a potato chip is energy dense, while a baked potato is nutrient dense
                            \begin{itemize}
                                \item Processing removes Potassium and adds Sodium
                            \end{itemize}
                    \end{itemize}
            \end{itemize}

        \subsection{Table 2.3 How Much Should You Eat from Each Food Group TODO}
        %TODO (ADD TABLE)
            \begin{itemize}
                \item \textbf{Grains:} Includes all foods made with wheat, rice, oats, cornmeal, or barley, such as bread, pasta, oatmeal, breakfast cereals, tortillas, and grits. In general, 1 slice of bread, 1 cup of ready-to-eat-cereal, or 1/2 cup of cooked rice, pasta, or cooked cereal is considered 1 ounce equivalent(oz eq) from the grains group. \textbf{At least half of all grains consumed should be whole grains such as whole-wheat bread, oats, or brown rice.}
                \item \textbf{Protein:} In general, 1 ounce of lean meat, poultry, or fish, 1 egg, 1 tablespoon peanut butter, 1/4 cup cooked dry beans, or 1/2 ounce of nuts or seeds is considered 1 ounce equivalent (oz eq) from the protein foods group.
                \item \textbf{Dairy:} Includes all fat-free and low-fat milk, yogurt, and cheese. In general, 1 cup of milk or yogurt 1.5 ounces of natural cheese, or 2 ounces of processed cheese is considered 1 cup from the dairy group.
                \item \textbf{Oil:} Includes vegetable oils such as canola, corn, olive, soybean, and sunflower oil, fatty fish, nuts, avocados, mayonnaise, salad dressings made with oils, and soft margarine. (Oils are not considered a food group but should be added to your diet for good health)
            \end{itemize}
        
        \subsection{Nutrition in the real World: When a Portion Isn't a Portion}
            \begin{itemize}
                \item A \textbf{portion} is the amount of food eaten in one sitting
                \item The FDA defines \textbf{serving size} as a standard amount of food that is customarily consumed
                    \begin{itemize}
                        \item Standardizing allows for consistency and helps consumer get a ballpark idea of what a typical serving should be
                    \end{itemize}
                \item The restaurant industry has appealed to your desire to get the most food for your money by expanding restaurant portion sizes, especially of inexpensive foods, such as fast foods
            \end{itemize}

        \subsection{Examining the Evidence: Does the Time of Day You Eat Impact Your Health?}
            \begin{itemize}
                \item Eating breakfast means more energy and fewer calories throughout the day
                    \begin{itemize}
                        \item Skipping breakfast may reduce nutrient quality of your diet
                    \end{itemize}
                \item Snacking associated with consumption of excess calories and obesity
                \item Eating more during evenings and weekends can lead to overconsumption of calories
                \item Recommendations:
                    \begin{itemize}
                        \item Start your day with nutrient-rich breakfast
                        \item Choose breakfast foods that are satisfying and improve appetite control throughout the day
                        \item Control calorie intake on nights and weekends
                    \end{itemize}
            \end{itemize}
            
        \subsection{What Is a Food Label and Why Is It Important?}
            \begin{itemize}
                \item The food label tells you what's in the package
                    \begin{itemize}
                        \item To help consumers make informed food choices
                    \end{itemize}
                \item Food and Drug Administration (FDA) mandates that every packaged food be labeled with:
                    \begin{itemize}
                        \item Name of the food
                        \item Net weight
                        \item Name and address of manufacturer or distributor
                        \item List of ingridients in descending order by weight
                        \item \textbf{Nutrition Facts Panel}
                    \end{itemize}
                \item The \textbf{Nutrition Facts Label} must also contain:
                    \begin{itemize}
                        \item Serving sizes that are uniform among similar products
                        \item How a serving of food fits into an overall daily diet
                        \item Uniform definitions for descriptive label's terms such as "fat-free" and "light"
                        \item Health claims that are accurate and science-based
                        \item Presence of any of the eight common allergens
                    \end{itemize}
                \item Food exempt from nutrition labeling:
                    \begin{itemize}
                        \item Plain coffee/tea, spices, flavorings, bakery foods, ready-to-eat foods prepared and sold in restaurants or produced by small businesses
                    \end{itemize}
                \item The food label can help you make healthy food choices
                \item \textbf{Nutrition Facts Panel:} area on food label that provides uniform listing of specific nutrients obtained in one serving
                    \begin{itemize}
                        \item Calories
                        \item Total fat, saturated fat, and \textbf{trans} fat
                        \item Cholesterol
                        \item Sodium
                        \item Total carbohydrate, dietary fiber, and added sugars
                        \item Protein
                        \item Vitamin D, Calcium, Iron, and Potassium
                    \end{itemize}
                \item \textbf{Daily Values (DVs):} established reference levels of nutrients, based on 2,000-calorie diet, listed on the food label
                    \begin{itemize}
                        \item Given as percentages
                        \item High: 20 percent or more of the DV
                        \item Low: 5 percent or less of the DV
                    \end{itemize}
                \item There are no DVs listed on the label for \textbf{trans} fat, total sugars, and protein
                \item Three types of label claims:
                    \begin{enumerate}
                        \item \textbf{Nutrient Content Claims:} describe the level or amount of a nutrient in food product
                        \item \textbf{Health Claims:} describe a relationship between a food or dietary compound and a disease or health-related condition
                            \begin{itemize}
                                \item Example: Soluble fiber that naturally occurs in oats has been shown to lower blood cholesterol levels, which can help reduce the risk of heart disease.
                            \end{itemize}
                        \item \textbf{Structure/function Claims:} describe how a nutrient or dietary compound affects the structure or function of the human body
                    \end{enumerate}
            \end{itemize}
        
        \subsection{Functional Foods: What Role Do They Play in Your Diet?}
            \begin{itemize}
                \item \textbf{Functional Foods:} have a positive effect on health beyond providing basic nutrients
                    \begin{itemize}
                        \item Example: broccoli contains beta-carotene, a plant chemical called a \textbf{phytochemical}, that protects cells from damaging substances that increase risk of chronic diseases (heart disease, cancer)
                        \item \textbf{Zoochemicals:} compounds in animal food products that benefit health
                            \begin{itemize}
                                \item Example: omega-3 fatty acids in fatty fish
                            \end{itemize}
                        \item Manufacturers also fortify food products with phytochemicals or zoochemicals
                    \end{itemize}
                \item Global market of functional food will reach 255 billion dollars by 2024
                    \begin{itemize}
                        \item Health Connection: Functional Foods and Cholesterol
                    \end{itemize}
                \item Benefits of functional foods:
                    \begin{itemize}
                        \item Economical way for health professionals to treat chronic disease
                            \begin{itemize}
                                \item Example: cholesterol-lowering oats and/or plant sterols may be preferable to drugs
                            \end{itemize}
                    \end{itemize}
                \item Concerns with functional foods:
                    \begin{itemize}
                        \item Confusion over claims
                        \item Excess consumption may cause problems
                    \end{itemize}
                \item How to use functional foods:
                    \begin{itemize}
                        \item Consume narturally occurring phytochemicals and zoochemicals
                        \item Don't overconsume packaged functional foods
                        \item Get advice from a registered dietitian nutritionist (RDN)
                    \end{itemize}
            \end{itemize}

    \section{Chapter 3: The Basics of Digestion}
        \subsection{What is Digestion and Why Is It Important?}
            \begin{itemize}
                \item \textbf{Digestive process:} a multi-step process of breaking down foods into absorbable components using mechanical and chemical means in the \textbf{gastrointestinal (GI) tract}
                \item The GI tract is 30 feet long and the cells lining it function for 3-5 days, then shed into the \textbf{lumen} (interior of the intestinal tract).
                \item Gastrointestinal tract consists of:
                    \begin{itemize}
                        \item Mouth
                        \item Esophagus
                        \item Stomach
                        \item Small and large intestines
                        \item Accessory organs: pancreas, liver, gallbladder
                    \end{itemize}
                \item Main roles of the GI tract are to:
                    \begin{itemize}
                        \item Break down food into smallest components
                        \item Absorb nutrients
                        \item Prevent microorganisms or other harmful compounds in food from entering tissues of the body
                    \end{itemize}
            \end{itemize}
        
        \subsection{Digest Is Mechanical and Chemical}
            \begin{itemize}
                \item \textbf{Mechanical digestion:} chewing, grinding food to aid swallowing
                \item \textbf{Chemical digestion:} digestive juices and enzymes break down food into absorbable nutrients
                    \begin{itemize}
                        \item \textbf{Peristalsis:} the forward, rhythmic motion that moves chyme through digestive system
                        \item \textbf{Segmentation:} ("sloshing motion") mixes chyme with chemical secretions; increases time food comes into contact with intestinal walls
                        \item \textbf{Pendular movement:} (constructive wave) enhances nutrient absorption in small intestine
                    \end{itemize}
                \item \textbf{Peristalsis} helps mix food with digestive secretions and propels the mixture, called a \textbf{bolus}, from the esophagus through the large intestine.
                \item The three types of mechanical digestion move partially-digested, semi-liquid food mass, called chyme, at a rate of 1cm per minute.
                \item Total contact time in small intestine: 3 to 6 hours, depending on amount and type of food
            \end{itemize}
        
        \subsection{Nutrition in the Real World: Tinkering with your Body's Digestive Process}
            \begin{itemize}
                \item Many people consider using weight loss aids
                \item Alli is a popular aid in drugstores
                    \begin{itemize}
                        \item First FDA-approved, over-the-counter drug containing orlistat
                        \item Blocks the absorption of about 25 percent of the fat in a meal by preventing lipase enzyme from breaking down dietary fat
                        \item Can experience unpleasant side effects - bathroom urgency, fatty/oily stools, frequent bowel movements
                            \begin{itemize}
                                \item Symptoms are less present with lower fat meals
                            \end{itemize}
                        \item Need for vitamin supplement with the fat-soluble vitamins A,D,E,K and the antioxidant beta-carotene
                    \end{itemize}
            \end{itemize}

        \subsection{What Are the Organs of the GI Tract and Why Are They Important?}
            \begin{itemize}
                \item Both mechanical and chemical digestion begin in the mouth
                    \begin{itemize}
                        \item \textbf{Saliva} released: contains wayer, electrolytes, \textbf{mucus}, and a few enzymes
                            \begin{itemize}
                                \item Softens, lubricates, dissolves food particles
                            \end{itemize}
                        \item \textbf{Bolus} (food mass) moves into pharynx, is swallowed, and enters the esophagus
                        \item \textbf{Epiglottis} closes off trachea during swallowing to prevent food from lodging in the windpipe
                    \end{itemize}
                \item Once swalloed, a bolus is pushed down the \textbf{esophagus} by \textbf{peristalsis} into the stomach
                \item \textbf{Gastoesophageal sphincter} (lower esophageal sphincter (LES)): bottom of esophagus narrows and relaxes to allow food into stomach
                \item The gastoesophageal sphincter then closes to prevent backflow of \textbf{hydroclhoric acid (HCL)} from \textbf{stomach}
                    \begin{itemize}
                        \item Gastoesophageal reflux disease (GERD) is "reflux" of stomach acid that causes "heartburn" (irritation of esophagus lining)
                    \end{itemize}
            \end{itemize}

        \subsection{The Stomach Stores, Mixes, and Prepares Food for Digestion}
            \begin{itemize}
                \item The \textbf{stomach} is a muscular organ that continues mechanical digestions by churning and contracting to mix food with digestive juices for several hours
                \item \textbf{Stomach} produces powerful digestive secretions:
                    \begin{itemize}
                        \item \textbf{HCl:} activates enzyme \textbf{pepsin}, enhances absorption of minerals, breaks down connective tissue of meat
                            \begin{itemize}
                                \item Mucus protects stomach lining from damage
                            \end{itemize}
                        \item Digestive enzymes, intrinsic factor (for vitamin B12 absorption), stomach hormone \textbf{gastrin}
                    \end{itemize}
                \item \textbf{Bolus} becomes \textbf{chyme}, semiliquid substance of partially digested food and digestive juices
                \item \textbf{Gastrin:} hormone stimulates digestive activities and increases gastric motility and emptying
                \item Liquids, carbohydrates, low-fiber, and low-calorie foods exit stomach faster
                \item High-fiber, high-fat, and high-protein foods exit slower, keep you feeling full longer
                \item \textbf{Pyloric sphincter:} located between the stomach and small intestine; allows about 1tsp of chyme to enter the small intestine every 30 seconds
                    \begin{itemize}
                        \item Prevents backflow of intestinal contents
                    \end{itemize}
            \end{itemize}

        \subsection{Most Digestion and Absorption Occurs in the Small Intestine}
            \begin{itemize}
                \item \textbf{Small intestine:} long, narrow, coiled
                    \begin{itemize}
                        \item Three segments:
                            \begin{enumerate}
                                \item Duodenum (shortest segment)
                                \item Jejunum
                                \item Ileum (longest segment)
                            \end{enumerate}
                        \item In total the small intestine accounts for about 20 feet of the GI tract
                        \item Interior surface area tremendously increased by \textbf{villi, microvilli}, circular folds
                    \end{itemize}
            \end{itemize}

        \subsection{Large Intestine Eliminates Waste and Absorbs Water and Some Nutrients}
            \begin{itemize}
                \item \textbf{Ileocecal sphincter:} prevents backflow of fecal matter into ileum as chyme enters the \textbf{large intestine}
                \item Most nutrients in chyme have been absorbed when it reaches large intestine
                \item \textbf{Large intestine} has three sections: cecum, colon, rectum
                    \begin{itemize}
                        \item About 5 feet long, 2.5 inches in diameter (twice the diameter of the small intestine)
                        \item Absorbs water and electrolytes
                        \item No digestive enzymes; chemical digestion done by bacteria
                    \end{itemize}
                \item Intestinal matter passes through colon in 12 to 24 hours depending on age, health, diet, fiber intake
                    \begin{itemize}
                        \item Bacteria in colon produce vitamin K and biotin and break down fiber and undigested carbohydrates, producing methane, carbon dioxide, hydrogen gas, and other compounds
                    \end{itemize}
                \item \textbf{Stool} or \textbf{feces}, is stored in the \textbf{rectum}
                \item \textbf{Anus} is connected to the rectum and controlled by two sphincters: internal and external
                    \begin{itemize}
                        \item Final stage of defecation is under voluntary control
                            \begin{itemize}
                                \item Influenced by age, diet, prescription medicines, health, and abdominal muscle tone
                            \end{itemize}
                    \end{itemize}
            \end{itemize}
        
        \subsection{The Liver, Gallbladder, and Pancreas Are Accessory Organs}
            \begin{itemize}
                \item \textbf{Liver:} largest internal organ of the body
                    \begin{itemize}
                        \item Produces bile needed for fat digestion
                        \item Metabolism of carbohydrates, fats, and protein
                        \item Stores nutrients: vitamins A, D, B12, E; copper; iron; gylocogen (glucose storage form)
                        \item Detoxifies alcohol
                    \end{itemize}
                \item \textbf{Gallbladder:} concentrates and stores bile
                    \begin{itemize}
                        \item Released into GI tract when fat is ingested
                    \end{itemize}
                \item \textbf{Pancreas}
                    \begin{itemize}
                        \item Produces hormones: insulin and glucagon regulate blood glucose
                        \item Produces and secretes digestive enzymes and bicarbonate
                            \begin{itemize}
                                \item Bicarbonate neutralizes acidic chyme, protects enzymes from inactivation by acid
                            \end{itemize}
                    \end{itemize}
            \end{itemize}
        
        \subsection{How Do Hormones, Enzymes, and Bile Aid Digestion?}
            \begin{itemize}
                \item \textbf{Hormones} are released from endocrine glands throughout the lining of the stomach and small intestine and regulate digestion
                \item They control digestive secretions and regulate \textbf{enzymes} and cellular activity
                \item \textbf{Enzymes} are substances that produce chemical changes or catalyze chemical reactions
                \item Gastrin stimulates stomach to release HCl and enzyme \textbf{gastric lipase}
                \item \textbf{Ghrelin} stimulates hunger
                \item Secretin causes pancreas to release \textbf{bicarbonate} to neutralize HCl
                \item \textbf{Cholecystokinin} stimulates pancreas to secrete digestive enzymes, controls pace of digestion
                \item \textbf{Enzymes} drive process of digestion
                    \begin{itemize}
                        \item Speed up chemical reactions that break down food into absorbably nutrient components
                        \item Secreted by salivary glands, stomach, pancreas, and small intestine
                        \item Enzymes from pancreas are responsible for large portion of digested nutrients
                            \begin{itemize}
                                \item \textbf{Amylase:} digests carbohydrates
                                \item \textbf{Lipase:} digests fats
                                \item \textbf{Trypsin, chymotrypsin, and carboxypeptidase:} digest protein
                            \end{itemize}
                    \end{itemize}
                \item \textbf{Bile} helps digest fat
                    \begin{itemize}
                        \item Yellowish-green substance made in liver and stored in gallbladder
                        \item Breaks down large fat globules into smaller fat droplets
                        \item Can be reused by recycling through liver
                            \begin{itemize}
                                \item Bile is recycled back to the liver from the large intestine through \textbf{enterohepatic} (\textbf{entero} = intestine, \textbf{hapatic} = liver) \textbf{circulation}
                            \end{itemize}
                    \end{itemize}
            \end{itemize}

        \subsection{How Are Digested Nutrients Absorbed?}
            \begin{itemize}
                \item After digestion, nutrients that have completely broken down are absorbed and move into the tissues
                \item \textbf{Absorption} of nutrients through the walls of the intestines go into the body's two transport systems:
                    \begin{enumerate}
                        \item Circulatory system (blood)
                        \item Lumphatic system
                    \end{enumerate}
                \item Sent to the liver for processing before delivery to the body's cells
                \item GI tract is highly efficient: 92 to 97 percent of nutrients from food are digested and absorbed
                \item Nutrients absorbed by three methods:
                    \begin{enumerate}
                        \item \textbf{Passive diffusion:} nutrients pass through the cell membrane due to concentration gradient
                            \begin{itemize}
                                \item When concentration in GI tract exceeds that of intestinal cell, nutrient is forced across cell membrane
                            \end{itemize}
                        \item \textbf{Facilitated diffusion:} similar to passive method, but requires specialized protein to carry nutrients
                        \item \textbf{Active transport:} differs from other two methods
                            \begin{itemize}
                                \item Nutrients absorbed from low to high concentration, requiring both carrier and energy
                            \end{itemize}
                    \end{enumerate}
            \end{itemize}

        \subsection{What Happens to Nutrients After They Are Absorbed?}
            \begin{itemize}
                \item Circulatory system distributes nutrients through your blood
                    \begin{itemize}
                        \item Water-soluble nutrients
                    \end{itemize}
                \item Lymphatic system distributes some nutrients through your lymph vessels
                    \begin{itemize}
                        \item Fat-soluble nutrients
                    \end{itemize}
                \item Your body can store some surplus nutrients
                \item Excretory system passes waste out of the body
            \end{itemize}
    
        \subsection{What Other Body Systems Affect Your Use of Nutrients?}
            \begin{itemize}
                \item Nervous system stimulates your appetite
                    \begin{itemize}
                        \item Hormone ghrelin signals your brain to eat
                    \end{itemize}
                \item Endocrine system releases hormones that help regulate the use of absorbed nutrients
                    \begin{itemize}
                        \item Insulin and glucagon help regulate blood levels of glucose
                    \end{itemize}
            \end{itemize}

        \subsection{What Are Some Common Digestive Disorders?}
            \begin{itemize}
                \item Disorders of the mouth and esophagus:
                    \begin{itemize}
                        \item Gingivitis and periodontal disease
                    \end{itemize}
                \item Swallowing problems
                    \begin{itemize}
                        \item Dysphagia: difficulty swallowing
                    \end{itemize}
                \item Esophageal problems
                    \begin{itemize}
                        \item Heartburn (acid reflux) may be caused by weak lower esophageal sphincter (LES)
                            \begin{itemize}
                                \item Chronic heartburn can be a symptom of \textbf{gastroesophageal disease (GERD)}
                                \item Certain foods and behaviors (smoking, drinking alcohol, reclining after eating, large evening meals) may worsen condition
                            \end{itemize}
                    \end{itemize}
                \item Disorders of the stomach:
                    \begin{itemize}
                        \item \textbf{Gastroenteritis:} stomach flu, caused by virus or bacteria
                        \item \textbf{Peptic ulcers:} sore or erosion caused by drugs, alcohol, or bacteria
                    \end{itemize}
                \item Gallbladder disease:
                    \begin{itemize}
                        \item \textbf{Gallstones:} small, hard, crystalline, structures
                            \begin{itemize}
                                \item May require surgery
                            \end{itemize}
                    \end{itemize}
                \item Disorders of the intestines:
                    \begin{itemize}
                        \item \textbf{Flatulence:} release of intestinal gas from the rectum
                        \item \textbf{Constipation} and \textbf{diarrhea}
                            \begin{itemize}
                                \item \textbf{Constipation} often due to insufficient fiber or water intake
                                \item \textbf{Diarrhea} causes loss of fluids and electrolytes; serious if lasting for extended period
                            \end{itemize}
                        \item \textbf{Hemorrhoids:} swelling and inflammation in veins of rectum and anus
                    \end{itemize}
                \item More serious intestinal disorders:
                    \begin{itemize}
                        \item \textbf{Irritable Bowel Syndrome (IBS):} functional disorder involving changes in colon rhythm
                        \item \textbf{Celiac disease:} autoimmune, genetic disorder related to gluten consumption
                        \item \textbf{Colon cancer:} one of the leading forms of cancer, but curable if detected early
                    \end{itemize}
            \end{itemize}

    \section{Chapter 4: Carbohydrates - Sugars, Starches, and Fiber}

        \subsection{What Are Carbohydrates and Why Do You Need Them?}
            \begin{itemize}
                \item Found primarily in plant-based foods
                    \begin{itemize}
                        \item Grains, vegetables, fruits, nuts, legumes
                        \item Carbohydrate-based foods are staples in numerous cultures around the world
                    \end{itemize}
                \item Most desirable form of energy for body
                    \begin{itemize}
                        \item \textbf{Glucose}
                        \item Brain and red blood cells especially rely on glucose for fuel source
                    \end{itemize}
                \item Plants convert the sun's energy into glucose by \textbf{photosynthesis}
                \item During \textbf{photosynthesis}, plants use the chlorophyll in their leaves to absorb the energy in sunlight.
                \item \textbf{Glucose} is the most abundant carbohydrate in nature
                    \begin{itemize}
                        \item Used as energy by plants or combined with minerals from soil to make other compounds, such as protein and vitamins
                        \item Glucose units are linked together and stored in form of starch
                    \end{itemize}
                \item Two categories, simple and complex, based on number of units joined together
                \item \textbf{Simple carbohydrates} contain one or two sugar units: \textbf{monosaccharides} and \textbf{disaccharides}
                    \begin{itemize}
                        \item There are three \textbf{monosaccharides: glucose, fructose, galactose}
                        \item There are three \textbf{disaccharides:} two monosaccharides joined together
                            \begin{enumerate}
                                \item \textbf{Maltose} = glucose + glucose
                                \item \textbf{Sucrose} (table sugar) = glucose + fructose
                                \item \textbf{Lactose} (milk sugar) = glucose + galactose
                            \end{enumerate}
                    \end{itemize}
                \item Complex carbohydrates: \textbf{polysaccharides}
                    \begin{itemize}
                        \item Long chains and branches of sugars linked together
                        \item Starch, fiber, and glycogen
                    \end{itemize}
                \item \textbf{Starch} is the storage form in plants
                    \begin{itemize}
                        \item Amylose: straight chains of glucose units
                        \item Amylopectin: branched chains of glucose units
                    \end{itemize}
                \item \textbf{Fiber} is a nondigestible polysaccharide
                    \begin{itemize}
                        \item Examples: cellulose, hemicellulose, lignins, gums, pectin
                        \item Humans lack digestive enzyme needed to break down fiber
                        \item \textbf{Dietary fiber:} naturally found in foods
                        \item \textbf{Functional fiber:} added to food for beneficial effect
                            \begin{itemize}
                                \item Example: psyllium added to cereals
                            \end{itemize}
                        \item Total fiber = dietary fiber + functional fiber
                    \end{itemize}
                \item Fiber is also sometimes classified by its affinity for water
                    \begin{itemize}
                        \item \textbf{Soluble fiber:} dissolves in water and is fermented by intestinal bacteria
                            \begin{itemize}
                                \item Many are viscous, have thickening properties
                                \item Move more slowly through GI tract
                                \item Examples: pectin in fruits and vegetables, beta-glucan in oats and barley, gums in legumes, psyllium
                            \end{itemize}
                        \item \textbf{Insoluble fiber:} cellulose, hemicellulose, lignins
                            \begin{itemize}
                                \item Moves more rapidly through GI tract, laxative effect
                                \item Examples: bran of whole grains, seeds, fruits, vegetables
                            \end{itemize}
                    \end{itemize}
                \item \textbf{Glycogen} is the storage form of glucose in animals
                    \begin{itemize}
                        \item Branched glucose similar to amylopectin
                        \item Stored in liver and muscle cells
                            \begin{itemize}
                                \item Only limited amounts
                            \end{itemize}
                        \item Glycogen stored in animals breaks down when the animal dies, so these carbohydrates are not accessible for humans
                    \end{itemize}
            \end{itemize}

        \subsection{What Happens to the Carbohydrates You Eat?}
            \begin{itemize}
                \item You digest carbohydrates in your mouth and intestines
                    \begin{itemize}
                        \item Saliva contains amylase enzyme, which starts breaking down amylose and amylopectin into smaller starch units and maltose
                        \item In small intestine, pancreatic amylase breaks down remaining starch into maltose
                        \item Maltose and other disaccharides are broken down to monosaccharides and absorbed into blood
                        \item Fiber continues to the large intestine, where some is metabolized by bacteria in the colon and the majority elminiated in your stool
                    \end{itemize}
            \end{itemize}

        \subsection{What Is Lactose Maldigestion and Lactose Intolerance?}
            \begin{itemize}
                \item \textbf{Lactose:} principal carbohydrate (disaccharide) found in dairy products
                    \begin{itemize}
                        \item People with a deficiency of the enzyme \textbf{lactase} cannot digest lactose properly
                        \item \textbf{Lactose malabsorption} is natural part of aging
                            \begin{itemize}
                                \item People with lactose malabsorption can still consume dairy and should not eliminate it from their diets.
                            \end{itemize}
                        \item \textbf{Lactose intolerance:} when lactose maldigestion results in nausea, cramps, bloating, diarrhea, and flatulence within two hours of eating or drinking foods containing lactose
                    \end{itemize}
                \item Tips for tolerating lactose:
                    \begin{enumerate}
                        \item Gradually add dairy products to your diet
                        \item Eat smaller amounts throughout day rather than large amount at one time
                        \item Eat dairy foods with a meal or snack
                        \item Try reduced-lactose milk and dairy products
                        \item Consume lactase pills with lactose-laden meals or snacks
                    \end{enumerate}
            
            \end{itemize}

        \subsection{How Does Your Body Use Carbohydrates?}
            \begin{itemize}
                \item Glucose supplies energy to the body
                \item \textbf{Hormones} regulate the amount of glucose in your blood
                \item The hormone \textbf{insulin} is released from the pancreas and regulates glucose in your blood
                \item Insulin is released by the pancreas in response to rising blood glucose levels after a meal that contains carbohydrates
                    \begin{itemize}
                        \item Directs conversion of glucose in excess of immediate energy needs into glycogen (\textbf{glycogenesis}) in liver and muslce cells (limited capacity)
                        \item Rest of excess glucose converted to fat
                    \end{itemize}
            \end{itemize}

        \subsection{Carbohydrates Fuel Your Body between Meals}
            \begin{itemize}
                \item When blood glucose beings to drop, pancreas releases the hormone \textbf{glucagon} to raise blood glucose levels
                    \begin{itemize}
                        \item Directs release of glucose from stored glycogen in liver = \textbf{glycogenolysis}
                        \item Signals liver to start \textbf{gluconeogenesis} = making glucose from noncarbohydrate sources, mostly protein
                    \end{itemize}
                \item Your body will also break down fat stores to provide energy for your tissues
                \item Epinephrine (adrenaline) also stimulates glycogenolysis and increases blood glucose levels
                    \begin{itemize}
                        \item "Fight-or-flight" hormone: stress, bleeding, low blood glucose levels trigger its release
                    \end{itemize}
            \end{itemize}

        \subsection{Carbohydrates Fuel Your Body during Fasting and Prevent Ketosis}
            \begin{itemize}
                \item Liver glycogen stores depleted after about 18 hours
                \item Without glucose, fat can't be broken down completely and acidic \textbf{ketone bodies} are produced
                    \begin{itemize}
                        \item \textbf{Ketosis:} elevated ketone levels after fasting about two days
                        \item Protein from muscle and organs broken down to make glucose
                            \begin{itemize}
                                \item Brain switches to using ketone bodies for fuel to spare protein-rich tissues
                                \item If fasting continues, protein reserves are depleted and death occurs
                            \end{itemize}
                    \end{itemize}
            \end{itemize}

        \subsection{How Much Carbohydrates Do You Need and What Are the Best Food Sources?}
            \begin{itemize}
                \item Minimum amount of carbohydrates needed daily
                    \begin{itemize}
                        \item DRI: 130 grams per day for brain function
                        \item Consume diet with low to moderate amounts of simple carbohydrates and higher amounts of fiber and other complex carbohydrates
                        \item Choose carbohydrates from a variety of nutrient dense, low saturated fat foods
                    \end{itemize}
                \item Whole grains can help meet starch and fiber needs
                    \begin{itemize}
                        \item Select whole-grain breads and cereals that have at least 2-3 grams of total fiber per serving
                    \end{itemize}
                \item Fruits and vegetables provide surgars, starch, and fiber
                \item Legumes, nuts, and seeds are excellent sources of carbohydrates and fiber
                \item Low-fat and fat-free dairy products provide some simple sugars
                \item Be careful when selecting packaged foods
                    \begin{itemize}
                        \item Can be good sources of carbohydrates, but may also have added sugar, salt, fat, and calories
                    \end{itemize}
            \end{itemize}

        \subsection{How Much Fiber Do You Need and What Are Its Food Sources}
            \begin{itemize}
                \item Filling up on fiber
                    \begin{itemize}
                        \item DRI: 14 grams of fiber per 1,000 calories to promote heart health
                            \begin{itemize}
                                \item Most Americans fall short: about 15 grams per day
                            \end{itemize}
                        \item Gradually increasing fiber in your diet will minimize side effects, such as flatulence
                            \begin{itemize}
                                \item As you add fiber to your diet, you should also drink more fluids
                            \end{itemize}
                    \end{itemize}
            \end{itemize}

        \subsection{Nutrition in the Real World: Grains, Glorious Whole Grains}
            \begin{itemize}
                \item Grains: important staple and source of nutrition
                    \begin{itemize}
                        \item Three edible parts: \textbf{bran, endosperm, germ}
                        \item \textbf{Refined grains:} milling removes bran and germ
                            \begin{itemize}
                                \item Some B vitamins, iron, phytochemicals, and dietary fiber lost as a result
                                \item Examples: wheat or white bread, white rice
                            \end{itemize}
                        \item \textbf{Enriched grains:} folic acid, thiamin, niacin, riboflavin, and iron added to restore some of the lost nutrition
                        \item \textbf{Whole-grain} foods contain all three parts of kernel
                            \begin{itemize}
                                \item Examples: brown rice, oatmeal, whole-wheat bread
                                    \begin{itemize}
                                        \item Dark bread is not necessarily whole-grain bread
                                    \end{itemize}
                            \end{itemize}
                    \end{itemize}
            \end{itemize}

        \subsection{What's the Difference between Natural and Added Sugars?}
            \begin{itemize}
                \item \textbf{Naturally occurring sugars} are found in fruits and dairy
                    \begin{itemize}
                        \item Usually more nutrient dense; provide more nutrition per bite
                    \end{itemize}
                \item \textbf{Added sugars} are added by manufacturers and are often empty calories (calories that provide little nutrition)
                    \begin{itemize}
                        \item Examples: soda, candy
                    \end{itemize}
                \item Taste buds can't distinguish between naturally occurring and added sugars
                \item Yearly consumption of added sugars has increased since 1970
            \end{itemize}

        \subsection{Processed Foods and Sweets Often Contain Added Sugars}
            \begin{itemize}
                \item Sugar does not cause hyperactivity in kids
                \item Too much sugar can contribute to dental caries, but so can any type of carbohydrate
                \item Too much sugar in the diet can raise "bad" LDL cholesterol and triglycerides
                \item Added sugars have more empty calories and few nutrients and may lead to weight gain
                \item Eating too much added sugar may increase the risk of diabetes
                    \begin{itemize}
                        \item Moderation, balance, and staying within daily calorie needs are essential when it comes to added sugars
                    \end{itemize}
                \item Finding the added sugars in your foods:
                    \begin{itemize}
                        \item Sugars on food labels appear under many different names
                            \begin{itemize}
                                \item Honey and fructose are not nutritionally superior to sucrose
                                    \begin{itemize}
                                        \item Honey should not be given to children younger than one year of age in order to prevent \textbf{Clostridium botulinum} spores that cause botulism
                                    \end{itemize}
                                \item High-fructose corn syrup (HFCS) is less expensive than sucrose and has replaced sweets and soft drinks
                            \end{itemize}
                        \item Naturally occurring sugars are not distinguished from added sugars on the Nutrition Facts panel
                    \end{itemize}
            \end{itemize}
        
        \subsection{Nutrition in the Real World: Avoiding a Trip to the Dentist}
            \begin{itemize}
                \item Carbohydrates play a role in dental caries
                    \begin{itemize}
                        \item Fermentable sugars and starch feed bacteria coating teeth, producing acid to erode tooth enamel and case
                        \item \textbf{Early childhood tooth decay} (baby bottle tooth decay)
                    \end{itemize}
                \item To minimize tooth decay:
                    \begin{itemize}
                        \item Eat three balanced meals daily
                        \item Keep sncaking to a minimum, choosing whole fruits and raw vegetables
                        \item Include foods that fight dental caries: cheese, sugarless gum
                            \begin{itemize}
                                \item Cheese is rich in protein, calcium, phosphorus, can assist in \textbf{remineralization} of your teeth.
                            \end{itemize}
                        \item Regular dental care and good dental hygiene
                        \item Drink water and avoid sugar-sweetened beverages
                    \end{itemize}
            \end{itemize}
    
        \subsection{How Much Added Sugar Is Too Much?}
            \begin{itemize}
                \item Latest conclusions from the report of the 2105 \textbf{Dietary Guidelines for Americans:}
                    \begin{itemize}
                        \item 10 percent or less of your total daily calories should come from added sugars
                    \end{itemize}
                \item The American Heart Association has recommended:
                    \begin{itemize}
                        \item Women should consume no more than 100 calories (6tsp) of added sugar daily
                        \item Men should consume no more than 150 calories (9tsp) of added sugar daily
                    \end{itemize}
                \item American adults currently consume 73 grams of added sugar daily (about 17tsp)
            \end{itemize}

        \subsection{Examining the Evidence: Do Sugar-Sweetened Beverages Cause Obesity?}
            \begin{itemize}
                \item Every day 50 percent of Americans consume some form of sugary drinks equivalent to about one 12-ounce soda.
                \item Major theories on relationship between sugar-sweetened drink consumption and weight gain
                    \begin{itemize}
                        \item Additional calories leads to excess overall calorie intake
                        \item Sugar in liquid form increases our appetite
                    \end{itemize}
                \item Bottomline: There is not yet enough evidence to say that sugar sweetened beverages alone contribute more to obesity than other calorie sources
            \end{itemize}

        \subsection{Why is Diabetes a Growing Epidemic?}
            \begin{itemize}
                \item \textbf{Diabetes mellitus:} individual has high blood glucose levels due to insufficient insulin or insulin resistance
                    \begin{itemize}
                        \item \textbf{Insulin resistance:} Glucose can't enter cells because the cells do not respond to insulin
                            \begin{itemize}
                                \item Without glucose, acidic ketone bodies build up, causing life-threatening diabetic ketoacidosis: if untreated can result in coma, death
                            \end{itemize}
                    \end{itemize}
                \item \textbf{Type 1 diabetes:} an autoimmune disease that usually begins in childhood or early adult years
                    \begin{itemize}
                        \item 5 to 10 percent of diabetes cases
                        \item Autoimmune disease: insulin-producing cells in pancreas destroyed; insulin injections required
                    \end{itemize}
                \item \textbf{Type 2 diabetes:} seen in people who have become insulin resistant
                    \begin{itemize}
                        \item 90 to 95 percent of diabetes cases
                        \item Cells are resistant to insulin; eventually insulin-producing cells are exhausted and medication and/or insulin is required
                        \item People 45 and older or at risk for diabetes should be tested
                    \end{itemize}
                \item \textbf{Prediabetes:} may be precursor to type 2
                    \begin{itemize}
                        \item Blood glucose higher than normal but not yet high enough to be classified as diabetes
                        \item Heart disease and stroke can occur
                    \end{itemize}
                \item What effects does diabetes have on your body?
                    \begin{itemize}
                        \item High levels of glucose in your blood can result in long-term damage
                            \begin{itemize}
                                \item High blood glucose levels damage vital organs
                                    \begin{itemize}
                                        \item Nerve damage, numbness, poor circulation leading to infections and leg and foot amputations
                                        \item Eye damage, blindness
                                        \item Tooth and gum problems
                                        \item Kidney damage
                                        \item Increased risk of heart disease
                                        \item Diabetic \textbf{ketoacidosis}
                                    \end{itemize}
                            \end{itemize}
                    \end{itemize}
                \item Low blood sugar levels can also be dangerous
                \item \textbf{Hypoglycemia:} blood glucose level below 70 mg/dL
                    \begin{itemize}
                        \item Symptoms: hunger, shakiness, dizziness
                        \item May occur in people with diabetes when they don't eat regularly to balance effects of insulin or blood glucose-lowering medication
                            \begin{itemize}
                                \item Can cause fainting, coma
                            \end{itemize}
                        \item Uncommonly, may occur after eating (reactive hypoglycemia) or fasting (fasting hypoglycemia)
                    \end{itemize}
                \item How is diabetes treated and controlled?
                    \begin{itemize}
                        \item Blood glucose control is key
                        \item Nutrition and lifestyle goals:
                            \begin{itemize}
                                \item Physical exercise
                                \item Well-balanced diet containing:
                                    \begin{itemize}
                                        \item High-fiber carbohydrates from whole grains, fruits, vegetables
                                        \item Low-fat milk
                                        \item Adequate lean protein sources
                                        \item Unsaturated fats
                                    \end{itemize}
                            \end{itemize}
                    \end{itemize}
                \item Glycemic index (GI) and glycemic load (GL) classify effects of carbohydrate-containing foods on blood glucose
                    \begin{itemize}
                        \item GI: ranks foods' effects on blood glucose compared with equal amounts of pure glucose
                        \item GL: adjusts GI to take into account the amount of carbohydrate consumed
                    \end{itemize}
                \item Eating carbohydrate-heavy foods with protein, fat lowers GI
                \item Sugar is not prohibited; starch causes same rise in blood glucose levels
                \item Total calories important for weight management
                \item Diabetes incidence on the rise
                    \begin{itemize}
                        \item Seventh leading cause of death in the United States
                        \item Adult cases more than tripled since 1980s
                        \item Rapid increase among children
                            \begin{itemize}
                                \item Obesity, overweight, and physical inactivity increase risk
                            \end{itemize}
                    \end{itemize}
                \item Preventing type 2 diabetes:
                    \begin{itemize}
                        \item Lose excess weight, exercise more, limit sugar-sweetened beverages, eat heart-healthy, plant-based diet
                    \end{itemize}
            \end{itemize}

        \subsection{What Are Sugar Substitutes and What Forms Can They Take?}
            \begin{itemize}
                \item \textbf{Sugar substitutes} are as sweet or sweeter than sugar, but contain fewer calories
                    \begin{itemize}
                        \item Must be approved by FDA and deemed safe before allowed in food products in the United States
                        \item Many of these substitutes will not promote dental caries and do not affect blood glucose levels
                    \end{itemize}
                \item Reduced-Calorie Sweeteners
                    \begin{itemize}
                        \item Polyols (sugar alcohols): sorbitol, mannitol, xylitol
                            \begin{itemize}
                                \item Absorbed more slowly than sugar, don't cause spike in blood glucose but not calorie free
                                \item Not completely absorbed; can cause diarrhea
                                \item Found in sugar free chewing gum and candies
                                    \begin{itemize}
                                        \item Can be labeled, "sugar-free" but
                                    \end{itemize}
                                \item Hydrogenated starch hydrolysates (HSH)
                                \item Tagalose
                                    \begin{itemize}
                                        \item Derived from lactose and found in some dairy products
                                        \item 90 percent as sweet as sucrose
                                    \end{itemize}
                            \end{itemize}
                    \end{itemize}
                \item Calorie-free sweeteners:
                    \begin{itemize}
                        \item Saccharin (Sweet'N Low): 200-700 times sweeter than sucrose
                            \begin{itemize}
                                \item The oldest sugar substitute, founded in 1879
                            \end{itemize}
                        \item Aspartame (Nutrasweet, Equal): 200 times sweeter
                            \begin{itemize}
                                \item Derived from amino acids aspartic acid and phenylalanine
                                \item People with PKU need to monitor all dietary sources of phenylalanine, including aspartame
                            \end{itemize}
                        \item Neotame: 7,000-13,000 times sweeter
                            \begin{itemize}
                                \item Also made from amino acids
                            \end{itemize}
                    \end{itemize}
                \item Acesulfame-K (sSunette): 200 times sweeter
                    \begin{itemize}
                        \item The human body does not metabolize
                    \end{itemize}
                \item Sucralose (Splenda): 600 times sweeter
                    \begin{itemize}
                        \item Modified sugar molecule that the body doesn't absorb
                    \end{itemize}
                \item Rebaudioside A (Truvia, PureVia, Sun Crystals): 200 times sweeter
                    \begin{itemize}
                        \item Combination of a sugar alcohol and stevia extract
                    \end{itemize}
                \item Monk fruit (Nectresse): 150-300 times sweeter
                    \begin{itemize}
                        \item Extract of the luo han guo fruit
                    \end{itemize}
                \item Advantame is the newest sugar substitute made from aspartame and vanillin: 20,000 times sweeter than sugar, 100 times sweeter than aspartame
            \end{itemize}

        \subsection{Why is Fiber So Important?}
            \begin{itemize}
                \item Fiber is nondigestible but has many powerful health effects
                \item Fiber helps lower risk of developing:
                    \begin{itemize}
                        \item Constipation
                        \item Diverticulosis, diverticulitis
                        \item Obesity: high-fiber foods add to satiation
                        \item Heart disease: soluble fibers lower elevated blood cholesterol levels
                        \item Colorectal cancer
                        \item Diabetes mellitus: slows digestion and absorption of glucose
                    \end{itemize}
                \item Too much fiber can cause health problems, such as diarrhea, flatulence, and bloating
                    \begin{itemize}
                        \item Gradually increase fiber intake to allow your body time to adjust
                    \end{itemize}
                \item Long-term constipation may play role in diverticulosis
                \item \textbf{Increased} pressure in the colon causes weak spots in the colon to bulge out, forming \textbf{diverticula}
                \item \textbf{Diverticulitis:} Infection of the diverticular
                    \begin{itemize}
                        \item Stomach pain, fever, nausea, vomiting, cramping, and chills
                        \item Consume diet with adequate diet to prevent
                    \end{itemize}
                \item To increase daily fiber intake, here are som easy food substitutions:
                    \begin{itemize}
                        \item Oatmeal or bran flakes instead of corn flakes
                        \item Whole grain crackers instead of cheese crackers
                        \item Whole grain bread instead of white bread
                        \item Popcorn instead of pretzels
                    \end{itemize}
            \end{itemize}
    
    \section{Chapter 5: Fats, Oils, and Other Lipids}
        \subsection{What Are Fats and Why Do You Need Them?}
            \begin{itemize}
                \item \textbf{Lipids:} category of compounds containing carbon, hydrogen, and oxygen that are \textbf{hydrophobic} (insoluble in water)
                \item Fat is the common name for just one type of lipid, known as triglyceride
                    \begin{itemize}
                        \item Fats serve multiple functions in foods:
                            \begin{itemize}
                                \item Give flaky texture to baked goods
                                \item Make meats tender
                                \item Provide flavor and aromas
                                \item Contribute to satiety
                            \end{itemize}
                    \end{itemize}
                \item Fats and other lipids perform important functions in the body:
                    \begin{itemize}
                        \item Energy storage
                        \item Insulation
                        \item Transport of proteins in blood
                        \item Cell membrane structure
                    \end{itemize}
                \item Three types of lipids found in foods and in your body:
                    \begin{itemize}
                        \item Triglycerides (fats), phospholipids, and sterols
                        \item Basic unit of triglycerides and phospholipids is \textbf{fatty acid}
                    \end{itemize}
            \end{itemize}

        \subsection{Fatty Acids Are Found in Triglycerides and Phospholipids}
            \begin{itemize}
                \item \textbf{Fatty acids:} chain of carbon and hydrogen atoms with acid group (COOH at one end)
                    \begin{itemize}
                        \item Over 20 different fatty acids
                        \item Can vary by:
                            \begin{enumerate}
                                \item Length of chain
                                \item Whether carbons have double or single bonds between them
                                \item Total number of double bonds
                            \end{enumerate}
                    \end{itemize}
            \end{itemize}

        \subsection{Fatty Acids Vary in Length and Structure}
            \begin{itemize}
                \item There are three main types of fatty acids:
                    \begin{enumerate}
                        \item \textbf{Saturated fatty acids:} all carbons bonded to hydrogen
                            \begin{itemize}
                                \item Example: stearic acid, 18 carbons, solid at room temperature
                            \end{itemize}
                        \item \textbf{Monounsaturated fatty acids (MUFAs):} one double bond
                            \begin{itemize}
                                \item Example: oleic acid, 18 carbons (olive oil), liquid at room temperature
                            \end{itemize}
                        \item \textbf{Polyunsaturated fatty acids (PUFAs):} more than one double bond
                            \begin{itemize}
                                \item Example: \textbf{essential fatty acids, linoleic acid}, and \textbf{alpha-linolenic acid} (soybean oil)
                            \end{itemize}
                    \end{enumerate}
            \end{itemize}

        \subsection{Triglycerides Contain Three Fatty Acid Chains}
            \begin{itemize}
                \item \textbf{Triglyceride:} three fatty acids connected to glycerol "backbone"
                    \begin{itemize}
                        \item Most common lipid found in foods and body
                        \item Referred to as fats
                            \begin{itemize}
                                \item Saturated fats have mostly saturated fatty acids
                                \item Unsaturated fats have mostly unsaturated fatty acids
                                \item Oils are fats that are liquid at room temperature
                            \end{itemize}
                    \end{itemize}
            \end{itemize}

        \subsection{Phospholipids Contain Phosphate}
            \begin{itemize}
                \item \textbf{Phospholipids:} have \textbf{glycerol} backbone, but two fatty acids and a phosphorus group
                    \begin{itemize}
                        \item Phosphorus-containing head is \textbf{hydrophilic}
                        \item Fatty acid tail is hydrophobic
                        \item Cell membranes made of phospholipid bilayer
                            \begin{itemize}
                                \item Major phospholipid in cell membrane = lecithin
                                    \begin{itemize}
                                        \item Lecithin used as \textbf{emulsifier} in foods such as salad dressings to keep oils and water mixed together
                                    \end{itemize}
                            \end{itemize}
                    \end{itemize}
            \end{itemize}

        \subsection{Sterols Have a Unique Ring Structure}
            \begin{itemize}
                \item \textbf{Sterols} are composed mainly of four connecting rings of carbon and hydrogen
                    \begin{itemize}
                        \item Example: cholesterol
                            \begin{itemize}
                                \item Important role in cell membrane structure
                                \item \textbf{Precursor} of important compounds in body
                                    \begin{itemize}
                                        \item Converted into vitamin D and other substances
                                    \end{itemize}
                                \item Not required in diet since body makes all cholesterol needed
                            \end{itemize}
                    \end{itemize}
            \end{itemize}

        \subsection{What Happens to the Fat You Eat?}
            \begin{itemize}
                \item Mouth: chewing and lingual lipase start digestion
                \item Stomach: gastric lipase breaks down fat into \textbf{diglyceride} and one fatty acid
                \item Small intestine: most digestion occurs here
                    \begin{itemize}
                        \item \textbf{Bile} acids: emulsify fat, break fat globules into smaller pieces
                        \item Pancreatic lipase: two fatty acids and \textbf{monoglyceride}
                        \item Lecithin in bile is packaged with monoglycerides and fatty acids to create \textbf{micelles} (small carriers) for absorption
                        \item Short-chain fatty acids enter bloodstream and travel to liver
                        \item Long-chain fatty acids enter \textbf{lymph} and need transport carriers
                    \end{itemize}
                \item \textbf{Lipoproteins} transport fat through the lymph and blood
                    \begin{itemize}
                        \item \textbf{Chylomicrons:} carry digested fat through lymph into bloodstream
                        \item \textbf{Very low-density lipoproteins} (VLDL): deliver fat made in liver to cells
                        \item \textbf{Low-density lipoproteins} (LDL, "bad" cholesterol): deposite cholesterol on walls of arteries
                        \item \textbf{High-density lipoproteins} (HDL, "good" cholesterol): remove cholesterol from body and deliver to liver for excretion
                    \end{itemize}
            \end{itemize}

        \subsection{How Does Your Body Use Fat and Cholesterol}
            \begin{itemize}
                \item Fat
                    \begin{itemize}
                        \item An energy-dense source of fuel: 9 calories per gram
                            \begin{itemize}
                                \item Glucagon also stimulates release of fat from fat cells to provide energy forr heart, liver, and muscle when blood glucose level declines
                            \end{itemize}
                        \item Is needed for absorption of fat-soluble vitamins A, D, E, K, and carotenoids
                        \item Insulates body to maintain body temperature
                        \item Cushions bones, organs, nerves
                    \end{itemize}
                \item Two polunsaturated fatty acids, linoleic acid (an omega-6 fatty acid) and alpa-linolenic acid (an omega-3 fatty acid), are essential
                    \begin{itemize}
                        \item The essential fatty acids help maintain healthy skin cells, nerves, and cell membranes and are precursors to eicosanoids, eicosapentaenoic acid, and docosahexaenoic acid
                        \item \textbf{Eicosanoids:} hormone-like substances made from essential fatty acids, which are involved in inflammation, blood clotting, blood pressure
                        \item \textbf{Eicosapentaenoic acid (EPA)} and \textbf{docosahexaenoic acid (DHA):} two omega-3 fatty acids that are heart-healthy
                            \begin{itemize}
                                \item Fatty fish such as salmon, herring, and sardines are rich sources
                            \end{itemize}
                    \end{itemize}
                \item Cholesterol has many important roles:
                    \begin{itemize}
                        \item Part of cell membranes
                        \item Precursor for vitamin D, bile acids, sex hormones
                        \item Cholesterol in your diet does not determine your blood cholesterol
                        \item Your body makes all the cholesterol it needs
                    \end{itemize}
            \end{itemize}

        \subsection{How Much Fat Do You Need Each Day?}
            \begin{itemize}
                \item You need to consume a specific percentage of your daily calories from fat
                    \begin{itemize}
                        \item AMDR of DRI: 20 to 35 percent of total daily calories should come from fat
                        \item Remember that dietary fat has more than twice the calories per gram of carbohydrates or protein
                        \item For heart health, you should consume less than 10 percent (ideally less than 7 percent) lf your calories from saturated fats
                    \end{itemize}
                \item You need to consume a specific amount of essential fatty acids daily
                    \begin{itemize}
                        \item Between 5 and 10 percent of the total calories in your diet should come from linoleic acid
                        \item Alpha-linolenic acid should make up 0.6 to 1.2 percent of your total calories
                    \end{itemize}
                \item You should minimize saturated fat and \textbf{trans} fat in your diet
                    \begin{itemize}
                        \item Consuming too much saturated fat can lead to higher levels of the "bad" LDL cholesterol carrier
                        \item \textbf{Trans} fats are created by food manufacturers through the process of \textbf{hydrogenation}
                            \begin{itemize}
                                \item Made by manufacturers to resist \textbf{randicidity} and increase shelf life
                            \end{itemize}
                        \item \textbf{Trans} fats are actually worse for heart health than saturated fat
                            \begin{itemize}
                                \item Raise LDL cholesterol and lower HDL cholesterol
                            \end{itemize}
                    \end{itemize}
                \item Your body makes all the cholesterol it needs, so you do not need to consume it in your diet
                \item Healthy individuals over the age of 2 are advised to limit their dietary cholesterol to less than 300 mg daily
                \item The latest research suggests that dietary cholesterol has less impact on blood cholesterol levels than saturated fat
                \item However, very high intakes of dietary cholesterol can increase blood cholesterol levels
            \end{itemize}

        \subsection{What Are the Best Food Sources of Fats?}
            \begin{itemize}
                \item Foods that contain unsaturated fats (both monounsaturated and polyunsaturated fats) are better for your health than foods high in saturated fat, cholesterol, and/or \textbf{trans} fat
                    \begin{itemize}
                        \item Unsaturated fats are abundant in vegetable oils as well as soybeans, walnuts, peanut butter, flaxseeds, and wheat germ
                        \item Vegetable oils, buts, and flaxseeds are also good sources of essential fatty acids
                    \end{itemize}
            \end{itemize}

        \subsection{What Are Fat Substitutes and How Can They Be Part of a Healthy Diet?}
            \begin{itemize}
                \item \textbf{Fat substitutes} are designed to provide all the creamy properties of fat for fewer calories and total fat grams
                    \begin{itemize}
                        \item Because fat has more than double the calories per gram of carbohydrates or protein, fat substitutes have the potential to reduce calories from fat by more than 50 percent
                    \end{itemize}
                \item Fat substitutes can be carbohydrate-, protein-, or fat-based
                    \begin{itemize}
                        \item The majority are carbohydrate-based and use plant polysaccharides
                    \end{itemize}
            \end{itemize}

        \subsection{What Is Heart Disease and What Increases Your Risk?}
            \begin{itemize}
                \item Heart disease begins with a buildup in the arteries
                    \begin{itemize}
                        \item \textbf{Atherosclerosis:} narrowing of arteries due to buildup of \textbf{plaque} (hardened debris of cholesterol-laden foam cells, platelets, and other substances)
                            \begin{itemize}
                                \item Though to begin with injury to lining of arteries, contributed by high blood pressure, high cholesterol levels, and smoking
                                \item Increases chance of blood clots blocking the vessel, causing \textbf{heart attack} or \textbf{stroke}
                            \end{itemize}
                    \end{itemize}
            \end{itemize}

        \subsection{Risk Factors for Heart Disease}
            \begin{itemize}
                \item Risk factors you \textbf{can't} control:
                    \begin{itemize}
                        \item age, gender, family history, and Type 1 Diabetes
                    \end{itemize}
                \item Risk factors you \textbf{can} control:
                    \begin{itemize}
                        \item Regular exercise can help lower LDL and raise HDL
                        \item Losing excess weight and quitting smoking can help increase HDL levels
                    \end{itemize}
                \item Other potential risk factors: high levels of homocysteine, Lp(a) protein, C-reactive protein (sign of inflammation), Apolipoprotein B (ApoB)
                \item \textbf{Metabolic Syndrome:} group of risk factors, including insulin resistance, that increase the risk of heart disease
            \end{itemize}

        \subsection{What Can You Do to Maintain Healthy Blood Cholesterol Levels and Reduce Your Risk of Heart Disease?}
            \begin{enumerate}
                \item Minimize saturated fats, \textbf{trans} fats, cholesterol in diet
                \item Include fish in your weekly choices
                \item Eat plenty of plant foods
                \item Select foods rich in antioxidants and phytochemicals
                \item Strive for plenty of exercise and manage your weight
                \item Moderate use of alcohol may reduce risk of heart disease, but some should avoid alcohol
                \item The whole is greater than the sum of its parts
            \end{enumerate}

        \subsection{Examining the Evidence: The Traditional Mediterranean Diet}
            \begin{itemize}
                \item Traditional diet of Mediterranean region is associated with lower risk of heart disease and cancer
                    \begin{itemize}
                        \item Very active lifestyle as well as long, relaxing family meals, afternoon naps, supportive community
                        \item Plant-based diet of whole grains, fruits, vegetables, legumes, and nuts
                            \begin{itemize}
                                \item With olive oil, low-fat dairy, water
                                \item Occasional fish, poultry, eggs, meat, sweets, wine
                            \end{itemize}
                    \end{itemize}
            \end{itemize}

        \subsection{Nutrition in the Real World: Mercury and Fish}
            \begin{itemize}
                \item Methylmercury is a toxic chemical especially harmful to the nervous systems of unborn children
                    \begin{itemize}
                        \item Accumulates in larger fish with a longer life span
                            \begin{itemize}
                                \item Examples: swordfish, shark, king mackerel, tilefish
                            \end{itemize}
                    \end{itemize}
                \item Women of childbearing age age and young children should avoid these four types of fish
                    \begin{itemize}
                        \item Pregnant women/women of childbearing age: should consume 8 to 12 ounces of other fish (variety) weekly
                            \begin{itemize}
                                \item Ideally 2-3 4 ounce servings of fish per week from the "Best Choices" list or 1 serving from the "Good Choices" list
                                \item Canned albacore tuna has more mercury than light tuna: 6 ounce/week limit
                            \end{itemize}
                        \item Can server 1 to 2 two ounce servings/week to children starting at 2 years old
                    \end{itemize}
            \end{itemize}

        \subsection{Eat Plenty of Plant Foods}
            \begin{itemize}
                \item Eating more plant foods high in soluble fiber may be one of the easiest ways to decrease LDL
                \item Although all plant foods are cholesterol free, they do contain \textbf{phytosterols}, which are plant sterols similar to cholesterol found in the plant's cell membranes
                    \begin{itemize}
                        \item Plant sterols can help lower LDL cholesterol levels by competing with cholesterol for absorption in the intestinal tract
                        \item Sources of sterols include soybean oil, many fruits, vegetables, legumes, sesame seeds, nuts, cereals, and other plant foods
                    \end{itemize}
            \end{itemize}

        \subsection{Routinely Select Foods Rich in Antioxidants and phytochemicals}
            \begin{itemize}
                \item Antioxidants may help reduce LDL cholesterol levels
                \item \textbf{Flavonoids} are phytochemicals found in fruits, vegetables, tea, nuts, and seeds that may offer some antioxidant protection as well as potentially inhibiting platelet aggregation, which can perpetuate a blood clot
            \end{itemize}
        
        \subsection{Made Over Made Better}
            \begin{itemize}
                \item Make these replacements to keep saturated fat intake no more than 7-10 percent of your total calories every day
                    \begin{itemize}
                        \item Replace 2 percent milk and whipped cream in your coffee with a black coffee
                        \item Replace butter with jelly on toast
                        \item Choose reduced-fat cheese instead of full-fat cheddar cheese
                        \item Choose low-fat frozen yogurt instead of chocolate ice cream
                    \end{itemize}
            \end{itemize}

    \section{Chapter 6: Proteins and Amino Acids}

        \subsection{What Are Proteins and Why Are They Important?}
            \begin{itemize}
                \item \textbf{Proteins} are the predominant structural and functional materials in every cell
                    \begin{itemize}
                        \item Contain carbon, hydrogen, oxygen (like carbohydrates and fats)
                        \item Also contain nitrogen
                        \item Each amino acid has:
                            \begin{itemize}
                                \item \textbf{Acid group (COOH)}
                                \item \textbf{Amine group (NH2)}
                                \item \textbf{Side chain (unique)}
                            \end{itemize}
                    \end{itemize}
                \item All proteins consist of a chain of some combination of 20 unique \textbf{amino acids}
                \item An acid group and amine group are connected by \textbf{peptide bonds} to create \textbf{dipeptides}
            \end{itemize}

        \subsection{Essential, Nonessential, and Conditional Amino Acids}
            \begin{itemize}
                \item Nine \textbf{essential amino acids}
                    \begin{itemize}
                        \item Cannot be made by the body
                        \item It is "essential" to obtain them from the diet
                    \end{itemize}
                \item Eleven \textbf{nonessential amino acids}
                    \begin{itemize}
                        \item Can be synthesized in the body from other amino acids or by adding nitrogen to carbon-containing \textbf{structures}
                    \end{itemize}
                \item \textbf{Conditionally essential amino acids}
                    \begin{itemize}
                        \item Under certain conditions, some nonessential amino acids cannot be synthesized and must be consumed in the diet
                    \end{itemize}
            \end{itemize}

        \subsection{Denaturation of Proteins Changes Their Shape}
            \begin{itemize}
                \item \textbf{Denaturation:} the alteration (unfolding) of a protein's shape, which changes the structure and function of the protein
                    \begin{itemize}
                        \item Examples: cooking meat, eggs changing texture
                        \item Stomach acid untangles protiens to aid in digestion
                    \end{itemize}
            \end{itemize}
        
        \subsection{What Happens to the Protein You Eat?}
            \begin{itemize}
                \item Dietary proteins are digested and absorbed in stomach and small intestine
                    \begin{itemize}
                        \item Stomach acids denature protein and activate pepsin, which breaks down protein into shorter polypeptides
                        \item In the small intestine, polypeptides are broken down into tripeptides, dipeptides, and amino acids
                        \item Amino acids enter blood and travel to liver
                    \end{itemize}
            \end{itemize}

        \subsection{Your Body Degrades and Synthesizes Proteins}
            \begin{itemize}
                \item Amino acids come from:
                    \begin{itemize}
                        \item Diet
                        \item Breakdown of proteins in the body
                        \item A limited supply is stored in \textbf{amino acid pools} in blood and cells for needed protein synthesis
                    \end{itemize}
                \item \textbf{Protein turnover:} process of continuous breakdown and synthesis of protein from its amino acids
                \item Amino acids can be used to make:
                    \begin{itemize}
                        \item Body proteins
                        \item Non-protein substances
                            \begin{itemize}
                                \item Examples: thyroid hormones, melanin
                            \end{itemize}
                    \end{itemize}
                \item After amine groups are removed (converted to urea, excreted in urine), amino acids can also be:
                    \begin{itemize}
                        \item Burned for energy
                        \item Stored as fat
                        \item Made into glucose
                    \end{itemize}
            \end{itemize}

        \subsection{DNA Directs Synthesis of New Proteins}
            \begin{itemize}
                \item DNA in the cell nucleus contains instructions for protein synthesis
                \item \textbf{Gene:} DNA segment that codes for specific protein
                \item Specialized RNA molecules carry out instructions for protein synthesis
                    \begin{itemize}
                        \item \textbf{Messenger RNA (mRNA)} and transfer \textbf{RNA (tRNA)} perform very specific roles during protein synthesis
                    \end{itemize}
                \item When abnormalities occur during protein synthesis, serious medical conditions may result
                    \begin{itemize}
                        \item Example: \textbf{sickle-cell anemia}
                    \end{itemize}
            \end{itemize}

        \subsection{How Does Your Body Use Proteins}
            \begin{itemize}
                \item Proteins provide structural and mechanical support and help maintain body tissues
                    \begin{itemize}
                        \item \textbf{Collagen:} a ropelike, fibrous protein that is the most abundant protein in your body
                        \item \textbf{Connective tissue:} the most abundant tissue type in the body; made up primarily of collagen, it supports and connects body parts as well as providing protectionand insulation
                    \end{itemize}
                \item Most enzymes and many hormones are composed of proteins
                \item Proteins help maintain fluid balance
                \item Proteins help maintain acid-base balance
                    \begin{itemize}
                        \item \textbf{Buffers:} substances that help maintain the proper pH in a solution by attracting or donating hydrogen ions
                    \end{itemize}
                \item Proteins transport substances throughout the body
                    \begin{itemize}
                        \item \textbf{Transport proteins} shuttle oxygen, waste products, lipids, some vitamins, and sodium and potassium through your blood and into and out of cells through cell membranes
                    \end{itemize}
                \item Proteins contribute to a healthy immune system
                    \begin{itemize}
                        \item Specialized protein "soldiers" called \textbf{antibodies} eliminate potentially harmful substances
                    \end{itemize}
                \item Proteins can provide energy
                \item Protein improves satiety and appetite control
            \end{itemize}

        \subsection{How Much Protein Do You Need?}
            \begin{itemize}
                \item Healthy adults should be in \textbf{nitrogen balance}
                    \begin{itemize}
                        \item Amount of nitrogen consumed (in dietary protein) amount excreted (in urine)
                    \end{itemize}
                \item Nitrogen imbalances
                    \begin{itemize}
                        \item \textbf{Positive nitrogen balance:} more nitrogen is retained (for protein synthesis) than is excreted
                            \begin{itemize}
                                \item Examples: infantts, children, pregnant women
                            \end{itemize}
                        \item \textbf{Negative nitrogen balance:} more nitrogen is excreted than consumed (body proteins broken down)
                            \begin{itemize}
                                \item Examples: starvation, serious injury, or illness
                            \end{itemize}
                    \end{itemize}
            \end{itemize}

        \subsection{Not All Protein Is Created Equal}
            \begin{itemize}
                \item \textbf{Protein quality} is determined by two factors:
                    \begin{itemize}
                        \item The protein's \textbf{digestibility}
                        \item The protein's \textbf{amino acid profile:} the types and amounts of amino acids (essential, nonessential, or both) that the protein contains.
                            \begin{itemize}
                                \item \textbf{Complete proteins:} all essential amino acids, plus some nonessential amino acids
                                    \begin{itemize}
                                        \item Sources: soy, quinoa, and animal protein
                                    \end{itemize}
                                \item \textbf{Incomplete proteins:} low in one or more essential amino acids
                                    \begin{itemize}
                                        \item Sources: plant foods
                                    \end{itemize}
                            \end{itemize}
                    \end{itemize}
                \item Plant proteins "upgraded" to complete proteins by:
                    \begin{itemize}
                        \item Consuming modest amounts of soy or animal protein OR
                        \item Being complemented with other plant proteins that provide enough of the limiting amino acid
                            \begin{itemize}
                                \item Complementary proteins do not need to be eaten in the same meal, only the same day
                            \end{itemize}
                    \end{itemize}
                \item \textbf{Protein Digestibility Corrected Amino Acid Score (PDCAAS)}
                    \begin{itemize}
                        \item Measure of protein quality taking into acount digestibility and amino acid profile
                        \item Basis of protein as percent Dailty Value on food labels
                    \end{itemize}
            \end{itemize}

        \subsection{You Can Determine Your Personal Protein Needs}
            \begin{itemize}
                \item Protein recommendations (DRI)
                    \begin{itemize}
                        \item 10 to 35 percent of total daily calories from protein
                            \begin{itemize}
                                \item Average intake in the United States = 15 percent
                            \end{itemize}
                        \item 0.8g of protein/kg of body weight needed daily
                    \end{itemize}
                \item Calculating your daily protein needs
                    \begin{itemize}
                        \item Convert weight to pounds by dividing by 2.2 pounds/kg
                            \begin{itemize}
                                \item For example: 130 pounds / 2.2 = 59kg, 59kg x 0.8g = 47g of protein/day
                            \end{itemize}
                    \end{itemize}
            \end{itemize}

        \subsection{What Are the Best Food Sources of Protein?}
            \begin{itemize}
                \item Some amount of protein is found in many foods, but it is particularly abundant in meat, fish, poultry, and meat alternatives such as dried beans, peanut butter, buts, and soy
            \end{itemize}

        \subsection{Examining the Evidence: Protein Supplements: Are They Necessary?}
            \begin{itemize}
                \item Varied products promise many benefits, but not needed with adequate diet
                \item Protein shakes and powder
                    \begin{itemize}
                        \item Made of whey, soy, or rice protein
                        \item May contain unwanted additives
                    \end{itemize}
                \item Amino acid supplements
                    \begin{itemize}
                        \item Sold as remedies for various health issues
                        \item May have negative effects
                    \end{itemize}
                \item Protein and energy bars
                    \begin{itemize}
                        \item Convenient, but expensive and high in calories
                    \end{itemize}
            \end{itemize}

        \subsection{What Happens if You Eat Too Much or Too Little Protein?}
            \begin{itemize}
                \item Eating too much protein:
                    \begin{itemize}
                        \item May increase risk of heart disease, kidney stones, calcium loss from bones
                        \item Can displace other nutrient- and fiber- rich foods associated with a reduced risk of chronic diseases
                            \begin{itemize}
                                \item Whole grains, fruits, vegetables
                            \end{itemize}
                    \end{itemize}
                \item Eating too little protein
                    \begin{itemize}
                        \item May lead to reduction of lean body mass, especially in older adults
                        \item Risk of increased frailty, impaired healing, decreased immune function
                    \end{itemize}
                \item \textbf{Protein-energy malnutrition (PEM)}
                    \begin{itemize}
                        \item Inadequate calories and/or protein
                        \item More common in children, because they are growing
                        \item Factors: poverty, poor food quality, insufficient food, unsanitary living conditions, ignorance, stopping lactation (nursing) too early
                    \end{itemize}
            \end{itemize}

        \subsection{Eating Too Little Protein Can Lead to Poor Health and Malnutrition}
            \begin{itemize}
                \item \textbf{Kwashiorkor:} severe dificiency of dietary protein
                    \begin{itemize}
                        \item Signs: edema, muscle loss, skin rashes, hair changes, water and electrolyte
                        \item Seen in children weaned to low-protein cereals
                    \end{itemize}
                \item \textbf{Marasmus:} severe deficiency of calories
                    \begin{itemize}
                        \item Signs: emaciation, lack of growth, loss of fat stores
                    \end{itemize}
                \item \textbf{Marasmic kwashiorkor:} worst of both conditions
                \item Treatment includes a multi-step approach
                    \begin{enumerate}
                        \item Address life-threatening factors
                            \begin{itemize}
                                \item Severe dehydration
                                \item Fluid/nutrition imbalances
                            \end{itemize}
                        \item Restore deplated tissue
                            \begin{itemize}
                                \item Gradually provide nutritionally dense kilocalories and high-quality protein
                            \end{itemize}
                        \item Transition to foods and introduce physical activity
                    \end{enumerate}
            \end{itemize}

        \subsection{How Do Vegetarians Meet Protein Needs?}
            \begin{itemize}
                \item \textbf{Vegetarians} can meet protein needs by consuming:
                    \begin{itemize}
                        \item Variety of plant foods
                        \item Protein-rich meat alternatives:
                            \begin{itemize}
                                \item Soy
                                \item Dried beans and other legumes
                                \item Nuts
                                \item Eggs, dairy (lacto-ovo-vegetarians)
                            \end{itemize}
                    \end{itemize}
            \end{itemize}

        \subsection{Potential Benefits and Risks of Vegetarian Diets}
            \begin{itemize}
                \item Benefits
                    \begin{itemize}
                        \item May reduce risk of heart disease, high blood pressure, diabetes, cancer, stroke, and obesity
                        \item Vegetarian diet food staples are rich in fiber and low in saturated fat and cholesterol
                    \end{itemize}
                \item Risks
                    \begin{itemize}
                        \item Potential deficiencies of nutrients found in animal foods
                            \begin{itemize}
                                \item Protein, iron, zinc, calcium, vitamin D, riboflavin, vitamins B12 and A, omega-3 fatty acids
                            \end{itemize}
                    \end{itemize}
            \end{itemize}

        \subsection{Nutrition in the Real World: The Joy of Soy}
                \begin{itemize}
                    \item Benefits of soy
                        \begin{itemize}
                            \item High-quality protein source
                            \item Low in saturated fat
                            \item Contains \textbf{isoflavones} (phytoestrogens)
                                \begin{itemize}
                                    \item Have chemical structure similar to human \textbf{estrogen}
                                \end{itemize}
                            \item Lowers blood cholesterol levels
                            \item May reduce risk of heart disease, certain cancers
                        \end{itemize}
                \end{itemize}

    \section{Chapter 7: Vitamins}
        
        \subsection{What Are Vitamins?}
            \begin{itemize}
                \item Vitamins are essential nutrients
                    \begin{itemize}
                        \item Tasteless, organic compounds needed in small amounts
                        \item A deficiency will cause physiological symptoms
                        \item Consuming too much of some vitamins will cause adverse effects
                    \end{itemize}
                \item Vitamins are either fat-soluble or water-soluble
                    \begin{itemize}
                        \item Fat-soluble vitamins (A, D, E, and K) are absorbed with dietary fat and can be stored in body
                        \item Water-soluble vitamins (B vitamins and C) are absorbed with water and enter the bloodstream directly
                            \begin{itemize}
                                \item Not stored in body, but excesses can still be harmful
                            \end{itemize}
                    \end{itemize}
                \item Some vitamins function as \textbf{antioxidants}, which counteract \textbf{oxidation} by neutralizing substances called free radicals.
                    \begin{itemize}
                        \item Vitamins A, C, and E, and beta-carotene are antioxidants
                        \item \textbf{Free radicals} are unstable oxygen-containing molecules that can damage the cells of the body and possibly contribute to increased risk of chronic diseases
                    \end{itemize}
                \item Vitamins differ in \textbf{bioavailability:} the degree to which a nutrient is absorbed from foods and used in the body
                    \begin{itemize}
                        \item Vitamins can be destroyed by air, water, or heat
                        \item Don't expose your produce to air
                        \item A little water is enough for cooking
                        \item Reduce your cooking time
                        \item Keep your food cool
                    \end{itemize}
                \item Overconsumption of some vitamins can be toxic
                \item \textbf{Provitamins} can be converted to vitamins by the body
            \end{itemize}

        \subsection{Vitamins Can Be Destroyed by Air, Water, or Heat}
            \begin{itemize}
                \item Air exposure can destroy water-soluble vitamins and fat-soluble vitamins A, E, and K.
                    \begin{itemize}
                        \item Store in airtight, covered containers and use soon after purchase
                    \end{itemize}
                \item To reduce vitamin loss, cook vegetables in a minimal amount of liquid
                    \begin{itemize}
                        \item Steaming or microwaving with minimal water may help preserve some vitamins in vegetables
                    \end{itemize}
                \item Heat will also destroy water soluble vitamins, especially vitamin C
                    \begin{itemize}
                        \item Microwaving, steaming, or stir-frying can preserve more vitamins than boiling
                    \end{itemize}
                \item Cooler temperatures help preserve vitamins, so store produce in the refrigerator rather than pantry
            \end{itemize}

        \subsection{Overconsumption of Some Vitamins Can Be Toxic}
            \begin{itemize}
                \item Vitamin \textbf{toxicity}, or hypervitaminosis, is very rare
                \item Vitamin toxicity does not occur be eating a normal balanced diet
                \item Can result when individuals consume \textbf{megadose} levels of vitamin supplements, usually in the mistaken belief that more is better
                \item To prevent excessive intake, the Dietary Reference Intakes include a tolerable upper intake level for most vitamins
            \end{itemize}
        
        \subsection{Provitamins Can Be Converted to Vitamins by the Body}
            \begin{itemize}
                \item \textbf{Provitamins} are substances found in foods that are not in a form directly usably by the body, but that can be converted into an active form once they are absorbed
                    \begin{itemize}
                        \item Example: beta-carotene, which is split into two molecules of vitamin A in the small intestinal cell wall or in the liver cells
                    \end{itemize}
                \item Vitamins found in foods that are already in the active form, called \textbf{preformed} vitamins, do not undergo conversion in the body.
            \end{itemize}
        
        \subsection{Vitamin A}
            \begin{itemize}
                \item \textbf{Vitamin A: retinoids} (retinol, retinal, retinoic acid)
                    \begin{itemize}
                        \item Preformed vitamin A only found in animal foods: liver, eggs, fortified milk and cheese
                        \item Some plants contain \textbf{provitamin A carotenoids}, which are converted to retinol in your body
                            \begin{itemize}
                                \item Carotenoids, including \textbf{beta-carotene}, are pigments that give color to carrots, cantaloupe, sweet potatoes, spinach, broccoli
                                    \begin{itemize}
                                        \item Like fat-soluble vitamins, are absorbed more efficiently if fat is present in intestinal tract
                                    \end{itemize}
                            \end{itemize}
                    \end{itemize}
                \item Functions:
                    \begin{itemize}
                        \item Essential for healthy eyes
                            \begin{itemize}
                                \item Component of \textbf{rhodopsin} and \textbf{iodopsin}, light-sensitive proteins needed for vision
                            \end{itemize}
                        \item Involved in cell differentation, reproduction, and immunity by promoting gene expression for:
                            \begin{itemize}
                                \item Healthy skin, mucous membranes
                                \item Bone growth
                                \item Fetal development
                                \item White blood cells to fight harmful bacteria
                            \end{itemize}
                    \end{itemize}
                \item Daily needs:
                    \begin{itemize}
                        \item Adult males: 900 micrograms ($\mu$g) \textbf{retinol activity equivalents (RAE)}
                        \item One RAE = 3.3 \textbf{international units (IU)}
                        \item Adult females: 700 micrograms RAE
                    \end{itemize}
                \item Food sources: organ meats (liver), milk, eggs, carrots, spinach, sweet potatoes, pumpkin
                \item Too much or too little:
                    \begin{itemize}
                        \item Excessive amounts of preformed vitamin A can accumulate to toxic levels
                            \begin{itemize}
                                \item Upper limit for adults: 3,000 micrograms
                            \end{itemize}
                        \item Carotenoids in food are not toxic
                            \begin{itemize}
                                \item Excess carotenoids in diet cause nonthreatening condition: \textbf{carotenodermia}
                            \end{itemize}
                        \item Chronic vitamin A deficiency causes \textbf{night blindless}
                        \item Prolonged vitamin A deficiency leads to \textbf{xerophthalmia} (permanent damage to the cornea)
                            \begin{itemize}
                                \item Main cause of preventable blindness in children
                            \end{itemize}
                        \item Vitamin A deficiency also associated with stunting of bones
                    \end{itemize}
            \end{itemize}
            

        \subsection{Vitamin E}
            \begin{itemize}
                \item Alpha-tocopherol is most active form in body
                \item Functions:
                    \begin{itemize}
                        \item Acts as apowerful antioxidant
                            \begin{itemize}
                                \item Protects cell membranes, prevents oxidation of LDL cholesterol
                            \end{itemize}
                        \item Acts as an anticoagulant, inhibiting formation of harmful clots inside bloodstream
                    \end{itemize}
                \item Daily needs: Adults need 15mg of alpha-tocopherol equivalents
                \item Food sources: vegetable oils, nuts, seeds, fortified cereals, some green leafy vegetables
                \item Too much or too little:
                    \begin{itemize}
                        \item No known risk of consuming too much vitamin E from natural food sources
                            \begin{itemize}
                                \item Overconsumption of synthetic form in dietary supplements and fortified foods can increase risk of a \textbf{hemorrhage:} upper limit is 1,000 mg/day
                            \end{itemize}
                        \item Although rare, chronic deficiency of vitamin E can cause nerve problems, muscle weakness, and free radical damage to cell membranes
                    \end{itemize}
            \end{itemize}

        \subsection{Vitamin K}
            \begin{itemize}
                \item Two forms of vitamin K
                    \begin{itemize}
                        \item \textbf{Menaquinone} synthesized by intestinal bacteria
                        \item Phylloquinone found in green plants
                    \end{itemize}
                \item Functions:
                    \begin{itemize}
                        \item Essential for blood clotting \textbf{(coagulation)}
                            \begin{itemize}
                                \item Involved in synthesizing four blood \textbf{clotting factors}
                            \end{itemize}
                        \item Important to bone health
                            \begin{itemize}
                                \item Enables bone protein \textbf{osteocalcin} to bind with calcium
                            \end{itemize}
                    \end{itemize}
                \item Daily needs: based on current consumption, since amount contributed by intestinal synthesis is unknown
                    \begin{itemize}
                        \item Men need 120 $\mu$g/day
                        \item Women need 90 $\mu$g/day
                    \end{itemize}
                \item Food sources: green vegetables such as broccoli, asparagus, spinach, salad greens, brussels sprouts, cabbage; also vegetable oils and margarine
                \item Too much or too little:
                    \begin{itemize}
                        \item No known problems of consuming too much vitamin K from foods or supplements
                        \item People taking anticoagulant medications such as \textbf{warfarin} (Coumadin) need to keep vitamin K intake consistent
                            \begin{itemize}
                                \item Changes in intake can increase or decrease drug effectiveness
                            \end{itemize}
                        \item Vitamin K deficiency that is severe enough to affect blood clotting is extremely rare
                            \begin{itemize}
                                \item At risk: people with problems absorbing fat
                            \end{itemize}
                    \end{itemize}
            \end{itemize}

        \subsection{Vitamin D}
            \begin{itemize}
                \item Called "sunshin vitamin" because it is made in the body with help of sunlight \textbf{(UV)}
                    \begin{itemize}
                        \item Cholesterol-containing compound in skin is converted to inactive form of vitamin D
                        \item People with insufficient sunlight exposure must meet needs through diet; vitamin D in foods is also an inactive form
                        \item Inactive form converted to circulating form in liver, then to active form in kidneys
                    \end{itemize}
                \item Functions: active form acts as a hormone
                    \begin{itemize}
                        \item Regulates two important bone minerals: calcium (Ca) and phosphorus (P)
                            \begin{itemize}
                                \item Stimulates intestinal absorption of Ca and P to maintain healthy blood levels and build and maintain bones
                                \item When dietary calcium is inadequate, vitamin D and \textbf{parathyroid hormone} cause calcium to leave bones to maintain necessary blood levels
                            \end{itemize}
                        \item May aid prevention of some cancers, diabetes, heart disease, and other conditions
                    \end{itemize}
                \item Daily needs:
                    \begin{itemize}
                        \item Sun exposure cannot meet everyone's vitamin D needs
                            \begin{itemize}
                                \item Skin pigment melanin and use of sunscreen reduce vitamin D production
                                \item Sunlight intensity during winter in northern and southern latitudes not sufficient to make vitamin D
                            \end{itemize}
                        \item Therefore, vitamin D needs are based on dietary sources
                        \item Adults: 15 to 20 micrograms (600 to 800 IU) per day, depending on age (19 to 70 yo and > 70 yo, respectively)
                    \end{itemize}
                \item Food sources: fortified milk and yogurt, fortified cereals, fatty fish (examples: sardines, salmon)
                \item Too much or too little:
                    \begin{itemize}
                        \item Overuse of supplements may lead to \textbf{hypervitaminosis D}, which causes \textbf{hypercalcemia}
                            \begin{itemize}
                                \item Damaging calcium deposited in kidneys, lungs, blood vessels, heart
                                \item UL: 4,000 IU (100$\mu$g)
                            \end{itemize}
                        \item \textbf{Rickets:} vitamin D deficiency disease in children
                            \begin{itemize}
                                \item On the rise in United States due to decreased milk consumption, other factors
                                \item Bones inadequately mineralized with calcium and phosphorus, causing them to weaken and leading to bowed legs
                            \end{itemize}
                        \item \textbf{Osteomalacia:} adult equivalent of rickets
                    \end{itemize}
            \end{itemize}

        \subsection{The B Vitamins and Vitamin C Are Water-Soluble}
            \begin{itemize}
                \item Water-soluble vitamins are not stored in body
                    \begin{itemize}
                        \item Excess is excreted in urine
                        \item However, routine intakes of excssive amounts can be harmful
                    \end{itemize}
                \item B vitamins share common role as \textbf{coenzymes}
                    \begin{itemize}
                        \item Help many enzymes produce chemical reactions in cells
                    \end{itemize}
            \end{itemize}

        \subsection{Thiamin (B1)}
            \begin{itemize}
                \item First B vitamin discovered
                \item Functions:
                    \begin{itemize}
                        \item Transmission of nerve impulses
                        \item Metabolism of carbohydrates and certain amino acids
                        \item Plays role in breakdown of alcohol in body
                    \end{itemize}
                \item Daily needs: men: 1.2 mg/day; women: 1.1 mg/day
                \item Food sources: enriched and whole grain products, pork
                \item Too much or too little:
                    \begin{itemize}
                        \item No known toxicity, no UL set
                        \item \textbf{Beriberi:} thiamin deficiency disease
                            \begin{itemize}
                                \item Symptoms can include rapid heartbeat, edema, confusion, loss of coordination
                                \item Rare in United States due to enrichment of grains
                                \item Chronic alcohol abuse can lead to advanced form, \textbf{Wernicke-Korsakoff syndrome:} progressively damaging brain disorder
                                    \begin{itemize}
                                        \item Due to thiamin-deficient diet, and alcohol interfering with thiamin absorption
                                    \end{itemize}
                            \end{itemize}
                    \end{itemize}
            \end{itemize}

        \subsection{Riboflavin (B2)}
            \begin{itemize}
                \item Light-sensitive vitamin, abundant in milk
                    \begin{itemize}
                        \item Opaque containers preserve riboflavin content
                    \end{itemize}
                \item Functions:
                    \begin{itemize}
                        \item Important for energy metabolism
                        \item Keeps cells healthy
                        \item Enhances functions of other B vitamins, such as niacin and B12
                    \end{itemize}
                \item Daily needs:
                    \begin{itemize}
                        \item Men: 1.3 mg/day
                        \item Women: 1.1 mg/day
                    \end{itemize}
                \item Food sources: milk, yogurt, enriched cereals, grains
                \item Too much or too little:
                    \begin{itemize}
                        \item Excess riboflavin excreted in urine: bright yellow color
                        \item No UL set
                        \item Deficiency symptomps rarely seen in healthy individuals eating a balanced diet:
                            \begin{itemize}
                                \item Sore throat, swelling inside mouth, inflamed and purplish-red tongue (glossitis), dry and scaly lips
                            \end{itemize}
                    \end{itemize}
            \end{itemize}

        \subsection{Niacin (B3)}
            \begin{itemize}
                \item Active forms: \textbf{nicotinic acid and nicotinamide}
                \item \textbf{Functions:}
                    \begin{itemize}
                        \item Energy metabolism
                        \item Synthesize fat and cholesterol
                        \item Keep skin cells and digestive system healthy
                    \end{itemize}
                \item Sometimes prescribed in high doses (50 times UL) by physicians to decrease blood LDL cholesterol and triglycerides, increase HDL
                \item Daily needs: men: 16 mg/day; women: 14 mg/day
                    \begin{itemize}
                        \item Can also be made in the body from the amino acid \textbf{tryptophan:} daily needs expressed in niacin equivalents (NE)
                    \end{itemize}
                \item Food sources: meat, fish, poultry, enriched whole-grain breads, fortified cereals
                    \begin{itemize}
                        \item Protein-rich foods are good sources of tryptophan
                    \end{itemize}
                \item Too much or too little:
                    \begin{itemize}
                        \item Overconsumption of niacin supplements can cause flushing, nausea, vomiting; be toxic to liver; raise blood glucose levels: UL is 35 mg/day to prevent \textbf{flushing}
                        \item \textbf{Pellagra:} niacin deficiency disease
                            \begin{itemize}
                                \item Four Ds: dermatitis, diarrhea, dementia, death
                                \item Once common in South, due to corn-based diet
                            \end{itemize}
                    \end{itemize}
            \end{itemize}

        \subsection{Vitamin B6}
            \begin{itemize}
                \item Active forms: \textbf{pyridoxine}, pyridoxal, and pyridoxamine
                \item Functions: as coenzyme with over 100 enzymes in protein metabolism, needed to:
                    \begin{itemize}
                        \item Make nonessential amino acids, convert tryptophan to niacin and hemoglobin in red blood cells
                        \item Keep immune and nervous systems healthy
                        \item Metabolize fats and carbohydrates and break down glycogen
                    \end{itemize}
                \item Daily needs: men: 1.3 to 1.7 mg/day; women: 1.3 to 1.5 mg/day, depending on age
                \item Food sources
                    \begin{itemize}
                        \item Meat, fish, poultry, fortified cereals, nuts, legumes, peanut butter, many fruits and vegetables
                    \end{itemize}
                \item Too much or too little:
                    \begin{itemize}
                        \item UL is 100 mg/day to prevent nerve damage
                        \item Has been used to alleviate symptoms of \textbf{premenstrual syndrome (PMD)}
                        \item Deficiency symptoms:
                            \begin{itemize}
                                \item Sore tongue, skin inflammation, depression, confusion, anemia
                            \end{itemize}
                        \item Those with alcoholism are at risk for deficiency due to poor diet, and because alcohol causes body to lose B6
                    \end{itemize}
            \end{itemize}

        \subsection{Folate}
            \begin{itemize}
                \item Naturally occurring form in foods
                \item \textbf{Folic acid:} synthetic form of folate added to foods and supplements
                \item Functions: vital for DNA synthesis
                    \begin{itemize}
                        \item To create and maintain new cells, including red blood cells
                        \item To help body use amino acids
                        \item Folate deficiency during pregnancy can result in \textbf{neural tube birth defects} (examples: \textbf{spina bifida}, anencephaly)
                        \item Reduces risks of some cancers
                    \end{itemize}
                \item Daily needs:
                    \begin{itemize}
                        \item Adults need 400 micrograms of \textbf{dietary folate equivalents (DFE)}
                        \item Folic acid is absorbed 1.7 times more efficiently than folate found naturally in foods
                        \item Women who might become pregnant need 400 micrograms extra from fortified foods/supplements
                    \end{itemize}
                \item Food sources:
                    \begin{itemize}
                        \item Enriched grains (rice, pasta, breads, cereals), legumes, broccoli, asparagus, leafy greens such as spinach
                    \end{itemize}
                \item Too much or too little:
                    \begin{itemize}
                        \item UL = 1,000 micrograms/day of folic acid from enriched/fortified foods and supplements
                            \begin{itemize}
                                \item Too much folic acid (not naturally occurring folate in foods) masks vitamin B12 deficiency anemia
                            \end{itemize}
                        \item Folate deficiency can lead to macrocytic anemia
                    \end{itemize}
            \end{itemize}

        \subsection{Vitamin B12}
            \begin{itemize}
                \item Also called cobalamine because it contains the element cobalt
                \item Requires \textbf{intrinsic factor}, protein made in stomach, in order to be absorbed in small intestine
                    \begin{itemize}
                        \item \textbf{Pernicious anemia} results in people who cannot make intrinsic factor; treatment requires B12 injection to bypass intestine
                        \item Symptoms may take years to appear since B12 is stored in the liver
                    \end{itemize}
                \item Functions:
                    \begin{itemize}
                        \item To make DNA
                        \item To use certain fatty acids and amino acids
                        \item For healthy nerves and cells, especially red blood cells
                    \end{itemize}
                \item Daily needs:
                    \begin{itemize}
                        \item Adults: 2.4 $\mu$g/day
                        \item Ability to absorb naturally occurring B12 from foods declines with age
                    \end{itemize}
                \item Food sources:
                    \begin{itemize}
                        \item Naturally occurring B12 only found in animal foods (meat, fish, poultry, dairy)
                        \item Synthetic B12 found in fortified foods such as soy milk and some cereals
                    \end{itemize}
                \item Too much or too little:
                    \begin{itemize}
                        \item No upper level set since no known risk from consuming too much B12 natural or synthetic
                        \item Deficiency can cause macrocytic anemia (because folate can't be utilized properly)
                            \begin{itemize}
                                \item Lack of intrinsic factor causes pernicious anemia, involves nerve damage
                            \end{itemize}
                    \end{itemize}
            \end{itemize}

        \subsection{Vitamin C}
            \begin{itemize}
                \item Also known as \textbf{ascorbic acid}
                \item Function: coenzyme to synthesize and use certain amino acids
                    \begin{itemize}
                        \item Needed to make collagen, most abundant protein in body, present in connective tissue
                            \begin{itemize}
                                \item Important for healthy bones, skin, blood vessels, teeth
                            \end{itemize}
                        \item Also acts as an antioxidant
                        \item Helps absorb iron from plant foods
                        \item Breaks down histamine cause of inflammation
                        \item Helps to maintain a strong immune system
                    \end{itemize}
                \item Daily needs:
                    \begin{itemize}
                        \item Men: 90 mg/day
                        \item Women: 75/day
                        \item Smokers: 35+ mg/day
                    \end{itemize}
                \item Food sources: fruits and vegetables (tomatoes, peppers, broccoli, oranges, cantaloupe)
                \item Too much or too little:
                    \begin{itemize}
                        \item UL = 2,000 mg/day to avoid nausea, stomach cramps, diarrhea
                            \begin{itemize}
                                \item People with a history of kidney stones or \textbf{hemochromatosis} (body stores too much iron) should avoid excess
                            \end{itemize}
                        \item Deficiency disease: \textbf{scurvy}
                    \end{itemize}
            \end{itemize}

        \subsection{Pantothenic Acid and Biotin}
            \begin{itemize}
                \item Functions: assist in energy metabolism of carbohydrates, fats, proteins
                \item Daily needs for adults:
                    \begin{itemize}
                        \item Daily needs for adults:
                            \begin{itemize}
                                \item Pantothenic acid: 5 mg/day
                                \item Biotin: 30 $\mu$g/day
                            \end{itemize}
                        \item Food sources:
                            \begin{itemize}
                                \item Widespread in foods such as whole grains and cereals, nuts, legumes, peanut butter, meat, milk, eggs
                                \item Biotin also synthesized by intestinal bacteria
                            \end{itemize}
                    \end{itemize}
                \item Too much or too little:
                    \begin{itemize}
                        \item No UL, no known adverse effects from consuming too much of either vitamin
                        \item Deficiencies of these vitamins are rare
                            \begin{itemize}
                                \item "Burning feet" syndrome seen in WWII prisoners of war in Asia due to pantothenic acid-deficient diet of polished rice
                                \item Biotin deficiency: hair loss, skin rash, fatigue, nausea, depression
                                    \begin{itemize}
                                        \item \textbf{Avidin} protein in raw egg whites binds biotin, preventing absorption
                                    \end{itemize}
                            \end{itemize}
                    \end{itemize}
            \end{itemize}

        \subsection{Are There Other Important Nutrients?}
            \begin{itemize}
                \item \textbf{Choline:} essential nutrient needed for healthy cells and nerves
                    \begin{itemize}
                        \item Not classified as a vitamin; body can synthesize it, but dietary sources may be needed
                        \item Daily needs: men: 550 mg; women: 425 mg
                        \item Widely available in foods: milk, eggs, peanuts, liver
                        \item UL of 3,500 mg/day to prevent \textbf{hypotension}, sweating, vomiting, fishy odor
                    \end{itemize}
                \item \textbf{Carnitine, lipoic acid, inositol} are not essential because body can synthesize adequate amounts
            \end{itemize}

        \subsection{Examining the Evidence: Myths and Facts about the Common Cold}
            \begin{itemize}
                \item The truth about catching a cold:
                    \begin{itemize}
                        \item Direct or indirect contact with cold virus
                    \end{itemize}
                \item Vitamin C and the common cold
                    \begin{itemize}
                        \item Research shows vitamin C to be ineffective in preventing colds, but may reduce severity in some people
                        \item Other cold remedies (echinacea, zinc): jury is still out
                            \begin{itemize}
                                \item Zinc may have some benefits
                            \end{itemize}
                    \end{itemize}
                \item What you can do: wash hands frequently in soap and water to reduce risk of cold
            \end{itemize}

        \subsection{How Should You Get Your Vitamins?}
            \begin{itemize}
                \item Food is still the best way to meet your vitamin needs
                    \begin{itemize}
                        \item Dietary Guidelines recommend a variety of foods and increased amounts of fruits, vegetables, whole grains, lean dairy to meet needs
                    \end{itemize}
                \item Fortified foods can provide additional nutrients but should not displace vitamin-/mineral-rich foods
                    \begin{itemize}
                        \item Excessive use of fortified foods can increase risk of overconsumption of some nutrients
                    \end{itemize}
                \item Vitamin supplements are not a substitute for healthy eating
                    \begin{itemize}
                        \item Cannot provide all missing substances of a healthy diet
                    \end{itemize}
                \item Who might benefit from a supplement?
                    \begin{itemize}
                        \item People who cannot meet their needs through a regular, varied diet, such as pregnant or lactating women; older people; strict vegetarians; people with food allergies, with medical conditions, or on low-calorie diets
                    \end{itemize}
                \item FDA approval not required for ingredients in use prior to 1994; FDA cannot remove supplement from marketplace until shown to be harmful
                \item Consult health professional before taking vitamin/mineral supplements
                    \begin{itemize}
                        \item Read supplement label carefully
                            \begin{itemize}
                                \item \textbf{U.S. Pharmacopoeia (USP)} seal of approval ensures quality and safety, but does not endorse or validate health claims
                            \end{itemize}
                    \end{itemize}
            \end{itemize}

    \section{Chapter 8: Minerals and Water}
            
        \subsection{Why Is Water so Important?}
            \begin{itemize}
                \item Water is the most abundant substance in body
                    \begin{itemize}
                        \item Average healthy adult is about 60 percent water
                            \begin{itemize}
                                \item Muscle tissue is 75 percent water, fat up to 20 percent
                            \end{itemize}
                        \item Can survive only a few days without water
                        \item Water is balanced among fluid compartments
                            \begin{itemize}
                                \item \textbf{Intracellular fluids:} inside cells
                                \item \textbf{Extracellular fluids:} interstitial fluid between cells and fluid in the blood
                            \end{itemize}
                        \item \textbf{Electrolytes:} minerals that help maintain fluid balance
                    \end{itemize}
                \item Acts as universal solvent and transport medium
                    \begin{itemize}
                        \item Medium for chemical reactions in body
                        \item As part of blood, helps transport oxygen, nutrients, hormones to cells
                        \item As part of interstitial fluid, helps transport waste products away from cells for excretion
                    \end{itemize}
                \item Helps maintain body temperature
                \item Lubricant for joints, eyes; part of mucus and saliva
                \item Protective cushion for brain, organs, fetus
            \end{itemize}

        \subsection{What Is Water Balance and How Do You Maintain It?}
            \begin{itemize}
                \item \textbf{Water balance:} water consumed = water lost
                \item You take in water through beverages and food
                \item You lose water through your kidneys (as urine), large intestine, lungs, skin
                    \begin{itemize}
                        \item \textbf{Insensible water loss:} through evaporation from skin and when you exhale
                        \item \textbf{Sensible water loss:} through urine, feces, and sweat
                    \end{itemize}
            \end{itemize}

        \subsection{Losing Too Much Water Can Cause Dehydration}
            \begin{itemize}
                \item Dehydration can result from inadequate water intake or too much water loss from diarrhea, vomiting, high fever, or use of diuretics
                \item Your thirst mechanism signals dehydration
                    \begin{itemize}
                        \item Dry mouth due to increased electrolyte concentration in blood: less water available to make saliva
                        \item Blood volume decreases, sodium concentration increases in blood
                            \begin{itemize}
                                \item Brain triggers thirst mechanism and secretion of antidiuretic hormone (ADH) to reduce urine output
                                \item Fluid inside cells moves into blood by osmosis
                            \end{itemize}
                    \end{itemize}
                \item Other ways to tell if you're dehydrated:
                    \begin{itemize}
                        \item Cornerstone method: measure body weight before and after exercise
                            \begin{itemize}
                                \item Weight loss = water loss
                            \end{itemize}
                        \item Monitor urine color
                            \begin{itemize}
                                \item Color darkens with concentration, indicating water loss
                            \end{itemize}
                    \end{itemize}
            \end{itemize}
            
        \subsection{Consuming Too Much Water Can Cause Hyponatremia}
            \begin{itemize}
                \item \textbf{Hyponatremia} is a condition of too little sodium in the blood
                \item For healthy individuals who consume a balanced diet, it is difficult to consume too much water
                \item However, some individuals have experienced water toxicity
                    \begin{itemize}
                        \item Examples: soldiers in training, endurance athletes
                    \end{itemize}
            \end{itemize}

        \subsection{How Much Water Do You Need and What Are the Best Sources?}
            \begin{itemize}
                \item Daily water needs depend on physical activity, environmental factors, diet
                \item Recommendations based on reported total water intake of healthy Americans
                \item Men: 16 cups/day (about 13 cups of beverages)
                \item Women: 12 cups/day (about 9 cups of beverages)
                    \begin{itemize}
                        \item About 80 percent from beverages, 20 percent from foods
                        \item Physical activity increases needs
                    \end{itemize}
            \end{itemize}

        \subsection{Nutrition in the Real World: Tap Water or Bottled Water: Is Bottled Better?}
            \begin{itemize}
                \item False assumption: bottled water is purer than tap water
                \item Tap water is perfectly safe
                    \begin{itemize}
                        \item Monitored by Environmental Protection Agency (EPA)
                        \item Provides fluoride, helps prevent dental caries
                    \end{itemize}
                \item Bottled water is very popular
                    \begin{itemize}
                        \item Most products conform to FDA requirements
                        \item May actually be tap water
                        \item High cost
                        \item Various "designer" waters on the market
                    \end{itemize}
            \end{itemize}

        \subsection{What Are Minerals and Why Do You Need Them?}
            \begin{itemize}
                \item \textbf{Inorganic} elements needed in relatively small amounts
                \item Mineral absorption depends on \textbf{bioavailability}
                    \begin{itemize}
                        \item Some minerals compete for absorption: too much of one can decrease absorption of another
                            \begin{itemize}
                                \item Example: excess zinc can reduce copper absorption
                            \end{itemize}
                        \item Some substances bind minerals, making them unavailable for absorption
                            \begin{itemize}
                                \item Example: oxalates in spinach bind calcium
                            \end{itemize}
                    \end{itemize}
                \item \textbf{Major minerals} (macrominerals): needed in amounts greater than 100 mg/day
                \item \textbf{Trace minerals} (microminerals): needed in amounts less than 20 mg/day
                \item You need major minerals in larger amounts
                    \begin{itemize}
                        \item Sodium, chloride, potassium, magnesium, sulfur play key roles in fluid balance
                        \item Calcium, phosphorus, magnesium work together to strengthen bones and teeth
                    \end{itemize}
                \item Trace minerals are needed in small amounts
                    \begin{itemize}
                        \item Play essential roles as important as major minerals
                        \item Chromium and iodine help certain hormones
                        \item Iron maintains healthy red blood cells
                        \item Fluoride protects teeth
                        \item Iron, zinc, copper, manganese, and molybdenum are cofactors that work with enzymes in critical chemical reactions
                    \end{itemize}
                \item Overconsumption of minerals can be toxic
                    \begin{itemize}
                        \item Difference between recommended and excessive amount may be minimal
                            \begin{itemize}
                                \item Example: magnesium, which can cause gastrointestinal problems
                            \end{itemize}
                    \end{itemize}
                \item Foods alone rarely provide excessive amounts
                    \begin{itemize}
                        \item Problems usually arise with supplements
                        \item Another good reason to eat a varied diet
                    \end{itemize}
            \end{itemize}
            
        \subsection{Exploring Sodium}
            \begin{itemize}
                \item What are sodium and salt?
                    \begin{itemize}
                        \item Sodium is an electrolyte (charged ion) in blood and in the fluid surrounding cells
                        \item About 90 percent of sodium consumed is in form of sodium chloride, table salt
                    \end{itemize}
                \item Functions: chief role is regulation of fluid balance
                    \begin{itemize}
                        \item Also transports substances such as amino acids across cell membranes
                    \end{itemize}
                \item Sodium balance in your body
                    \begin{itemize}
                        \item Sodium levels is maintained by the kidneys reducing or increasing sodium excretion as needed
                        \item Smaller amounts lost in stool and sweat
                    \end{itemize}
                \item Daily needs: 1,500 mg/day for adults under 51
                \item Food sources: 75 percent of sodium consumed by Americans comes from processed foods
                    \begin{itemize}
                        \item About 10 percent occurs naturally in foods; another 5-10 percent added during cooking and at table
                    \end{itemize}
                \item Too much or too little:
                    \begin{itemize}
                        \item UL for adults is set at 2,300 mg/day to reduce the risk of \textbf{hypertension} (high blood pressure)
                            \begin{itemize}
                                \item Cut back on processed foods and salt added to foods to lower sodium intake
                            \end{itemize}
                        \item Sodium deficiency is rare in healthy individuals consuming a balanced diet
                    \end{itemize}
            \end{itemize}

        \subsection{You and Your Blood Pressure}
            \begin{itemize}
                \item Blood pressure: a measure of force that blood exerts on the walls of arteries
                \item Expressed as \textbf{systolic pressure} (when heart beats) over \textbf{diastolic pressure} (when heart is at rest between beats)
                    \begin{itemize}
                        \item < 120/80 mmHg is normal
                        \item Systolic > 120 or diastolic > 80 = prehypertension
                        \item >= 140/90 = hypertension
                    \end{itemize}
                \item \textbf{Hypertension} is a silent killer
                    \begin{itemize}
                        \item No symptoms - have blood pressure checked regularly
                        \item Contributes to atherosclerosis, heart enlarges and weakens
                        \item Damages arteries leading to brain and kidneys, increasing risk of stroke and kidney disease
                    \end{itemize}
                \item To control hypertension:
                    \begin{itemize}
                        \item Reduce weight, increase physical activity, eat a balanced diet
                    \end{itemize}
            \end{itemize}

        \subsection{Exploring Potassium}
            \begin{itemize}
                \item Important mineral with many functions:
                    \begin{itemize}
                        \item Fluid balance: electrolyte inside cells
                        \item A blood buffer: helps keep blood pH and acid-base balance correct
                        \item Muscle contraction and nerve impulse conduction
                        \item Can help lower high blood pressure
                        \item Aids in bone health: helps increase bone density
                        \item Reduces kidney stones by helping to excrete citrate (binds with calcium to form \textbf{kidney stones})
                    \end{itemize}
                \item Daily needs:
                    \begin{itemize}
                        \item Adults: 4,700 mg/day
                        \item Adult females consume only about 2,400 mg/day and adult males only 3,170 mg/day on average
                    \end{itemize}
                \item Food sources:
                    \begin{itemize}
                        \item Fruits and vegetables
                            \begin{itemize}
                                \item Minimum of 4.5 cups/day will help meet potassium needs
                            \end{itemize}
                        \item Dairy foods, nuts, legumes also good sources
                    \end{itemize}
                \item Too much or too little:
                    \begin{itemize}
                        \item Too much from supplements or salt substitutes can cause \textbf{hyperkalemia} in some individuals
                            \begin{itemize}
                                \item Can cause irregular heartbeats, damage heart, and be life-threatening
                            \end{itemize}
                        \item Too little can cause \textbf{hypokalemia}
                            \begin{itemize}
                                \item Can cause muscle weakness, cramps, irregular heartbeats, and paralysis
                                \item Can occur as result of excessive vomiting and/or diarrhea, anorexia and/or bulimia eating disorders
                            \end{itemize}
                    \end{itemize}
            \end{itemize}

        \subsection{Exploring Calcium}
            \begin{itemize}
                \item Most abundant mineral in body
                    \begin{itemize}
                        \item More than 99 percent located in bones and teeth
                    \end{itemize}
                \item Functions:
                    \begin{itemize}
                        \item Helps build strong bones and teeth
                        \item Plays a role in muscles, nerves, and blood
                        \item May help lower high blood pressure
                        \item May fight colon cancer
                        \item May reduce risk of kidney stones (though supplements have opposite effect)
                    \end{itemize}
                \item Daily needs:
                    \begin{itemize}
                        \item 1,000 to 1,200 mg/day, depending on age
                    \end{itemize}
                \item Food sources:
                    \begin{itemize}
                        \item Milk, yogurt, cheese, broccoli, kale, canned salmon (with bones), tofu processed with calcium, calcium-fortified juices and cereals
                    \end{itemize}
                \item Too much or too little:
                    \begin{itemize}
                        \item UL: 2,500 mg/day (ages 19-50); 2,000 mg (51+)
                        \item Too much calcium leads to hypercalcemia: impaired kidneys, calcium deposits in body
                        \item Too little can lead to less dense, weakened, brittle bones, and increased risk for osteoporosis
                    \end{itemize}
                \item Calcium supplements:
                    \begin{itemize}
                        \item Consume in doses of 500mg or less
                        \item Some sources (oyster shell, bone meal, dolomite) may contain lead, other toxic metals
                        \item May be inadvisable if consuming enough in foods
                    \end{itemize}
            \end{itemize}
        
        \subsection{Osteoporosis: Not Just Your Grandmother's Problem}
            \begin{itemize}
                \item Bones are living tissue, constantly changing
                \item Peak bone mass occurs in early adulthood (20s)
                    \begin{itemize}
                        \item Then slowly more bone is lost than added
                        \item As bones lose mass, they become more porous and prone to fractures, leading to osteoporosis
                    \end{itemize}
                \item \textbf{Bone mineral density (BMD)} test measaures bone density
                    \begin{itemize}
                        \item Low score = \textbf{osteopenia} (low bone mass)
                        \item Very low score = \textbf{osteoporosis}
                    \end{itemize}
                \item Risk factors:
                    \begin{itemize}
                        \item Gender (females at higher risk due to loss of estrogen after menopause)
                        \item Ethnicity (Caucasian and Asian-American at higher risk)
                        \item Age (over 30)
                        \item Body type (small-boned/petite women at higher risk)
                        \item Family history of fractures increases risk
                        \item Level of sex hormones (amenorrhea, menopause, or mena with low levels of sex hormones)
                        \item Medications: glucocorticoids, antiseizure medications, aluminum-containing antacids, high amounts of thyroid replacement hormones
                        \item Smoking
                        \item Low physical activity: 30 minutes per day recommended
                        \item Alcohol (more than one drink for women, two for men)
                        \item Inadequate calcium and vitamin D (less than three cups/day of vitamin D-fortified milk or yogurt)
                    \end{itemize}
            \end{itemize}

        \subsection{Exploring Phosphorus}
            \begin{itemize}
                \item Second most abundant mineral in body
                    \begin{itemize}
                        \item 85 percent in bones; rest in cells and fluids outside cells, including blood
                    \end{itemize}
                \item Functions:
                    \begin{itemize}
                        \item Needed for bones and teeth
                        \item Important component of cell membranes
                        \item Needed for energy metabolism and stores
                        \item Acts as a blood buffer
                        \item Part of DNA and RNA
                    \end{itemize}
                \item Daily needs:
                    \begin{itemize}
                        \item Adults: 700 mg/day
                    \end{itemize}
                \item Food sources:
                    \begin{itemize}
                        \item Meat, fish, poultry, dairy
                        \item Abundant in diet
                    \end{itemize}
                \item Too much or too little:
                    \begin{itemize}
                        \item UL set at 4,000 mg/day for adults 19 to 50 to prevent \textbf{hyperphosphatemia}, which can lead to calcification of tissues; 3,000 mg for those aged 51 or older
                        \item Too little can result in muscle weakness, bone pain, rickets, confusion, death; would need to be in state of near starvation to experience deficiency
                    \end{itemize}
            \end{itemize}
    
        \subsection{Exploring Magnesium}
            \begin{itemize}
                \item Another abundant mineral in body
                    \begin{itemize}
                        \item About half in bones; most of rest inside cells
                    \end{itemize}
                \item Functions:
                    \begin{itemize}
                        \item Helps more than 300 enzymes, including energy metabolism
                        \item Used in synthesis of protein
                        \item Helps muscles and nerves function properly
                        \item Maintains healthy bones and regular heartbeat
                        \item May help lower high blood pressure and reduce risk of type 2 diabetes
                    \end{itemize}
                \item Daily needs:
                    \begin{itemize}
                        \item 19 to 30 years: males, 400 mg/day; females, 310 mg/day
                        \item > 30 years: males, 420 mg/day; females, 320 mg/day
                        \item Many Americans fall short (80 to 85 percent of needs)
                    \end{itemize}
                \item Food sources:
                    \begin{itemize}
                        \item Whole grains, vegetables, nuts, fruits; also milk, yogurt, meat, eggs
                    \end{itemize}
                \item Too much or too little:
                    \begin{itemize}
                        \item UL from supplements (not foods) = 350 mg/day to avoid diarrhea
                        \item Deficiencies are rare, but diuretics and some antibiotics can inhibit absorption
                    \end{itemize}
            \end{itemize}

        \subsection{Exploring Chloride}
            \begin{itemize}
                \item Chloride is part of hydrochloric acid in the stomach, which enhances protein digestion
                \item Functions:
                    \begin{itemize}
                        \item Sodium and chloride are major electrolytes outside cells and in blood to help maintain fluid balance
                        \item Acts as buffer to keep blood at normal pH
                    \end{itemize}
                \item Daily needs: adults: 2,300 mg/day
                \item Food sources: salt (NaCl) is main source
                \item Too much or too little: deficiencies are rare
                    \begin{itemize}
                        \item UL 3,600 mg/day to match sodium UL
                    \end{itemize}
            \end{itemize}

        \subsection{Exploring Sulfur}
            \begin{itemize}
                \item Component of other compounds in body, including the vitamins thiamin, biotin, pantothenic acid
                \item Functions:
                    \begin{itemize}
                        \item Helps give proteins 3-D shape as part of amino acids methionine, cystine, and cysteine
                        \item Sulfites used as food preservative
                    \end{itemize}
                \item Food sources: meat, poultry, fish, eggs, legumes, dairy, fruits, vegetables
                \item Too much or too little: no known toxicity or deficiency symptoms
            \end{itemize}
        
        \subsection{Exploring Iron}
            \begin{itemize}
                \item Most abundant mineral on Earth and main trace mineral in body
                \item Two forms: heme and nonheme iron
                    \begin{itemize}
                        \item Heme iron from animal sources is part of hemoglobin and myoglobin and easily absorbed
                        \item Nonheme iron in plant foods is not as easily absorbed, due to phytates and other substances
                        \item Body absorbs only 10 to 15 percent of iron consumed
                        \item Absorption increases if body stores are low
                        \item Not excreted in urine or stool; once absorbed, very little leaves body (95 percent recycled, reused)
                    \end{itemize}
                \item Functions:
                    \begin{itemize}
                        \item \textbf{Hemoglobin} in red blood cells transports oxygen from lungs to tissues and picks up carbon dioxide waste from cells
                        \item Myoglobin transports and stores oxygen in muscle cells
                        \item Aids brain function by helping enzymes that make neurotransmitters
                    \end{itemize}
                \item Daily needs:
                    \begin{itemize}
                        \item Men and women > 50: 8 mg/day
                        \item Women 19 to 50: 18 mg/day: higher due to iron lost during menstruation
                    \end{itemize}
                \item Food sources:
                    \begin{itemize}
                        \item Iron-enriched bread and grain products; heme iron in meats, fish, and poultry
                    \end{itemize}
                \item Too much or too little:
                    \begin{itemize}
                        \item Too much iron from supplements can cause constipation, nausea, vomiting, diarrhea
                        \item In United States, a leading cause of accidental poisoning deaths in children under 6 years
                        \item iron overload can damage heart, kidneys, liver, nervous system
                        \item \textbf{Hemochromatosis}, a genetic disorder, can cause iron overload
                        \item Iron deficiency: most common nutritional disorder in world
                        \item \textbf{Iron-deficiency anemia} occurs when iron stores are depleted and hemoglobin levels decrease
                    \end{itemize}
            \end{itemize}

        \subsection{Exploring Copper}
            \begin{itemize}
                \item Functions:
                    \begin{itemize}
                        \item Part of many enzymes and proteins
                        \item Important for iron absorption and transfer, synthesis of hemoglobin and red blood cells
                        \item Helps generate energy in cells, synthesize melanin, link the proteins collagen and elastin together in connective tissues
                        \item Helps enzymes protect cells from free radicals
                        \item Role in blood clotting and maintaining healthy immune system
                    \end{itemize}
                \item Daily needs:
                    \begin{itemize}
                        \item Adults: 900 micrograms/day
                    \end{itemize}
                \item Food sources:
                    \begin{itemize}
                        \item Organ meats, seafood, nuts, seeds, bran cereals, whole-grain products, cocoa
                    \end{itemize}
                \item Too much or too little:
                    \begin{itemize}
                        \item UL: 10,000 micrograms/day
                        \item Excess can cause stomach cramps, nausea, diarrhea, vomiting, liver damage
                        \item Copper deficiency rare in United States
                    \end{itemize}
            \end{itemize}

        \subsection{Exploring Zinc}
            \begin{itemize}
                \item Involved in function of more than 100 enzymes
                \item Functions:
                    \begin{itemize}
                        \item DNA synthesis, growth, and development
                        \item Healthy immune system and wound healing
                        \item Taste acuity
                        \item Treatment for common cold
                        \item May reduce risk of age-related macular degeneration
                    \end{itemize}
                \item Daily needs:
                    \begin{itemize}
                        \item Men: 11 mg/day; women: 8mg/day
                        \item Vegetarians may need as much as 50 percent more
                    \end{itemize}
                \item Food sources:
                    \begin{itemize}
                        \item Red meat, some seafood, whole grains
                    \end{itemize}
                \item Too much or too little:
                    \begin{itemize}
                        \item UL = 40 mg/day
                        \item As little as 50mg can cause stomach pains, nausea, vomiting, diarrhea
                        \item 60 mg/day can inhibit copper absorption
                        \item Excessive amounts can suppress immune system, lower HDL cholesterol
                        \item Deficiency: hair loss, impaired taste, loss of appetite, diarrhea, delayed sexual maturation, impotence, skin rashes, impaired growth
                    \end{itemize}
            \end{itemize}

        \subsection{Exploring Selenium}
            \begin{itemize}
                \item Part of class of proteins called selenoproteins, many of which are enzymes
                \item Functions of selenoproteins:
                    \begin{itemize}
                        \item Help regulate thyroid hormones
                        \item Act as antioxidants
                        \item May help fight cancer
                    \end{itemize}
                \item Daily needs: adults: 55 micrograms/day
                \item Food sources: meat, seafood, cereal, grains, dairy foods, fruits, vegetables
                    \begin{itemize}
                        \item Amount varies depending on soil content
                    \end{itemize}
                \item Too much or too little:
                    \begin{itemize}
                        \item UL = 400 micrograms/day
                        \item Too much can cause toxic condition \textbf{selenosis}
                            \begin{itemize}
                                \item Symptoms: brittleness and loss of nails and hair, stomach and intestinal discomfort, skin rash, garlicky breath, fatigue, nervous system damage
                            \end{itemize}
                        \item Selenium deficiency is rare in United States
                            \begin{itemize}
                                \item Deficiency can cause \textbf{Keshan disease} (heart damage): seen in children in rural areas that have selenium-poor soils
                            \end{itemize}
                    \end{itemize}
            \end{itemize}

        \subsection{Exploring Fluoride}
            \begin{itemize}
                \item Functions:
                    \begin{itemize}
                        \item Protects against dental caries
                            \begin{itemize}
                                \item Helps repair enamel eroded by acids from bacteria
                                \item Reduces amount of acid bacteria produce
                                \item Provides protective barrier
                            \end{itemize}
                        \item Fluoridated drinking water has reduced dental caries in United States
                    \end{itemize}
                \item Daily needs:
                    \begin{itemize}
                        \item Men: 3.8 mg/day
                        \item Women: 3.1 mg/day
                    \end{itemize}
                \item Sources: food are not a good source
                    \begin{itemize}
                        \item Best source is fluoridated drinking water and beverages made with this water
                    \end{itemize}
                \item Too much or too little:
                    \begin{itemize}
                        \item Too little increases risk of dental caries
                        \item Too much can cause \textbf{fluorosis} (mottling/staining) when teeth are forming during infancy/childhood
                            \begin{itemize}
                                \item Fluorosis of bones can occur when > 10 mg/day is consumed for 10 or more years
                            \end{itemize}
                        \item UL: adults: 10 mg/day, much lower for infants and children
                    \end{itemize}
            \end{itemize}

        \subsection{Exploring Chromium}
            \begin{itemize}
                \item Functions:
                    \begin{itemize}
                        \item Helps insulin in your body
                            \begin{itemize}
                                \item Increases effectiveness in cells
                            \end{itemize}
                    \end{itemize}
                \item May improve blood glucose control, but no large study confirms this theory
                    \begin{itemize}
                        \item Small study suggests chromium supplement may reduce risk of insulin resistance
                        \item FDA allows a \textbf{Qualified Health Claim} on chromium supplements, but label must state that evidence is not certain
                        \item Does not help build muscle mass
                    \end{itemize}
                \item Daily needs:
                    \begin{itemize}
                        \item Men: 30 to 35 micrograms
                        \item Women: 20 to 25 micrograms
                    \end{itemize}
                \item Food sources: grains, meat, fish, poultry, some fruits and vegetables
                \item Too much or too little:
                    \begin{itemize}
                        \item No known risk from consuming too much
                        \item Deficiency is rare in United States
                    \end{itemize}
            \end{itemize}

        \subsection{Exploring Iodine}
            \begin{itemize}
                \item Functions: needed by thyroid to make essential hormones
                    \begin{itemize}
                        \item Thyroid hormones regulate metabolic rate; help heart, nerves, muscle and intestine function properly
                    \end{itemize}
                \item Daily needs: adults: 150 micrograms/day
                \item Food sources: iodized salt (400 micrograms/tsp)
                    \begin{itemize}
                        \item Amount in foods in low; depends on iodine content of soil, water, fertilizer
                        \item Salt-water fish have higher amounts
                    \end{itemize}
                \item Too much or too little: UL = 1,100 micrograms/day
                    \begin{itemize}
                        \item Excess iodine can impair thyroid function, decrease synthesis and release of thyroid hormones
                        \item Early sign of deficiency = \textbf{goiter} (enlarged thyroid gland)
                            \begin{itemize}
                                \item Mandatory iodization salt has decreased iodine deficiency in Untied States but not in other parts of world
                                \item Iodine deficiency during early stages of fetal development can cause \textbf{cretinism} (congenital hypothyroidism)
                            \end{itemize}
                    \end{itemize}
            \end{itemize}

        \subsection{Exploring Manganese}
            \begin{itemize}
                \item Part of, or activities, many enzymes, in body
                \item Functions:
                    \begin{itemize}
                        \item Helps metabolize carbohydrates, fats, amino acids
                        \item Aids bone formation
                    \end{itemize}
                \item Daily needs:
                    \begin{itemize}
                        \item Men: 2.3 mg/day
                        \item Women: 1.8 mg/day
                    \end{itemize}
                \item Food sources: Whole grains, nuts, legumes, tea, vegetables, pineapples, strawberries, bananas
                \item Too much or too little:
                    \begin{itemize}
                        \item UL = 11 mg/day to avoid toxicity with Parkinson's disease-like symptoms
                    \end{itemize}
            \end{itemize}

        \subsection{Exploring Molybdenum}
            \begin{itemize}
                \item Functions: part of several enzymes involved in breakdown of certain amino acids and other compounds
                \item Daily needs: adults: 45 micrograms/day
                \item Food sources: legumes, grains, nuts
                \item Too much or too little:
                    \begin{itemize}
                        \item UL = 2 mg/day, based on animal studies in which too much molybdenum caused reproductive problems
                        \item No cases seen in healthy individuals
                    \end{itemize}
            \end{itemize}

        \subsection{Other Minerals}
            \begin{itemize}
                \item Arsenic, boron, nickel, silicon, and vanadium
                    \begin{itemize}
                        \item Exist in body but essential role in humans not established by research
                        \item May have function for some animals
                        \item Tolerable upper levels set for:
                            \begin{itemize}
                                \item Boron: 20 mg/day (10 times more than average American consumes)
                                \item Nickel: 1mg/day
                                \item Vanadium: 1.8 mg/day
                            \end{itemize}
                    \end{itemize}
            \end{itemize}

    \section{Chapter 9: Alcohol}
        
        \subsection{What Is Alcohol and How Is It Made?}
            \begin{itemize}
                \item \textbf{Alcohol} is \textbf{not} an essential nutrient
                \item \textbf{Ethanol} is the type of alcohol consumed in alcoholic beverages
                    \begin{itemize}
                        \item Methanol (in antifreeze) and isopropanol (rubbing alcohol) are both poisonous to humans
                        \item Ethanol is safe for consumption, but excessive amounts are toxic and can be fatal
                        \item Made by \textbf{fermentation} of yeast and natural sugars in grains (beer) and fruits (wine)
                            \begin{itemize}
                                \item Liquor is concentrated alcohol collected through \textbf{distillation}
                            \end{itemize}
                    \end{itemize}
            \end{itemize}
        
        \subsection{Why Do People Drink Alcohol?}
            \begin{itemize}
                \item People drink to relax, celebrate, and socialize
                \item \textbf{Social drinking:} drinking patterns that are considered acceptable by society
                \item Moderate alcohol consumption may have health benefits: may reduce risk of heart disease
                    \begin{itemize}
                        \item \textbf{Moderate alcohol consumption:} no more than one drink daily for adult women, two for men
                        \item Alcohol can increase HDL cholesterol and may make blood platelets less "sticky": less likely to form unwanted blood clots
                        \item Health benefits only shown in women >= 55 years of age and men >= 45 years old, not in younger people
                    \end{itemize}
                \item Moderate consumption is based on standard drink sizes, which contain about half an ounce of alcohol
                \item A standard drink is one of the following:
                    \begin{itemize}
                        \item 12-ounce serving of beer
                        \item 1.5-ounce shot of liquor
                        \item 5-ounce glass of wine
                    \end{itemize}
                \item Moderate drinkers should pay attention to:
                    \begin{itemize}
                        \item Size of drinks
                        \item Frequency of drinking
                            \begin{itemize}
                                \item Abstaining from alcohol for several days, then overdrinking one day is \textbf{not} moderate drinking
                            \end{itemize}
                    \end{itemize}
            \end{itemize}

        \subsection{What Happens to Alcohol in the Body?}
            \begin{itemize}
                \item Alcohol is a toxin, and the body works quickly to metabolize and eliminate it
                \item You absorb alcohol in your stomach and small intestine
                    \begin{itemize}
                        \item Some alcohol is metabolized by \textbf{alcohol dehydrogenase} enzyme in the stomach before it's absorbed
                        \item Women are more susceptible to effects of alcohol than men
                            \begin{itemize}
                                \item Have 20 to 30 percent less alcohol dehydrogenase than men, so absorb more alcohol in stomach
                            \end{itemize}
                        \item Food in stomach slows alcohol absorption
                        \item About 80 percent of alcohol is absorbed in the small intestine
                    \end{itemize}
                \item You metabolize alcohol primarily in your liver: one standard drink is metabolized in 1.5 to 2 hours
                    \begin{itemize}
                        \item Alcohol dehydrogenase converts alcohol to \textbf{acetaldehyde} (eventually metabolized to CO2 and water)
                        \item The \textbf{microsomal ethanol-oxidizing system} (MEOS) also metabolizes alcohol and is revved up when chronically high levels of alcohol are present in liver
                    \end{itemize}
                \item Alcohol circulates in your blood until metabolized
                    \begin{itemize}
                        \item \textbf{Blood alcohol concentration (BAC)} correlates with amount of alcohol in your breath
                    \end{itemize}
                \item Effects of alcohol on your brain
                    \begin{itemize}
                        \item Depressant of central nervous system
                        \item Slows down transmission of nerve impulses and reaction time to stimuli
                        \item Impairs thoughts, actions, behavior
                        \item The more consumed, the more areas of brain affected
                        \item If enough consumed, activities of brain stem are suppressed (breathing, heart rate), resulting in death
                    \end{itemize}
            \end{itemize}

        \subsection{How Can Alcohol Be Harmful?}
            \begin{itemize}
                \item Alcohol can disrupt sleep and cause hangovers
                    \begin{itemize}
                        \item Even moderate amount in late afternoon/evening can disrupt sleep cycle
                    \end{itemize}
                \item Alcoholic beverages may contain \textbf{congeners}, which contribute to hangover symptoms
                    \begin{itemize}
                        \item Symptoms: headache, fatigue, nausea, increased thirst, rapid heart beat, tremors, sweating, dizziness, depression, anxiety, irritability
                    \end{itemize}
                \item Alcohol is a diuretic; can cause dehydration and electrolyte imbalances
                \item Alcohol can interact with hormones
                    \begin{itemize}
                        \item Interferes with insulin and glucagon that regulate blood glucose level
                        \item Negatively affects parathyroid hormone and other bone-strengthening hormones; can increase risk of osteoporosis
                        \item Can increase estrogen levels in women; may increase risk of breast cancer
                        \item Affects reproductive hormones and associated with both male and female sexual dysfunction
                        \item Alcohol may lead to overnutrition and malnutrition
                            \begin{itemize}
                                \item Provides 7 calories per gram, contributing to weight gain
                                    \begin{itemize}
                                        \item Increases fat and weight around stomach
                                    \end{itemize}
                                \item Alcohol calories can displace nutritious foods
                                \item Excessive alcohol can interfere with absorption and/or use of protein, zinc, magnesium, thiamin, folate, and vitamins B12, A, D, E, K
                                    \begin{itemize}
                                        \item Thiamin deficiency affects brain function and increases risk of Wernicke-Korsakoff syndrome
                                    \end{itemize}
                            \end{itemize}
                    \end{itemize}
                \item Alcohol can harm your digestive organs, heart, and liver
                \item Excessive amounts of alcohol can cause:
                    \begin{itemize}
                        \item Inflammation of esophagus
                        \item Cancers of the esophagus, mouth, and throat
                        \item \textbf{Gastritis} and stomach ulcers
                        \item Hyptertension and damage to heart tissue
                        \item Increased \textbf{endotoxin}
                        \item \textbf{Alcoholic liver disease}
                            \begin{itemize}
                                \item Three stages: \textbf{fatty liver, alcoholic hepatitis}, cirrhosis
                            \end{itemize}
                    \end{itemize}
                \item Alcohol can put a healthy pregnancy at risk
                    \begin{itemize}
                        \item Exposure to alcohol prenatally can cause \textbf{fetal alcohol spectrum disorders (FASDs)}
                            \begin{itemize}
                                \item Most severe form is fetal alcohol syndrome (FAS)
                                    \begin{itemize}
                                        \item Causes physical, mental, and behavioral abnormalities
                                    \end{itemize}
                                \item Effects of FASDs are permanent
                                \item The only proven, safe amount of alcohol a pregnant woman can consume is \textbf{none}
                            \end{itemize}
                    \end{itemize}
            \end{itemize}

        \subsection{What is Alcohol Use Disorder?}
            \begin{itemize}
                \item \textbf{Alcohol use disorder (AUD)} is the continuation of alcohol consumption even though this behavior has created social, psychological, and/or physical health problems.
                \item \textbf{Binge drinking}, drinking and driving, and underage drinking are situations in which alcohol is being abused.
                \item Binge drinking: consumption of 5 or more drinks by men, 4 by women, in about two hours
                    \begin{itemize}
                        \item Increased likelihood of injuries, car accidents, drowning, unplanned sexual activity, death
                        \item Associated with sexual aggression, assaults, suicide, homicide, child abuse, and health problems (hypertension, heart attack, sexually transmitted disease)
                    \end{itemize}
                \item Binge drinking
                    \begin{itemize}
                        \item Can cause \textbf{blackouts} and lead to alcohol poisoning
                        \item Chronic drinking can lead to alcohol tolerance
                            \begin{itemize}
                                \item Brain becomes less sensitive to alcohol, needing more to get same intoxicating effect
                            \end{itemize}
                    \end{itemize}
                \item Drinking and driving: illegal to drive with BAC of 0.08
                    \begin{itemize}
                        \item One drink impairs alertness, judgment, coordination
                    \end{itemize}
                \item Underage drinking
                    \begin{itemize}
                        \item Inreases risk of violence, injuries, health risks
                        \item Can also interfere with brain development and lead to cognitive and memory damage in teenagers
                        \item Underage drinking and driving is extremely risky
                        \item The earlier in life a person starts drinking, the higher the risk for alcoholism
                    \end{itemize}
            \end{itemize}

        \subsection{How to Get Help for AUD}
            \begin{itemize}
                \item Research has shown that support from a provider can reduce alcohol consumption for those with mild AUD
                \item Those with more severe AUD may need to seek specialized counseling as well as medical support
                \item Some individuals have found success with ongoing support programs such as Alcoholics Anonymous (AA)
            \end{itemize}
        
        \subsection{Some People Should Avoid Consuming Alcohol}
            \begin{itemize}
                \item According to the \textbf{Dietary Guidelines for Americans}, the following people should also abstain from drinking alcohol:
                    \begin{itemize}
                        \item Women of childbearing age who may become pregnant
                        \item Pregnant women
                        \item Children and adolescents
                        \item Those taking medications that can interact with alcohol, which include prescription and over-the-counter medications
                        \item Those with specific medical conditions, such as liver disease
                        \item Those engaging in activities that require attention, skill, or coordination, such as driving or operating machinery
                        \item Those who cannot restrict their alcohol intake
                    \end{itemize}
            \end{itemize}

    \section{Chapter 10: Weight Management and Energy Balance}
        
        \subsection{What Is a Healthy Weight and Why Is Maintaing It Important?}
            \begin{itemize}
                \item \textbf{Healthy weight:} body weight relative to height that does not increase the risk of developing weight-related health problems or diseases
                    \begin{itemize}
                        \item \textbf{Weight management:} maintaining weight within a healthy range
                        \item \textbf{Overweight:} 10 to 15 pounds more than healthy weight
                            \begin{itemize}
                                \item More than 70 percent of Americans are overweight
                            \end{itemize}
                        \item \textbf{Obesity:} 25 to 40 pounds more than healthy weight
                            \begin{itemize}
                                \item Over 37 percent of those Americans are obese
                                \item Classified as a disease by the AMA in 2013
                            \end{itemize}
                    \end{itemize}
                \item Being overweight increases risk of:
                    \begin{itemize}
                        \item Hypertension and stroke
                        \item Heart disease
                        \item Gallbladder disease
                        \item Type 2 diabetes
                        \item Osteoarthritis
                        \item Some cancers
                        \item Sleep apnea
                    \end{itemize}
                \item Losing 5 to 10 percent of body weight can produce health benefits
                    \begin{itemize}
                        \item Lower blood pressure, cholesterol, and glucose
                    \end{itemize}
                \item \textbf{Underweight:} weighing too little for your height
                    \begin{itemize}
                        \item May be caused by excessive calorie restriction and/or physical activity, underlying medical condition, emotional stress
                        \item Increases risk for osteoporosis
                        \item Risks for:
                            \begin{itemize}
                                \item Young adults: nutrient deficiencies, electrolyte imbalance, low energy levels, decreased concentration
                                \item Older adults: low body protein and fat stores, depressed immune system, medical complications
                            \end{itemize}
                    \end{itemize}
            \end{itemize}

        \subsection{How Do You Know If You're at a Healthy Weight?}
            \begin{itemize}
                \item BMI measurements can provide a general guideline:
                    \begin{itemize}
                        \item Body mass index %TODO ADD FORMULA
                            \begin{itemize}
                                \item BMI >= 25 is overweight: modest increase in risk of dying from diseases
                                \item >= 30 is obese: 50 to 100 percent higher risk of dying prematurely compared to healthy weight
                                \item < 18.5 is underweight; can also be unhealthy
                            \end{itemize}
                        \item May not be accurate for everyone
                        \item Does not directly measure body fat percentage
                    \end{itemize}
                \item Measure your body fat and its location
                    \begin{itemize}
                        \item Average healthy adult male between 20 and 49 years of age: 16 to 21 percent of weight is body fat
                        \item Average healthy female: 22 to 26 percent body fat
                    \end{itemize}
                \item Techniques to measure body fat include skinfold thickness measurements and bioelectrical impedance
                \item \textbf{Central obesity} (excess \textbf{visceral fat}) vs \textbf{subcutaneous fat}, increases risk of heart disease, diabetes, hypertension
                    \begin{itemize}
                        \item Measure waist circumference
                    \end{itemize}
                \item There are many ways to measure percentage of body fat
            \end{itemize}
        
        \subsection{What Is Energy Balance and What Determines Energy Needs?}
            \begin{itemize}
                \item Energy balance is calories in versus calories out
                    \begin{itemize}
                        \item \textbf{Positive energy balance:} more calories consumed than expended (leads to fat storage, weight gain)
                            \begin{itemize}
                                \item \textbf{Energy excess}
                            \end{itemize}
                        \item \textbf{Negative energy balance:} more calories expended than consumed (leads to weight loss)
                            \begin{itemize}
                                \item \textbf{Energy deficit}
                            \end{itemize}
                    \end{itemize}
                \item Energy needs are different for everyone
                    \begin{itemize}
                        \item Energy needs comprise:
                            \begin{itemize}
                                \item Basal metabolism
                                \item Thermic effect of food
                                \item Physical activities
                            \end{itemize}
                    \end{itemize}
                \item Your basal metabolic rate \textbf{(BMR)} is the minimum amount of energy you need to function
                    \begin{itemize}
                        \item Amount needed to meet basic physiological needs, keep you alive
                        \item Makes up about 60 percent of total energy needs
                        \item Many factors affect BMR, chiefly \textbf{lean body mass}
                    \end{itemize}
                \item The \textbf{thermic effect of food} affects your energy needs
                    \begin{itemize}
                        \item Amount of calories expended to digest, absorb, and process fodd (about 10 percent of calories in food eaten)
                    \end{itemize}
                \item Physical activity will increase your energy needs
                    \begin{itemize}
                        \item Energy expended by sedentary people = less than half of BMR
                        \item Very active athletes can expend twice BMR
                            \begin{itemize}
                                \item Exercise causes small increase in energy expenditure after activity has stopped
                            \end{itemize}
                    \end{itemize}
                \item Calculating your energy needs:
                    \begin{itemize}
                        \item \textbf{Estimated energy requirement (EER):} daily energy need based on age, gender, height, weight, activity level
                    \end{itemize}
            \end{itemize}

        \subsection{Energy Imbalances over Time Can Lead to Changes in Body Weight}
            \begin{itemize}
                \item Reducing calories can lead to weight loss
                    \begin{itemize}
                        \item Stored glycogen and fat are used as fuel sources
                            \begin{itemize}
                                \item Amino acids from body protein breakdown can be used to make glucose
                                \item Prolonged fast depletes all liver glycogen
                                \item Ketone bodies generated from incomplete breakdown of fat
                                \item Fat stores and about one-third of lean tissue mass depleted in about 60 days
                            \end{itemize}
                    \end{itemize}
            \end{itemize}

        \subsection{What Are the Effects of an Energy Imbalance?}
            \begin{itemize}
                \item Excess calories can lead to weight gain
                    \begin{itemize}
                        \item Excess calories are stored as fat, regardless of source
                            \begin{itemize}
                                \item Limited capacity to store glucose as glycogen
                                \item Can't store extra protein
                                \item Unlimited capacity to store fat
                                    \begin{itemize}
                                        \item Body contains about 35 billion fat cells, which can expand
                                    \end{itemize}
                            \end{itemize}
                    \end{itemize}
            \end{itemize}

        \subsection{What Factors Are Likely to AffBody Weight?}
            \begin{itemize}
                \item Factors in weight management: what and how often you eat, physiology, genetics, environment
                \item Hunger and appetite affect what you eat
                    \begin{itemize}
                        \item Appetite is psychological desire for food
                        \item Hunger is physiological need for food; subsides as feeling of \textbf{satiation} sets in
                            \begin{itemize}
                                \item \textbf{Satiety} determines length of time between eating episodes
                            \end{itemize}
                    \end{itemize}
                \item Physiological mechanisms help regulate hunger
                    \begin{itemize}
                        \item Many hormones play a role:
                            \begin{itemize}
                                \item \textbf{Ghrelin:} produced in stomach when empty; increases hunger
                                \item When fat stores increase, \textbf{leptin} in fat tissue signals brain to decrease hunger and food intake
                                \item \textbf{Cholecystokinin:} released when stomach is distended, increasing feelings of satiation, decreasing hunger
                            \end{itemize}
                        \item Protein, fatty acids, and monosaccharides in small intestine stimulate feedback to brain to decrease hunger
                        \item Insulin also causes brain to decrease hunger
                    \end{itemize}
                \item Many people override feedback mechanisms, resulting in energy imbalance
                \item Genetics partially determines body weight
                    \begin{itemize}
                        \item Risk of becoming obese doubles if parents are overweight, triples if obese, five times greater if severely obese
                        \item Confirmed by studies of identical twins separated at birth
                    \end{itemize}
                \item Genetic differences in level or function of hormones, such as high ghrelin or low leptin levels, increase obesity
                    \begin{itemize}
                        \item Many obese have adequate leptin, but brain has developed resistance to it
                    \end{itemize}
                \item Genetic differences \textbf{in non-exercise-associated thermogenesis (NEAT):} energy expenditure in nonexercise movements, such as fidgeting, standing, chewing gum
                \item "Set point" theory holds that body opposes weight loss and works to maintain a set weight
                \item Environmental factors can increase appetite and decrease physical activity
                \item Environmental of cheap and easily obtainable energy-dense foos stimulates appetite
                    \begin{itemize}
                        \item \textbf{Gene-environment interaction:} increases risk of obesity in some people
                    \end{itemize}
                \item We work more and cook less
                    \begin{itemize}
                        \item 32 percent of calories come from ready-to-eat foods prepared outside of home
                        \item Frequent dining out associated with higher BMI
                    \end{itemize}
                \item We eat more (and more)
                    \begin{itemize}
                        \item Increased availability of food-service establishmentsand access to large variety of foods, larger portions encourage people to eat more
                    \end{itemize}
                \item We sit more and move less
                    \begin{itemize}
                        \item Americans are eating about 600 calories/day more than in 1970
                        \item Labor-saving devices at work and home, sedentary leisure activites ("screen time") result in decreased energy expenditure
                    \end{itemize}
            \end{itemize}

        \subsection{How Can You Lose Weight Healthfully?}
            \begin{itemize}
                \item National Institutes of Health: overweight individuals should aim to lose about 10 percent of body weight over 6-month period
                    \begin{itemize}
                        \item Example: 180-pound person should lose 18 pounds/6 months = 3 pounds/month, 0.75 pounds/week
                        \item To lose 1 pound of body fat, need 3,500-calorie deficit
                            \begin{itemize}
                                \item For a weight loss of 0.5 to 1 pounds/week, need to decrease daily calories by 250 to 500 calories
                            \end{itemize}
                    \end{itemize}
                \item Fad diets promise dramatic results but may carry risks
                \item Successful long-term weight loss requires changes in diet, physical activity, behavior
                \item Eat smart, because calories count: add satiation to low-calorie meals by including higher-volume foods
                    \begin{itemize}
                        \item Eat more vegetables, fruit, and fiber
                        \item Include some protein and fat in your meals
                            \begin{itemize}
                                \item Protein increases satiety most
                                \item Fat slows movement of food from stomach into intestines
                                \item Choose lean meat, skinless chicken, fish, nuts, unsaturated oils
                            \end{itemize}
                    \end{itemize}
                \item Use MyPlate as a weight-loss guide
                    \begin{itemize}
                        \item High volume of fruits, vegetables, whole grains, some lean protein, modest amounts of fat
                        \item Diet should contain variety of foods from all food groups
                            \begin{itemize}
                                \item Replace higher-calorie foods with lower-calorie options from each food group
                                    \begin{itemize}
                                        \item Example: replace full-fat dairy with nonfat products
                                        \item Replaces sodas with water
                                    \end{itemize}
                            \end{itemize}
                    \end{itemize}
                \item Move to lose
                    \begin{itemize}
                        \item 45 minutes/day of moderate-intensity activities can prevent becoming overweight and aid in weight loss
                            \begin{itemize}
                                \item 10,000 steps/dat can reduce risk of becoming overweight
                            \end{itemize}
                    \end{itemize}
                \item Break bad habits
                    \begin{itemize}
                        \item \textbf{Behavior modification:} change behaviors that contribute to weight gain or impede weight loss
                            \begin{itemize}
                                \item Techniques include keeping food log, controlling environments cues that trigger eating, managing stress
                            \end{itemize}
                    \end{itemize}
            \end{itemize}

        \subsection{Examining the Evidence: Evaluating Popular Diets}
            \begin{itemize}
                \item Reduction of calories, not composition of diet, is key to weight loss
                \item People who diligently adhere to diets lose the most weight
                    \begin{itemize}
                        \item High dropout rates for most extreme diets (Atkins and Ornish diets)
                    \end{itemize}
                \item Beware of fad diet claims and hype:
                    \begin{itemize}
                        \item "It's carbs, not calories, that make you fat!"
                        \item "Lose seven pounds in one week!"
                        \item Celebrity-endorsed miracle weight-loss products
                        \item "Natural" substances help lose weight without risk
                    \end{itemize}
            \end{itemize}

        \subsection{Dealing with Extreme Obesity}
            \begin{itemize}
                \item BMI > 40 = extreme obesity
                    \begin{itemize}
                        \item High risk of heart disease, stroke, dying
                        \item Requires aggressive weight-loss treatment, including very-low-calorie diets, medications, and/or surgery
                        \item \textbf{Very-low-calorie diets} (< 800 calories) are short-term and must be medically supervised
                        \item Medications such as Orlistat, Belviq, and Qsymia can't replace a lower-calorie diet, physical activity, and behavior modification
                            \begin{itemize}
                                \item However, they impact appetite and help individuals lose from 3-9 percent of their body weight wehn combined with diet and exercise
                            \end{itemize}
                    \end{itemize}
                \item Gastric bypass and gastric banding result in higher levels of satiety and lower levels of hunger
                    \begin{itemize}
                        \item Results in dramatic weight loss and reduction of hypertension, diabetes, high blood cholesterol, and sleep apnea
                        \item Small risk of gallstones, death from surgery
                    \end{itemize}
                \item Liposuction is performed for cosmetic reasons
                    \begin{itemize}
                        \item Fat may reappear; results are not permanent
                        \item Complications such as infections, scars, swelling
                    \end{itemize}
            \end{itemize}

        \subsection{How Can You Maintain Weight Loss?}
            \begin{itemize}
                \item \textbf{Weight cycling} (repeated gain and loss of body weight) is common result of fad diets
                \item Weight loss can be maintained if healthy habits used during weight loss are maintained
                    \begin{itemize}
                        \item National Weight Control Registry
                    \end{itemize}
                \item New, lower weight requires fewer calories to maintain weight
                    \begin{itemize}
                        \item Physical activity can close the \textbf{"energy gap,"} which is easier than further reducing caloric intake
                            \begin{itemize}
                                \item estimated that the energy gap is about 8 calories per pound of lost weight
                            \end{itemize}
                    \end{itemize}
                \item Gaining weight for the underweight is as challenging as losing weight is for the overweight
                \item Need to add at least 500 calories to daily energy intake for gain 1 pound/week
                    \begin{itemize}
                        \item Choose more energy-dense, but nutritious foods from each food group
                            \begin{itemize}
                                \item Examples: waffle instead of toast, coleslaw instead of cabbage
                            \end{itemize}
                        \item Eat more snacks during day to add more calories
                    \end{itemize}
            \end{itemize}

        \subsection{What Is Disordered Eating and What Are the Warning Signs?}
            \begin{itemize}
                \item \textbf{Disordered eating:} abnormal and potentially harmful eating behaviors that do not meet specific criteria for eating disorders
                \item \textbf{Eating disorders:} psychological illnesses that involve specific abnormal eating behaviors and other factors
                    \begin{itemize}
                        \item In United States, about 20 million women and 10 million men struggle with eating disorders at some point in life
                            \begin{itemize}
                                \item Most are adolescent or young adult white, middle- or upper-middle-class females, but increasing among males, minorities, other age-groups
                            \end{itemize}
                    \end{itemize}
                \item No single factor causes eating disorders
                \item Sociocultural factors
                    \begin{itemize}
                        \item Dseire/social pressure to be thin or "cut"
                    \end{itemize}
                \item Genetic factors
                    \begin{itemize}
                        \item Eating disorders "run in families"
                    \end{itemize}
                \item Psychological factors
                    \begin{itemize}
                        \item Depression, anxiety, perfectionism, sense of control contribute
                    \end{itemize}
                \item \textbf{Anorexia nervosa} results from severe calorie restriction
                    \begin{itemize}
                        \item Self-starvation and excessive weight loss
                        \item Intense fear of being "fat"
                        \item Distorted body image: see oneself as fat when underweight
                        \item Health consequences: electrolyte imbalance (low blood potassium) can be fatal
                        \item Other risks: decrease in heart rate and blood pressure, \textbf{lanugo} (downy hair), osteoporosis
                    \end{itemize}
                \item \textbf{Bulimia nervosa} involves cycle of binge eating and purging
                    \begin{itemize}
                        \item \textbf{Purging} can include self-induced vomiting; excessive exercising; strict dieting or fasting; abuse of diet pills, laxatives, diuretics
                            \begin{itemize}
                                \item Vomiting can cause tears in esophagus, swollen parotid glands, tooth decay, gum disease, broken blood vessels in eyes
                            \end{itemize}
                        \item Potentially fatal electrolyte imabalance can result
                    \end{itemize}
                \item \textbf{Binge eating disorder} involves compulsive overeating (without purging)
                    \begin{itemize}
                        \item Eat in secret, feelings of shame
                        \item Health effects are those associated with obesity
                            \begin{itemize}
                                \item High blood pressure, cholesterol levles
                                \item Risk of heart disease, type 2 diabtes, gallbladder disease
                            \end{itemize}
                    \end{itemize}
                \item Other disordered eating behaviors can be harmful
                \item \textbf{Avoidant/Restrictive Food Intake Distorder (ARFID)}
                \item \textbf{Pica:} desire to consume nonnutritive substances (clay, dirt, chalk)
                    \begin{itemize}
                        \item Can cause medical complications
                    \end{itemize}
                \item \textbf{Other Specified Feeding or Eating Disorder (OSFED)}
                    \begin{itemize}
                        \item Purging disorder
                        \item Atypical Anorexia Nervosa
                        \item \textbf{Orthorexia:} "healthy or righteous eating"
                            \begin{itemize}
                                \item Fixation on eating the "right" foods
                            \end{itemize}
                        \item \textbf{Night eating syndrome:} combination eating, sleep, mood disorder
                            \begin{itemize}
                                \item Person consumes most calories after evening meal, wake up at night to eat
                            \end{itemize}
                    \end{itemize}
                \item There are some common signs of disordered eating
                    \begin{itemize}
                        \item Hair loss
                        \item Significant/sudden weight changes
                        \item \textbf{Russell's sign:} scar tissue on knuckles of fingers used to induce vomiting (bulimia nervosa)
                        \item Avoiding social situations where food is present
                        \item Weighing often, obsessibely counting calories
                        \item Denial of problem
                    \end{itemize}
                \item Eating disorders can be treated
                \item Multidisciplinary team approach is most effective
                    \begin{itemize}
                        \item Psychological, medical, and nutrition professionals
                        \item Nutritional approaches include:
                            \begin{itemize}
                                \item Identifying binge triggers, safe and unsafe foods, hunger and fullness cues using food journals
                                \item Meal plans to ensure adequate calorie/nutrient intake (anorexia nervosa) or to avoid overeating (bulimia nervosa, binge eating disorder)
                            \end{itemize}
                        \item Best treated in early stages; no "quick fix"
                    \end{itemize}
            \end{itemize}

        \subsection{A Closer Look at Body Image}
            \begin{itemize}
                \item Body image is the way you perceive and what you believe about your physical appearance
                \item \textbf{Body dysmorphic disorder:} is a mental illness in which a person's preoccupation with minor or imaginary physical flaws cause signifcant distress
                \item Strategies to help maintain a positive body image:
                    \begin{itemize}
                        \item Know and accept what determines your physical appearance (genetics, age, etc.)
                        \item Avoid dieting
                        \item Avoid comparing yourself to others
                        \item Recognize that you are a whole person and not just individual aprts
                        \item Respect yourself and others based on the qualities of character and accomplishments, rather than appearance
                    \end{itemize}
            \end{itemize}

    \section{Chapter 11: Nutrition and Fitness}

        \subsection{What Is Physical Fitness and Why Is It Important?}
            \begin{itemize}
                \item \textbf{Physical fitness:} good health or physical condition, primarily the result of \textbf{exercise} and proper nutrition
                \item Physical fitness has five basic components:
                    \begin{itemize}
                        \item \textbf{Cardiorespiratory endurance:} ability to sustain cardiorespiratory exercise for extended time
                            \begin{itemize}
                                \item Examples: running, biking
                                \item Cardiovascular and respiratory systems must provide enough oxygen and energy to muscles
                            \end{itemize}
                        \item \textbf{Muscle strength:} ability to produce force for brief time
                    \end{itemize}
                \item \textbf{Muscle endurance:} ability to exert force for a long period of time without fatigue
                    \begin{itemize}
                        \item Muscle strength and endurance best achieved with strength training
                    \end{itemize}
                \item \textbf{Flexibility:} range of motion around a joint
                    \begin{itemize}
                        \item Improved with stretching
                    \end{itemize}
                \item \textbf{Body composition:} proportion of muscle, fat, water, and other body tissues that make up body weight
                \item Physical fitness provides numerous benefits
                    \begin{itemize}
                        \item Helps achieve and maintain healthy body weight
                        \item Reduces risk of cardiovascular disease, type 2 diabetes, and some types of cancer
                        \item Improves body composition, bone health, and immune system
                        \item Improves overall health, such as more restful sleep and stress reduction
                    \end{itemize}
                \item Over half of adults in United States do not meet regular physical activity recommendations
            \end{itemize}

        \subsection{What Does a Physical Fitness Program Look Like?}
            \begin{itemize}
                \item Cardiorespiratory exercise can improve cardiorespiratory endurance and body composition
                    \begin{itemize}
                        \item Continuous activities that use large muscle groups
                            \begin{itemize}
                                \item Examples: high-impact aerobics, stair climbing, brisk walking
                                \item Primarily aerobic because it uses oxygen
                                    \begin{itemize}
                                        \item Heart rate and stroke volume increased to maximize blood flow delivery to muscles
                                    \end{itemize}
                            \end{itemize}
                    \end{itemize}
                \item Strength training can improve muscle strength, muscle endurance, and body composition
                    \begin{itemize}
                        \item To increase muscle strength: low number of repititions using heavy weights
                        \item To increase muscle endurance: high number of repititions using lighter weights
                        \item Important to rest between sets of an exercise and between workouts to prevent muscle strains and injury
                    \end{itemize}
                \item Stretching can improve flexibility
                \item The \textbf{FITT Principle} can help you design a fitness program: Frequency, Intensity, Time, Type
                    \begin{itemize}
                        \item \textbf{Rate of perceived exertion} (RPE) is a self assessment that measures intensity of cardiorespiratory exercise
                        \item \textbf{Target heart rate} shows exercise intensity through heart rate (percentage of maximum)
                        \item \textbf{Repetition maximum} (RM) refers to intensity of strength training
                    \end{itemize}
                \item \textbf{Physical Activity Guidelines:} 60 miniutes/week of moderate-intensity activity for some health benefits; 150 minutes/week for substantial benefits and reduced risk of chronic disease
                    \begin{itemize}
                        \item 60 to 90 minutes daily to lose weight effectively
                    \end{itemize}
                \item The \textbf{progressive overload principle} can help improve fitness over time
                    \begin{itemize}
                        \item The body adapts to physical activities, producing fitness plateau
                        \item Modify one or more FITT principles to increase exercise and improve fitness
                    \end{itemize}
            \end{itemize}

        \subsection{How Are Carbohydrate, Fat, and Protein Used during Exercise?}
            \begin{itemize}
                \item Energy during first few minutes of physical activity is provided by \textbf{anaerobic energy production} (without oxygen) from breakdown of:
                    \begin{itemize}
                        \item \textbf{Adenosine triphosphate (ATP)}
                        \item \textbf{Creatine phosphate}
                            \begin{itemize}
                                \item Limited amount stored in cells
                            \end{itemize}
                    \end{itemize}
                \item As exercise continues, oxygen intake and aerobic energy production increase
                    \begin{itemize}
                        \item Carbohydrate (glucose) and fatty acids broken down to yield ATP energy via aerobic metabolism
                    \end{itemize}
                \item Carbohydrate is the primary energy source during high-intensity exercise
                    \begin{itemize}
                        \item Carbohydrate from blood glucose and stored glycogen in muscle and liver: about 2 hours of exercise
                        \item Well-trained muscles store 20 to 50 percent more glycogen than untrained muscles
                        \item Liver glycogen maintains normal blood glucose
                        \item Lactic acid is produced at high exercise intensities and shuttled to other tissues
                            \begin{itemize}
                                \item Used for energy during low-intensity exercise
                            \end{itemize}
                    \end{itemize}
                \item Intensity affecfts how much glucose and glycogen you use
                    \begin{itemize}
                        \item Glucose and glycogen use increases as intensity increases
                    \end{itemize}
                \item How much carbohydrate do you need for exercise?
                    \begin{itemize}
                        \item Depends on duration of activity
                            \begin{itemize}
                                \item During and/or after activity: bananas, bagels, corn flakes that are absorbed quickly
                                \item 2 hours before exercise: rice, oatmeal, pasta, corn enter blood more slowly for sustained energy
                            \end{itemize}
                    \end{itemize}
                \item Fat is the primary energy source during low-intensity exercise
                    \begin{itemize}
                        \item Two forms: fatty acids (from triglycerides) in adipose tissue and in muscle tissue
                        \item Converting fatty acids into energy is slow and requires more oxygen compared with carbohydrate
                    \end{itemize}
                \item Intensity and training affect how much fat you use
                    \begin{itemize}
                        \item Low-intensity exercise uses mostly fat from adipose tissue
                        \item Moderate-intensity exercise also uses fatty acids from muscle triglycerides
                        \item Well-trained muscles burn more fat than less trained muscles
                            \begin{itemize}
                                \item Body uses less glycogen and more fat, increases endurance
                            \end{itemize}
                    \end{itemize}
                \item How much fat do you need for exercise?
                    \begin{itemize}
                        \item 25 to 30 percent of calories should come from fat
                            \begin{itemize}
                                \item Consume unsaturated fats and limit saturated fat to <= 10 percent of total calories
                                \item Too little fat (<20 percent) has nutritional risks
                            \end{itemize}
                    \end{itemize}
                \item \textbf{Fat-burning zone:} 65 to 73 percent of maximum heart rate
                \item \textbf{"Cardio" zone:} >73 percent of maximum heart rate
                \item Not necessary to stay in fat-burning zone to lose weight
                    \begin{itemize}
                        \item Need to burn calories to produce overall calorie deficit
                        \item High-intensity exercise burns calories more quickly but lower-intensity workout can last longer and achieve more
                    \end{itemize}
                \item Protein is primarily needed to build and repair muscle
                    \begin{itemize}
                        \item Muscle damage results from exercise, especially in weight or strength training
                        \item Amino acids needed to promote muscle growth and recovery
                    \end{itemize}
                \item Body can use protein for energy but prefers carbohydrate and fat as main energy sources
                    \begin{itemize}
                        \item Amino acids are converted to glucose in liver
                    \end{itemize}
                \item Endurance athletes need 1.2 to 1.4 g of protein per kg of body weight
                \item Resistance/strength activities: 1.2 to 1.7 g/kg body weight
                \item Total calorie needs depend on the type and schedule of exercise
                \item Timing of meals affects fitness performance
                    \begin{itemize}
                        \item Optimal food choices vary before, during, and after exercise
                    \end{itemize}
            \end{itemize}

        \subsection{How Does the Timing of Meals Affect Fitness and Athletic Performance?}
            \begin{itemize}
                \item Optimal foods before exercise
                    \begin{itemize}
                        \item Allow adequate time for digestion
                            \begin{itemize}
                                \item Large meal: 3 to 4 hours
                                \item Smaller meals: 2 to 3 hours
                                \item Snack or liquid supplement: 0.5 to 1 hour
                            \end{itemize}
                    \end{itemize}
                \item Pre-exercise meal: 1 to 4.5 g carbohydrate/kg body weight, 1 to 4 hours before exercise
                    \begin{itemize}
                        \item Carbohydrate 15 to 30 minutes before gives muscles immediate energy, spares glycogen stores, helps reduce muscle damage
                        \item Consuming protein before exercise as well as during exercise increases muscle glycogen synthesis and protein synthesis after exercise is over
                        \item High-fat foods should be avoided before exercise: take longer to digest, may cause stomach discomfort and sluggishness
                    \end{itemize}
                \item Optimal foods during exercise
                    \begin{itemize}
                        \item For exercise > 1 hours, begin carbohydrate intake shortly after start and every 15 to 20 minutes
                            \begin{itemize}
                                \item 30 to 60 g carbohydrate/hour to avoid fatigue
                            \end{itemize}
                        \item Glucose, sucrose, maltodextrin are best choices for quick absorption
                            \begin{itemize}
                                \item Avoid fructose, which can cause GI problems
                            \end{itemize}
                        \item Consuming both carbohydrate and protein is best for muscle maintenance and growth
                    \end{itemize}
                \item Optimal foods after exercise
                    \begin{itemize}
                        \item The best post-exercise meal is consumed quickly and contains both carbohydrate and protein:
                            \begin{itemize}
                                \item Carbohydrate/protein ratio of 3:1 is ideal to promote muscle glycogen and protein synthesis and faster recovery time
                            \end{itemize}
                        \item Preferred protein choice: whey protein (in milk) is absorbed rapidly and contains all essential amino acids needed
                    \end{itemize}
                \item When consuming small snack or liquid supplement after exercise, should have high-carbohydrate, moderate-protein, low-fat meal within 2 hours
            \end{itemize}

        \subsection{What Vitamins and Minerals Are Important for Fitness}
            \begin{itemize}
                \item Vitamins and minerals play major role in metabolism of carbohydrate, fat, and protein for energy during exercise
                \item Some also act as antioxidants and help protect cells from the oxidative stress that can occur with exercise
                \item Antioxidants and cellular damage caused by exercise
                    \begin{itemize}
                        \item Using more oxygen during exercise increases free radicals that damage cells
                            \begin{itemize}
                                \item Supplements of antioxidant vitamins E and C not shown to improve athletic performance or decrease oxidative stress in highly trained athletes
                                \item Consume adequate amounts (RDA) from nuts, vegetable oils, broccoli, citrus fruits
                            \end{itemize}
                    \end{itemize}
                \item Some minerals can be of concern in highly active people
                \item \textbf{Iron:} Low iron levels can reduce hemoglobin and blood's ability to transport oxygen to cells, causing early fatigue during exercise
                    \begin{itemize}
                        \item Female athletes more at risk for iron-deficiency anemia
                            \begin{itemize}
                                \item Also long-distance runners, those in "make weight" sports and other sports
                                \item Iron-rich foods and iron supplements may be needed
                            \end{itemize}
                        \item "Sports anemia": Decreased hemoglobin can result from strenuous training due to increased blood volume
                            \begin{itemize}
                                \item Not same as iron-deficiency anemia and is self-correcting
                            \end{itemize}
                    \end{itemize}
                \item \textbf{Calcium:} important to reduce risk of bone fractures
                    \begin{itemize}
                        \item Calcium is lost in sweat
                        \item Exercise can increase bon mineral content and may be able to compensate for calcium lost in sweat
                        \item \textbf{Magnesium:} Higher needs with physical activity
                        \item Supplements not recommended unless food intake is inadequate
                    \end{itemize}
                \item Vitamin and mineral supplements are generally not necessary
                    \begin{itemize}
                        \item Everyone, not just athletes, should obtain vitamins and minerals through nutrient-dense foods before considering the use of supplememnts
                    \end{itemize}
            \end{itemize}
        
        \subsection{How Does Fluid Intake Affect Fitness?}
            \begin{itemize}
                \item Fluid and electrolyte balance and body temperature are affected by exercise
                    \begin{itemize}
                        \item Water is lost through sweat and exhalation
                        \item Sodium and chloride, and to a lesser extent potassium, are electrolytes lost in sweat
                            \begin{itemize}
                                \item Electrolyte imbalance can cause heat cramps, nausea, lowered blood pressure, edema
                            \end{itemize}
                        \item Evaporation of sweat helps cool the body
                            \begin{itemize}
                                \item Hot, humid weather reduces evaporation and body heat increases: increases risk of heat exhaustion and heat stroke
                            \end{itemize}
                    \end{itemize}
                \item You need fluids before, during, and after exercise
                    \begin{itemize}
                        \item The American College of Sports Medicine has specific recommendations for how much fluid to drink before and during exercise
                    \end{itemize}
            \end{itemize}

        \subsection{Some Beverages Are Better Than Others}
            \begin{itemize}
                \item Sports drinks contain 6 to 8 percent carbohydrate and sodium and potassium: beneficial in long endurance events
                    \begin{itemize}
                        \item For events less than 60 minutes, water is sufficient to replace fluids, and post-exercise food will replace electrolytes
                        \item Sports drinks should be avoided as a daily beverage: damage tooth enamel, provide unwanted calories
                    \end{itemize}
                \item Not recommended during physical activity: fruit juice (too high carbohydrate concentration); carbonated drinks (bloating); alcohol and caffeine (diuretics, unwanted side effects)
            \end{itemize}

        \subsection{Consuming Too Little or Too Much Fluid Can Be Harmful}
            \begin{itemize}
                \item Thirst is not a good indicator of fluid needs for athletes
                    \begin{itemize}
                        \item \textbf{Acute dehydration:} when not adequately hydrated over a short period of time
                        \item \textbf{Chronic dehydration:} when not adequately hydrated over extended period of time
                            \begin{itemize}
                                \item Fatigue, muscle soreness, poor recovery from workout, headaches, nausea, dark urine
                            \end{itemize}
                    \end{itemize}
                \item \textbf{Hyponatremia:} low sodium blood levels due to consuming too much water without electrolytes
            \end{itemize}

        \subsection{Can Dietary Supplements Contribute to Fitness?}
            \begin{itemize}
                \item Dietary supplements are not strictly regulated by FDA
                    \begin{itemize}
                        \item Manufacturers not required to prove safety or efficacy of supplement claims
                    \end{itemize}
                \item Dietary supplements and ergogenic aids may improve performance, but can have side effects
                    \begin{itemize}
                        \item \textbf{Creatine:} research data mixed on enhancement of performance
                            \begin{itemize}
                                \item Improves high-intensity, short-duration activities (like weight training) that rely on anaerobic metabolism
                            \end{itemize}
                    \end{itemize}
                \item \textbf{Caffeine} enhances athletic performance, mostly during endurance events.
                    \begin{itemize}
                        \item Stimulates central nervous system, breakdown of muscle glycogen, may increase fatty acid availability
                        \item Considered a banned substance by some athletic associations
                    \end{itemize}
                \item \textbf{Anabolic steroids:} testosterone-based substances that promote muscle growth and strength (anabolic effect)
                    \begin{itemize}
                        \item Androgenic effect (testosterone-promoting): hormone imbalance causes undesirable side effects in both men and women; also health risks
                    \end{itemize}
                \item \textbf{Growth hormone:} little research on effects on athletic performance, results mixed
                    \begin{itemize}
                        \item Increases muscle mass and reduces body fat but does not increase muscle strength
                        \item Excess can cause acromegaly and serious health issues
                    \end{itemize}
                \item \textbf{Erythropoietin and blood doping:} to increase oxygen-carrying capacity of the blood
                    \begin{itemize}
                        \item Can increase blood viscosity, increase risk of stroke and heart attack
                    \end{itemize}
                \item Sports bars, shakes, and meal replacers may provide benefits
                    \begin{itemize}
                        \item The main energy source in most sports bars and shakes is carbohydrate, with protein and fat contributing smaller amounts of energy
                        \item Convenient alternative, but more expensive than whole foods
                            \begin{itemize}
                                \item Often include vitamins and minerals, which may be unneeded
                            \end{itemize}
                    \end{itemize}
            \end{itemize}

        
    \section{Chapter 12: Consumerism and Sustainability: From Farm to Table}
        
        \subsection{How Do Advertising and Marketing Influence Your Food Choices?}
            \begin{itemize}
                \item As \textbf{food consumers}, we have influence over \textbf{food industry}, but advertising and marketing control many of our choices
                \item Food companies spend close to 10 billion dollars annually to promote their products
                    \begin{itemize}
                        \item Much promotion for nutritionally dubious products
                        \item Advertising for fruits, vegetables almost nonexistent
                            \begin{itemize}
                                \item Yale Rudd Center for Food Policy and Obesity Fast Food Facts in 2012
                                    \begin{itemize}
                                        \item McDonald's spent 3x as much advertising on products as did others advertising fruit, vegetables, bottled water, or milk
                                    \end{itemize}
                            \end{itemize}
                    \end{itemize}
                \item College-aged and young adults are increasingly targets of advertisers
            \end{itemize}

        \subsection{Where Does Your Food Come From?}
            \begin{itemize}
                \item Much of your food comes from small, family-run American farms
                    \begin{itemize}
                        \item To be a farm in the United States, must produce and sell at least 1,000 dollars of agricultural products/year
                        \item Just over 2 million farms, most in Midwest, Great Plains, California
                            \begin{itemize}
                                \item 80,000 fewer farms in 2012 than in 2007, continues to decline
                            \end{itemize}
                        \item Fewer than 960,000 Americans farmers (1 percent of population) produce food for population of 300+ million
                    \end{itemize}
                \item Challenges of farming
                    \begin{itemize}
                        \item High costs
                        \item Demand for low food prices
                        \item Competition
                        \item Dependence on nature's cooperation
                    \end{itemize}
                \item Technology, government support aid farmer
                    \begin{itemize}
                        \item Computers, Internet allow for \textbf{precision agriculture}
                        \item Government subsidies for commodity crops (e.g. corn, soybean, and wheat)
                    \end{itemize}
                \item The role of agribusiness
                    \begin{itemize}
                        \item \textbf{Agribusiness:} blending of agricultural and business entities that affect how food, clothes, home goods are developed, processed, distributed, and purchased
                            \begin{itemize}
                                \item Food portion includes food production, agricultural chemicals, finance and trade, management, environmental considerations, land development
                            \end{itemize}
                        \item Agriculture sector employs about 11 percent of U.S. population ~21 million Americans
                            \begin{itemize}
                                \item Food processing companies comprise large share
                            \end{itemize}
                    \end{itemize}
                \item Crops grown for food
                    \begin{itemize}
                        \item Top three food crops in United States: corn, soybeans, wheat
                            \begin{itemize}
                                \item World's largest corn producer: 10 billion bushels from > 400,000 farms in \textbf{Corn Belt}
                                    \begin{itemize}
                                        \item Most of the corn in the U.S. ends up as feed for livestock, poultry, and fish
                                    \end{itemize}
                                \item 50 percent of world's soybeans from > 290,000 U.S. farms
                                    \begin{itemize}
                                        \item 70 percent of soybeans used to feed livestock
                                    \end{itemize}
                                \item 13 percent of world's wheat from > 160,000 farms in Great Plains
                                    \begin{itemize}
                                        \item 70 percent used for food, 22 percent used for animal feed, rest used to replenish crops
                                    \end{itemize}
                            \end{itemize}
                        \item Most staple crops used for animal feed, not humans
                    \end{itemize}
                \item Animals raised for food on \textbf{feedlots}
                    \begin{itemize}
                        \item Dominant food animals in United States: cows, pigs, chickens
                    \end{itemize}
                \item Exporting foods: the good and bad news
                    \begin{itemize}
                        \item U.S. farmers help feed world
                            \begin{itemize}
                                \item Estimated 30 percent of farm income from foreign trade
                            \end{itemize}
                        \item Also exporting unhealthy eating habits
                            \begin{itemize}
                                \item Shift to high-calorie, high-fat, processed food diet
                                \item \textbf{Globesity} (growing incidence of obesity worldwide) becoming a global threat
                            \end{itemize}
                    \end{itemize}
                \item Food production outside the United States
                    \begin{itemize}
                        \item U.S. exports more agricultural products than it imports
                            \begin{itemize}
                                \item Most coffee, cocoa, and spices in the U.S. is imported from other countries
                                \item Most coffee comes from Colombia and Brazil
                            \end{itemize}
                    \end{itemize}
                \item Importing foods: the good and bad news
                    \begin{itemize}
                        \item Two primary reasons for U.S. food imports
                            \begin{itemize}
                                \item Demand for variety of products year round
                                \item Demand for cheap food
                            \end{itemize}
                        \item Problems:
                            \begin{itemize}
                                \item Environmental costs of long-distance shipping
                                \item Potential for food contamination overseas
                                    \begin{itemize}
                                        \item Because of the tremendous volume of imports, the FDA only inspects less than 2 percent of all food products brought into the United States
                                    \end{itemize}
                            \end{itemize}
                    \end{itemize}
            \end{itemize}

        \subsection{What Is a Sustainable Food System?}
            \begin{itemize}
                \item A \textbf{sustainable food system} is one that addresses concerns regarding the health of individuals, the community, and the environment in a way that intends to provide healthy food for the world's population for generations to come
                \item A sustainable food system must:
                    \begin{itemize}
                        \item Be environmentally friendly
                        \item Economically viable
                        \item Socially equitable
                    \end{itemize}
                \item Many food systems degrade environment, reduce biodiversity, pollute air and water
                \item Concerns about:
                    \begin{itemize}
                        \item Soil use: improper use degrades topsoil, endangers food soil web
                        \item Energy use: fossil fuels harm environment
                            \begin{itemize}
                                \item Using alternatives aids sustainability
                            \end{itemize}
                        \item Water use: growing consumption
                            \begin{itemize}
                                \item Convservation is necessary
                            \end{itemize}
                        \item Reducing food waste is part of sustainability
                            \begin{itemize}
                                \item 30-40 percent of all food produced is wasted
                            \end{itemize}
                    \end{itemize}
                \item Being a more sustainable food consumer
                    \begin{itemize}
                        \item Adopting "greener" habits can help
                            \begin{itemize}
                                \item Examples: eating less meat
                            \end{itemize}
                    \end{itemize}
                \item The most sustainable foods are locally grown and plant-based
                    \begin{itemize}
                        \item Plant based diets are also more health promoting
                    \end{itemize}
                \item Small farms often provide foods to people living in their communities through:
                    \begin{itemize}
                        \item Community-supported agriculture (CSA)
                        \item Farmers' markets
                        \item Contracts through local grocery stores
                    \end{itemize}
                \item Buying food from local farms doesn't guarantee that foods were grown in a sustainable way, nor does being from a distant farm mean that those farmers didn't practice sustainable agriculture.
            \end{itemize}

        \subsection{Nutrition in the Real World: You as a Sustainable Farmer: Growing Vegetables in a Container}
            \begin{itemize}
                \item Almsot anyone can be a home gardener
                    \begin{itemize}
                        \item Requirements:
                            \begin{itemize}
                                \item Container: ceramic pot, planter box, or other
                                \item Potting mixture: soil mix
                                \item Plant: various vegetables
                                \item Fertilizing: enrich soil with powdered fertilizer
                                \item Watering: avoid under- or overwatering
                                \item Harvesting: timing depends on plant type
                            \end{itemize}
                    \end{itemize}
            \end{itemize}
        
        \subsection{How Do We Balance the World Population's Demand for Food with Sustainability?}
            \begin{itemize}
                \item Costs and Benefits of Using Hormones
                    \begin{itemize}
                        \item Chemical compounds improve farm yields but also cause concern
                        \item Hormones
                            \begin{itemize}
                                \item In cows, bovine \textbf{growth hormone} and its synthetic version, \textbf{recombinant bovine somatotropin (rbST)}, stimulate milk production
                                \item FDA has found no negative effects, but some consumer groups question safety
                            \end{itemize}
                    \end{itemize}
                \item Costs and Benefits of Using Antibiotics
                    \begin{itemize}
                        \item Whether injected or given via feed, \textbf{antibiotics} are used for three purposes:
                            \begin{itemize}
                                \item To treat animals that are sick
                                \item To preventatively treat animals that may be at risk of being sick
                                \item To promote growth
                            \end{itemize}
                        \item Risks include growth of antibiotic-resistant bacteria, posing threat to humans when consumed
                            \begin{itemize}
                                \item Government agencies try to prevent overuse
                            \end{itemize}
                    \end{itemize}
                \item Costs and Benefits of Using Pesticides
                    \begin{itemize}
                        \item Control pests that threaten food supply
                    \end{itemize}
                \item Types of \textbf{pesticides}
                    \begin{itemize}
                        \item \textbf{Herbicides:} kill weeds
                        \item \textbf{Antimicrobials:} kill microorganisms (bacteria, viruses)
                        \item \textbf{Fungicides:} kill fungi (mold)
                        \item \textbf{Biopesticides:} derived from natural materials; include sex pheromones
                        \item \textbf{Organophosphates:} affect nervous system of pests
                    \end{itemize}
                \item Risks and regulation of pesticides
                    \begin{itemize}
                        \item When not used responsibly, can cause harm to animals, environment, humans
                        \item Use is heavily regulated in United States
                        \item \textbf{Risk assessment} (by EPA) is process to determine potential human health risks posed by exposure
                    \end{itemize}
                \item Alternatives to pesticides
                    \begin{itemize}
                        \item \textbf{Integrated pest management} uses methods to control pests but limit harmful impact on humans, environment
                            \begin{itemize}
                                \item Examples: crop rotation, pest-resistant crops, biopesticides, natural predators
                            \end{itemize}
                    \end{itemize}
            \end{itemize}

        \subsection{What Are the Risks and Benefits of Using Biotechnology in Agriculture}
            \begin{itemize}
                \item Humans have been manipulating genes of food products for generations
                    \begin{itemize}
                        \item \textbf{Biotechnology:} the application of biological techniques to living cells, which alters their genetic makeup
                            \begin{itemize}
                                \item \textbf{Gene editing}
                            \end{itemize}
                        \item \textbf{Plant breeding:} a type of biotechnology in which two plants are crossbred to produce offspring with desired traits from both
                    \end{itemize}
                \item Genetic engineering
                    \begin{itemize}
                        \item \textbf{Genetic engineering (GE):} the biological technique that isolates and manipulates the genes of organisms to produce a targeted, modified product
                        \item \textbf{Genetically modified organisms (GMOs):} organisms genetically engineered to contain both original and foreign genes
                        \item First GMO crops grown in early 1990s, designed to reduce pesticide, herbicide use
                            \begin{itemize}
                                \item Later versions added nutrients, improved shelf life
                            \end{itemize}
                        \item Proponents believe GMOs are good for environment and food supply
                    \end{itemize}
                \item Concerns and regulations associated with GE foods
                    \begin{itemize}
                        \item Opponents fear creation of "frankenfoods," but industry is tightly regulated by FDA, USDA, EPA
                        \item Many unanswered questions, including:
                            \begin{itemize}
                                \item Effects on natural environment, ecological balance
                                \item Production of plant toxins
                                \item Introduction of new allergens into food
                                \item Changes in nutrient content
                                \item Unsafe animal feed
                            \end{itemize}
                    \end{itemize}
            \end{itemize}

        \subsection{How Does Food Policy Affect the Foods Available to You to Buy and Consume?}
            \begin{itemize}
                \item Various government agencies regulate the food industry and set food and nutrition policy
                \item Food policy can help encourage food producers to create healthier products
                    \begin{itemize}
                        \item Example: \textbf{Dietary Guidelines for Americans} caused shift toward whole grains, improved diet
                    \end{itemize}
                \item Food policy can lead to relabeling and reformulating without providing a healthier food product
                    \begin{itemize}
                        \item Example: Food producers replaced \textbf{trans} fat with saturated fat, with no net positive effect
                    \end{itemize}
                \item What are the politics of the food industry?
                    \begin{itemize}
                        \item Government programs are food consumers
                            \begin{itemize}
                                \item Federal government is nation's biggest food consumer
                                \item Examples: National School Lunch Program, Summer Food Service Program, Emergency Food Assistance Program, Child and Adult Care Food Program
                            \end{itemize}
                        \item Food lobbyists exert influence
                            \begin{itemize}
                                \item Example: 2009 push to tax sugared beverages was blocked in Congress
                            \end{itemize}
                    \end{itemize}
            \end{itemize}

        \subsection{How Do You Know How Foods Were Produced?}
            \begin{itemize}
                \item Label terms provide information about how foods were produced
                    \begin{itemize}
                        \item USDA defines labeling for animal food products
                            \begin{itemize}
                                \item Prepackaged meat products:
                                    \begin{itemize}
                                        \item Certified
                                        \item Fresh poultry
                                        \item Free range
                                        \item Kosher
                                        \item Natural
                                        \item No hormones
                                        \item No antibiotics
                                    \end{itemize}
                            \end{itemize}
                    \end{itemize}
                \item Understand the meaning of the term \textbf{organic}
                    \begin{itemize}
                        \item Market for organic foods has grown rapidly
                        \item USDA developed National Organic Standards (NOS)
                        \item Organic farming means grown without some synthetic pesticides and fertilizers, bioengineering, irradiation
                            \begin{itemize}
                                \item Some pesticides may be used
                            \end{itemize}
                        \item No evidence that organic foods are nutritionally superior
                            \begin{itemize}
                                \item Advantages: Fewer synthetic pesticides and antibiotics; may have environmental benefits
                                \item Disadvantages: Often more expensive than conventionally grown foods
                            \end{itemize}
                    \end{itemize}
            \end{itemize}

    \section{Chapter 13: Food Safety and Technology}

        \subsection{What Causes Foodborne Illness?}
            \begin{itemize}
                \item United States enjoys one of safest food supplies in world
                    \begin{itemize}
                        \item Millions still suffer annually from foodborne illness
                            \begin{itemize}
                                \item Cost 36 billion dollars per year
                            \end{itemize}
                        \item Food safety practices and guidelines established to ensure safety of foods
                        \item Several government agencies work together to ensure safety of foods from farm to table
                    \end{itemize}
                \item Foodborne illnesses are often caused by \textbf{pathogens} (viruses, bacteria, parasites, fungal agents or prions)
                    \begin{itemize}
                        \item Can be spread by \textbf{fecal-to-oral transmission}
                        \item \textbf{Viruses} require living \textbf{host} to survive
                            \begin{itemize}
                                \item Examples: \textbf{Norovirus (gastroenteritis)}, hepatitis A
                            \end{itemize}
                    \end{itemize}
                \item \textbf{Bacteria} flourish on living and nonliving surfaces
                    \begin{itemize}
                        \item Some are beneficial: make vitamin K and biotin in inestines, used to make yogurt and cheese
                        \item Others can cause food spoilage and illness
                            \begin{itemize}
                                \item Most common: \textbf{Campylobacter, E. coli, Salmonella}
                                    \begin{itemize}
                                        \item Can lead to major complications such as Guillain-Barre syndrome, hemolytic uremic syndrome, and traveler's diarrhea
                                    \end{itemize}
                            \end{itemize}
                    \end{itemize}
                \item \textbf{Parasites:} microscopic organisms that takes nourishment from hosts
                    \begin{itemize}
                        \item Found in food and water, often transmitted by fecal-oral route
                    \end{itemize}
                \item \textbf{Prions} are an extremely rare but deadly infectious agent
                    \begin{itemize}
                        \item \textbf{Bovine spongiform ecephalopathy (BSE):} mad cow disease is caused by prions
                    \end{itemize}
                \item Chemical agents and \textbf{toxins} can also cause illness
                    \begin{itemize}
                        \item Naturally occurring toxins include poisonous mushrooms and some fish
                    \end{itemize}
                \item Some people are at higher risk for foodborne illness
                    \begin{itemize}
                        \item Older adults, young children, and those with compromised immune systems are more susceptible to ill effects
                    \end{itemize}
            \end{itemize}

        \subsection{Health Connection: Getting the Lowdown on Listeria}
            \begin{itemize}
                \item Listeriosis is an infection caused by the bacterium Listeria.
                \item Listeria is found in undercooked meats and unpasteurized milk and cheeses.
                \item The young, elderly, and pregnant women are at higher risk for foodborne illness
                \item Pregnant women should avoid eating the following to prevent contamination with Listeria
                    \begin{itemize}
                        \item Deli meats, hot dogs, salami, unpasteurized milk and cheese, undercooked or raw meat, chicken, or fish, raw alfalfa and broccoli sprouts, celery, and cantaloupes
                    \end{itemize}
            \end{itemize}

        \subsection{What Can You Do to Prevent Foodborne Illness?}
            \begin{itemize}
                \item Bacteria thrive when these conditions exist:
                    \begin{itemize}
                        \item Adequate nutrients
                        \item Moisture
                        \item Correct pH
                        \item Correct temperature
                        \item Time
                    \end{itemize}
                \item Fight BAC! for food safety:
                    \begin{itemize}
                        \item Clean
                        \item Separate
                        \item Cook
                        \item Chill
                    \end{itemize}
                \item \textbf{Clean} your hands and produce
                    \begin{itemize}
                        \item Hands: hot soapy water with agitation for 20 seconds
                        \item Sanitize cutting boards, sponges
                        \item Wash fruits and vegetables under cold running water, scrub firm skins with vegetable brush
                    \end{itemize}
                \item \textbf{Separate} meat and non-meat foods to combat \textbf{cross-contamination}
                    \begin{itemize}
                        \item Keep raw meat, poultry, fish separate from other foods during preparation, storage, transport
                        \item Don't use meat marinades as serving sauce
                        \item Use separate knives and cutting boards, clean dish towels
                    \end{itemize}
                \item \textbf{Cook} foods thoroughly
                    \begin{itemize}
                        \item Raw meats, poultry, fish can cause illness
                        \item Color not reliable indicator
                            \begin{itemize}
                                \item Meat may look brown but be undercooked
                                \item Cooked chicken may still look pink
                            \end{itemize}
                        \item For safety, measure internal temperature
                    \end{itemize}
                \item \textbf{Chill} foods at a low enough temperature
                    \begin{itemize}
                        %TODO ADD DEGREE SYMBOL INSTEAD OF *
                        \item Bacteria multiply rapidly in the "danger zone" between 40*F and 140*F
                            \begin{itemize}
                                \item Keep hot foods hot: above 140*F
                                \item Keep cold foods below 40*F: perishables shouldn't be left out more than two hours
                                \item Keep leftovers no more than four days in refrigerator, raw meats two days
                            \end{itemize}    
                        \item Freezer temperature: at or below 0*F
                    \end{itemize}
            \end{itemize}

        \subsection{Who Protects Your Food and How Do They Do It?}
            \begin{itemize}
                \item Several government agencies police the food supply
                    \begin{itemize}
                        \item \textbf{Food Safety Initiative (FSI):} joint effort of agencies has caused decline in foodborne illness
                        \item \textbf{DNA fingerprinting:} used in food safety to distinguish between different strains of a bacterium
                        \item Hazard Analysis and Critical Control Points (HACCP)
                            \begin{itemize}
                                \item Food safety program of FDA and USDA
                            \end{itemize}
                        \item \textbf{Farm-to-table continuum:} visual tool showing food safeguards from farmer to consumer
                    \end{itemize}
                \item \textbf{Food preservation:} Food manufacturers use preservation techniques to destroy contaminants
                    \begin{itemize}
                        \item \textbf{Pasteurization:} heating liquids, food at high enough temperatures to destroy foodborne pathogens
                            \begin{itemize}
                                \item Examples: milk, dairy foods, most juices
                            \end{itemize}
                        \item \textbf{Canning:} heating food at high temperature to kill bacteria, packing food in airtight container
                            \begin{itemize}
                                \item \textbf{Clostridium botulinum spores} can survive in environments without air
                                    \begin{itemize}
                                        \item Very rare cases of botulism usually occur from home canning
                                        \item Honey should not be fed to children under 1 year old
                                        \item \textbf{Retort canning:} high temperature after canning
                                    \end{itemize}
                            \end{itemize}
                        \item \textbf{Modified atmosphere packaging (MAP):} reducing oxygen inside packages of fruits/vegetables
                        \item \textbf{High-pressure processing (HPP):} high pressure pulses destroy microorganisms
                    \end{itemize}
                \item \textbf{Irradiation}
                    \begin{itemize}
                        \item Food subjected to radiant energy source without causing harmful chemical changes
                            \begin{itemize}
                                \item Kills bacteria but not viruses
                                \item Irradiated food must be labeled and have "radura" logo
                            \end{itemize}
                    \end{itemize}
                \item Product dating can help you determine peak quality
                \item \textbf{Closed (coded) dating:} packing numbers used by manufacturers on nonperishable foods to track inventory, rotate stock, identify products that may need to be recalled
                \item \textbf{Open dating:} calendar date on perishable foods to indicate food quality (not food safety)
                    \begin{itemize}
                        \item Labeled with "Sell By" or "Use By" date
                        \item If product not stored at proper temperature, may be unsafe even if used by the date
                    \end{itemize}
            \end{itemize}

        \subsection{What Are Food Additives and How Are They Used?}
            \begin{itemize}
                \item Commonly used \textbf{food additives} include preservatives, nutrients, and flavor enhancers
                \item Preservatives prevent spoilage and increase shelf life
                    \begin{itemize}
                        \item Most additives are preservatives
                        \item \textbf{Nitrites} and \textbf{nitrates:} salts added to prevent microbial growth
                            \begin{itemize}
                                \item In cured meats, prevent \textbf{Clostridium botulinum}
                                \item Use of these salts form carcinogenic nitrosamines
                            \end{itemize}
                        \item \textbf{Sulfites:} antioxidants that prevent browning, inhibit microbial growth
                    \end{itemize}
                \item Some additives enhance texture and consistency
                    \begin{itemize}
                        \item Gums and pectins used for consistency, texture
                        \item Emulsifiers improve stability, consistency, homogeneity
                        \item Leavening agents added to breads to cause them to rise
                        \item Anti-caking agnets prevent moisture absorption and lumping
                        \item Humectants increase moisture
                    \end{itemize}
                \item Some additives improve nutrient content
                    \begin{itemize}
                        \item Grains enriched with B vitamins, iron
                        \item Folic acid added to breads, cereal, grain products
                    \end{itemize}
                \item Color and flavor enhancers improve the appeal of foods
                    \begin{itemize}
                        \item FDA certifies color additives
                            \begin{itemize}
                                \item Both man-made and natural
                            \end{itemize}
                        \item \textbf{MSG} is a common flavor enhancer
                            \begin{itemize}
                                \item May cause headache, nausea, other side effects
                            \end{itemize}
                    \end{itemize}
                \item Food additives are closely regulated by the FDA
                    \begin{itemize}
                        \item Some exceptions based on \textbf{prior-sanctioned status} (pre-1958) and long history of use
                            \begin{itemize}
                                \item Nitrates for meat preservation
                                \item Salt, sugar, spices, other foods "generally recognized as safe" (GRAS status)
                            \end{itemize}
                    \end{itemize}
                \item Some food additives are unintentional
                    \begin{itemize}
                        \item Chemicals from processing
                        \item Dioxings used in paper bleaching (coffee filters)
                    \end{itemize}
            \end{itemize}

        \subsection{What Are Toxins and Chemical Agents?}
            \begin{itemize}
                \item Toxins occur naturally
                    \begin{itemize}
                        \item \textbf{Marine toxins:} cooking won't kill them
                            \begin{itemize}
                                \item Spoiled finfish can cause \textbf{scombrotoxic (histamine) fish poisoning}
                                \item Large reef fish can \textbf{bioaccumulate} ciguatoxins produced by dinoflagellates, causing \textbf{ciguatera poisoning}
                                \item Shellfish can be contaminated by \textbf{neurotoxins} produced by dinoflagellates, causing \textbf{paralytic shellfish poisoning}
                            \end{itemize}
                    \end{itemize}
                \item Toxins in other foods
                    \begin{itemize}
                        \item Potatoes exposed to light and turned green contain \textbf{solanine}
                        \item Wild lima beans, cassava contain \textbf{cyanogenic glycosides} that can cause cyanide poisoning
                    \end{itemize}
                \item Contamination is sometimes due to pollution
                    \begin{itemize}
                        \item \textbf{Polychlorinated biphenyls (PCBs)} may cause cancer in humans
                            \begin{itemize}
                                \item Now banned, but still in environment
                            \end{itemize}
                        \item PCBs and \textbf{methylmercury} can bioaccumulate in fish
                    \end{itemize}
            \end{itemize}

        \subsection{What Is Bioterrorism and How Can You Protect Yourself?}
            \begin{itemize}
                \item Food and water are potential targets
                    \begin{itemize}
                        \item As primary agents of \textbf{bioterrorism:} foodborne pathogens such as botulism, \textbf{Salmonella, E. coli} O157:H7, \textbf{Shigella}
                        \item As secondary agents: disrupting availability of adequate safe amounts or by limiting fuel needed to cook and refrigerate perishable foods
                    \end{itemize}
                \item Under Department of Homeland Security, various local, state, and federal agencies work together for \textbf{food biosecurity}
            \end{itemize}

    \section{Chapter 14: Life Cycle Nutrition: Pregnancy through Infancy}
        
        \subsection{What Nutrients and Behaviors Are Most Important before Attempting a Healthy Pregnancy?}
            \begin{itemize}
                \item A man's diet and lifestyle affect the health of his sperm
                    \begin{itemize}
                        \item Smoking, alcohol and drug abuse, obesity may decrease sperm production and function
                        \item Zinc and folate are associated with healthy sperm production
                        \item Antioxidants (vitamins C and E, carotenoids) may help protect sperm from free-radical damage.
                        \item Should consume well-balanced diet of fruits and vegetables, whole grains, and healthy protein foods
                    \end{itemize}
                \item Women need to adopt a healthy lifestyle before conception
                    \begin{itemize}
                        \item Attain a healthy weight
                        \item Get adequate folic acid
                            \begin{itemize}
                                \item For new cells and baby's growth and development
                            \end{itemize}
                        \item Moderate fish and caffeine consumption
                            \begin{itemize}
                                \item Methylmercury is a problem in some fish
                                    \begin{itemize}
                                        \item See Nutrition in the Real World Mercury in Fish, Chapter 5
                                    \end{itemize}
                                \item Consume < 200 mg of caffeine/day
                                    \begin{itemize}
                                        \item > 500 mg/day may delay conception
                                    \end{itemize}
                            \end{itemize}
                        \item Avoid cigarettes and other toxic substances
                    \end{itemize}
            \end{itemize}

        \subsection{What Nutrients and Behaviors Are Important in the First Trimester?}
            \begin{itemize}
                \item During the first trimester, the fertilized egg develops into a fetus
                    \begin{itemize}
                        \item Full-term pregnancy is approximately 40 weeks long, divided into three \textbf{trimesters}
                        \item Moment of conception marks first trimester
                        \item During first few days, fertilized egg travels down fallopian tube to embed in lining of uterus
                        \item After eighth week of pregnancy, developing \textbf{embryo} is called a \textbf{fetus}
                        \item \textbf{Placenta:} attached to the fetus via the \textbf{umbilical cord}
                    \end{itemize}
            \end{itemize}

        \subsection{What Nutrients and Behaviors Are Most Important in the First Trimester?}
            \begin{itemize}
                \item "Morning Sickness" and cravings are common
                    \begin{itemize}
                        \item Big myth of pregnancy is that morning sickness happens only in morning
                        \item Causes of morning sickness are unknown, but fluctuating hormone levels may play role
                        \item Some women develop aversion to certain foods; others have cravings
                            \begin{itemize}
                                \item Pica: craving for nonfood substances, such as cornstarch, clay, dirt, baking soda
                            \end{itemize}
                    \end{itemize}
                \item Adequate weight gain supports the baby's growth
                    \begin{itemize}
                        \item Healthy women generally advised to gain 25 to 35 pounds during entire pregnancy
                            \begin{itemize}
                                \item Women having twins: 37-54 pounds
                            \end{itemize}
                        \item Typical weight gain in first trimester: 1 to 4.5 pounds
                        \item Gaining excess weight may:
                            \begin{itemize}
                                \item Make it difficult to lose weight after delivery
                                \item Cause overweight in mother for years
                                \item Increase risk that baby will be obese later in life
                            \end{itemize}
                    \end{itemize}
                \item The need for certain nutrients increases
                    \begin{itemize}
                        \item From moment of conception, pregnant woman needs certain vitamins and minerals in higher quantities
                        \item Most needs can be met with diet, but some nutrients require special attention
                            \begin{itemize}
                                \item Folate/folic acid: need 400 micrograms/day before conception, continued after
                                \item Iron: supplement needed for red blood cells, anemia prevention, fatal development
                                \item Zinc and copper: key to baby's cell growth
                                \item Calcium and vitamin D: to preserve bone mass
                            \end{itemize}
                        \item Other nutrients also a concern, especially for those who eat no animal products (vegans, vegetarians)
                            \begin{itemize}
                                \item Omega-3 fat DHA (in seafood) for brain, retina
                                \item Choline for healthy cell division, especially in brain
                                \item Vitamin B12 (in animal foods) for nerve cells, red blood cells, production of nucleic acids
                            \end{itemize}
                        \item Too much preformed vitamin A can be toxic, increase risk for birth defects
                            \begin{itemize}
                                \item Limit supplements to no more than 3,000 IU/day
                            \end{itemize}
                    \end{itemize}
                \item Pregnancy increases risk for foodborne illness
                    \begin{itemize}
                        \item \textbf{Listeria monocytogenes} may cause miscarriage, premature labor, \textbf{low birth weight}, developmental problems, infant death
                        \item Avoid raw and undercooked meat, fish, or poultry; unpasteurized milk, cheese, juices; raw sprouts
                    \end{itemize}
                \item Pregnant women should avoid many other substances
                    \begin{itemize}
                        \item Nicotine, alcohol, illicit drugs
                            \begin{itemize}
                                \item Risks: SIDS, FASDS, birth defects
                            \end{itemize}
                        \item Restrict caffeine intake to 200 mg or less
                    \end{itemize}
                \item The importance of critical periods
                    \begin{itemize}
                        \item \textbf{Critical periods:} developmental stages in first trimester when cells and tissue rapidly grow and differentiate to form body structures
                        \item Embryo and fetus are highly vulnerable to nutritional deficiencies, toxins, other harmful factors
                        \item Risk of miscarriage is greatest
                            \begin{itemize}
                                \item First trimester
                            \end{itemize}
                    \end{itemize}
            \end{itemize}
        
        \subsection{What Nutrients and Behaviors Are Important in the Second Trimester?}
            \begin{itemize}
                \item Pregnant women need to consume adequate calories, carbohydrate, and protein to support growth
                    \begin{itemize}
                        \item Should consume 340 calories more in second trimester than before pregnancy, and gain about 1 pound/week
                        \item Need 175g of carbohydrates/day (vs 130g for nonpregnant women)
                        \item Protein needs increase 35 percent, to about 71 g/day, during second and third trimesters
                    \end{itemize}
                \item Exercise is important for pregnant women
                    \begin{itemize}
                        \item Should get at least 150 minutes of moderate-intensity aerobic activity per week, spread over 7 days
                        \item Benefits:
                            \begin{itemize}
                                \item Improves weight gain within targeted ranges
                                \item Decreases aches and pains
                                \item Lessens constipation
                                \item Improves energy level
                                \item Reduces stress
                                \item Improves sleep
                                \item Lowers risk of gestational diabetes
                            \end{itemize}
                    \end{itemize}
                \item Potential complications: gestational diabetes and hypertension
                \item \textbf{Gestational diabetes:} diabetes that occurs in women during pregnancy
                    \begin{itemize}
                        \item Certain factors increase risk of gestational diabetes
                        \item Can result in \textbf{macrosomia:} large baby, weighing more than 8 pounds, 13 ounces
                        \item Increases risk of baby having \textbf{jaundice}, breathing problems, birth defects
                    \end{itemize}
                \item \textbf{Pregnancy-induced hypertension} includes:
                    \begin{itemize}
                        \item \textbf{Gestational hypertension:} high blood pressure develops about halfway through pregnancy
                        \item \textbf{Preeclampsia:} includes hypertension and protein in urine, a sign of kidney damage
                            \begin{itemize}
                                \item Treatment includes bed rest, medication, even hospitalization until baby can be safely delivered
                                \item If untreated, can lead to eclampsia
                            \end{itemize}
                        \item \textbf{Eclampsia:} can cause seizures in mother and is major cause of death in women during pregnancy
                    \end{itemize}
            \end{itemize}

        \subsection{What Nutrients and Behaviors Are Important in the Third Trimester?}
            \begin{itemize}
                \item Women should be taking in 450 extra calories/day and gaining about 1 pound/week
                    \begin{itemize}
                        \item Growing baby puts pressure on stomach and intestines; hormones slow food through GI tract
                        \item Heartburn and constipation may result
                            \begin{itemize}
                                \item Small, frequent meals avoiding spicy foods can help heartburn
                                \item More fiber-rich foods can help prevent constipation
                            \end{itemize}
                        \item Hemorrhoids may also develop
                    \end{itemize}
            \end{itemize}

        \subsection{What Special Concerns Might Younger or Older Mothers-to-Be Face?}
            \begin{itemize}
                \item Pregnant teenagers face special challenges: still growing and likely to have unbalanced diets
                    \begin{itemize}
                        \item May be short on iron, folic acid, calcium, and calories
                        \item Higher risk of pregnancy-induced hypertension, premature and low birth weight babies
                    \end{itemize}
                \item Women older than 35 more likely to develop diabetes and hypertension
                    \begin{itemize}
                        \item Achieve healthy weight prior to conception, avoid smoking, eat balanced diet, consume adequate folic acid
                    \end{itemize}
            \end{itemize}

        \subsection{What Is Breast-Feeding and Why Is It Beneficial}
            \begin{itemize}
                \item \textbf{Lactation:} production of breast milk in woman's body after childbirth
                    \begin{itemize}
                        \item Prolactin causes milk to be produced
                        \item Oxytocin causes milk to be released (\textbf{let down response})
                    \end{itemize}
                \item \textbf{Breast-feeding} provides physical, emotional, and financial benefits for mothers
                    \begin{itemize}
                        \item Breast-feeding helps with pregnancy recovery; reduces risk of some chronic diseases
                        \item Breast milk is less expensive, more convenient than formula
                            \begin{itemize}
                                \item Infant formula costs 2000 dollars per year; breastfeeding costs 300 dollars per year
                            \end{itemize}
                        \item Breast-feeding promotes bonding with baby
                    \end{itemize}
                \item Breast-feeding provides nutritional and health benefits
                    \begin{itemize}
                        \item Breast milk is best for an infant's unique nutritional needs
                            \begin{itemize}
                                \item Composition of breast milk changes as infant grows
                                \item \textbf{Colostrum:} fluid produced after birth that contains antibodies, protein, minerals, vitamin A
                                \item Breast milk is high in lactose, fat, B vitamins
                                    \begin{itemize}
                                        \item Low in protein and in more digestible form: alpha-lactalbumin
                                    \end{itemize}
                            \end{itemize}
                        \item Breast-feeding protects against infections, allergies, and chronic diseases and may enhance brain development
                            \begin{itemize}
                                \item Decreases risk and severity of diarrhea; meningitis; respiratory, ear, and urinary tract infections
                                    \begin{itemize}
                                        \item Lactoferrin proteins protects against bacteria, viruses
                                    \end{itemize}
                                \item May reduce risk of childhood obesity
                                \item May help infant's intellectual development
                                    \begin{itemize}
                                        \item Two fatty acids, DHA and AA, aid nervous system, brain development
                                    \end{itemize}
                            \end{itemize}
                    \end{itemize}
            \end{itemize}

        \subsection{What Are the Best Dietary and Lifestyle Habits for a Breast-Feeding Mother?}
            \begin{itemize}
                \item In first 6 months, mother produces ~24 ounce of breast milk/day; in second 6 months ~16 ounce/day
                    \begin{itemize}
                        \item Mother needs about 13 cups of fluids/day
                        \item Extra calories needed:
                            \begin{itemize}
                                \item First 6 months: 500/day (170 from fat stores, 330 from food)
                                \item Second 6 months: 400/day
                            \end{itemize}
                        \item Well-balanced diet should be similar to diet during pregnancy
                        \item Avoid alcohol and illicit drugs, limit caffeine, and follow FDA's guidelines on fish consumption
                    \end{itemize}
            \end{itemize}

        \subsection{When Is Infant Formula a Healthy Alternative to Breast Milk?}
            \begin{itemize}
                \item Some women may not be able to breast-feed
                    \begin{itemize}
                        \item Women with HIV, AIDS, human T-cell leukemia, or active tuberculosis; receiving chemotherapy and/or radiation; or using illegal drugs should not breast-feed
                        \item Infants with galactosemia cannot metabolize lactose and should not breast-feed
                        \item Women taking prescription medications should check with health care provider regarding safety
                    \end{itemize}
                \item Formula is the best alternative to breast-feeding
                    \begin{itemize}
                        \item Cow's milk doesn't meet nutritional needs
                            \begin{itemize}
                                \item Too high in potassium and sodium
                                \item Too low in fat and linoleic acid
                                \item Iron is poorly absorbed
                            \end{itemize}
                        \item Infant formula is as similar as possible to breast milk
                            \begin{itemize}
                                \item Regulated by FDA: sets nutrient requirements
                                \item Made from cow's milk (altered to improve nutrient content and availability) or soy protein
                                \item Powdered, concentrated liquid, or read-to-use
                                    \begin{itemize}
                                        \item Avoid infants sleeping with bottles to avoid \textbf{early childhood caries}
                                    \end{itemize}
                            \end{itemize}
                    \end{itemize}
            \end{itemize}

        \subsection{What Are the Nutrient Needs of an Infant and Why Are They So High?}
            \begin{itemize}
                \item Infants grow at an accelerated rate
                    \begin{itemize}
                        \item Double birth weight by about 6 months of age; triple by 12 months
                    \end{itemize}
                \item Monitoring infant growth
                    \begin{itemize}
                        \item Infants not receiving adequate nutrition may have difficulty reaching milestones
                        \item Failure to thrive (FTT): delayed in physical growth or size or does not gain enough weight
                        \item \textbf{Growth charts} track physical development
                            \begin{itemize}
                                \item Head circumference, length, weight, and weight for length measures used to assess growth
                            \end{itemize}
                    \end{itemize}
                \item Infants have higher nutrient needs
                    \begin{itemize}
                        \item 82 calories/kg of body weight for first 6 months
                        \item Need for increased carbohydrate and protein with age
                        \item Fat and overall calories should not be limited
                        \item Vitamin K injection needed due to sterile gut
                        \item Vitamin D drops needed (not enough in breast milk to prevent rickets)
                        \item Iron-rich foods (fortified cereals, puréed meats) should be introduced at 6 months
                        \item Vitamin B12 may be needed if mother's diet is deficient
                        \item May need fluoride if not in water used for formula
                    \end{itemize}
            \end{itemize}

        \subsection{When Are Solid Foods Safe?}
            \begin{itemize}
                \item \textbf{Solid foods} may be introduced once certain milestones are met
                    \begin{itemize}
                        \item Infant needs to be nutritionally ready: at 6 months old, infant iron stores depleted
                        \item Infant needs to be physiologically ready
                            \begin{itemize}
                                \item GI tract and kidneys cannot process solid foods in early infancy
                                \item Doubled birth weight to at least 13 pounds
                                \item \textbf{Tongue-thrust reflex} has faded (4 to 6 months)
                                \item Swallowing skills have matured adequately
                                \item Has head and neck control, able to sit with support
                            \end{itemize}
                    \end{itemize}
                \item Solid foods should be introduced gradually
                    \begin{itemize}
                        \item First puréed, then mashed
                        \item Puréed meat, fortified rice cereal are good first foods
                        \item Parents should be patient; infants often reject food at first
                        \item Commercial baby foods are of high quality, but homemade food may be cheaper
                    \end{itemize}
                \item Some foods are dangerous for infants and should be avoided
                    \begin{itemize}
                        \item Certain foods, such as hot dog rounds or raw carrots, may present choking hazard
                        \item \textbf{Food allergens} (dangerous for those with allergies)
                        \item Peanuts can be introduced at 6 months to 12 years
                        \item Honey may contain bacteria that causes \textbf{botulism}, which can be fatal
                        \item Fruit juices may contribute unneeded calories, displace other nutritious foods
                        \item Overabundance of breast milk and infant formula may decrease interest in other foods important for growth
                    \end{itemize}
            \end{itemize}

        \subsection{Nutrition in the Real World: Feeding the Baby}
            \begin{itemize}
                \item Natural mechanisms help newborns eat
                    \begin{itemize}
                        \item Rooting reflex
                        \item Sucks/swallow reflex
                        \item \textbf{Tongue-thrust reflex}
                        \item Gag reflex
                    \end{itemize}
                \item Infants should eat when hungry, not on set schedule
                    \begin{itemize}
                        \item Cues include waking/tossing, sucking on fist, crying/fussing
                    \end{itemize}
            \end{itemize}

        \subsection{A Taste Could Be Dangerous: Food Allergies}
            \begin{itemize}
                \item \textbf{Food allergy:} abnormal reaction by immune system to a particular food
                    \begin{itemize}
                        \item Two stages: sensitization and allergic reaction
                            \begin{itemize}
                                \item Mast cells: antibodies attach to these cells, setting the state for future allergic reaction
                            \end{itemize}
                        \item Reactions can occur with minimal exposure
                        \item Symptoms: vomiting, diarrhea, hives, asthma
                            \begin{itemize}
                                \item \textbf{Anaphylactic reactions} are life-threatening
                            \end{itemize}
                        \item Common sources in children: eggs, milk, peanuts
                    \end{itemize}
                \item \textbf{Food intolerance:} adverse reaction to a food that does not involve immune response
                    \begin{itemize}
                        \item Example: lactose intolerance
                    \end{itemize}
            \end{itemize}

    \section{Chapter 15: Life Cycle Nutrition: Toddlers through the Later Years}

        \subsection{What Are the Issues Associated with Feeding Young Children?}
            \begin{itemize}
                \item Two age-groups of very young children
                    \begin{itemize}
                        \item \textbf{Toddlers:} 1 to 3 years old
                        \item \textbf{Preschoolers:} 3 to 5 years old
                    \end{itemize}
                \item Growth slows compared with infancy
                    \begin{itemize}
                        \item Average weight/height gain in second year: 3 to 5 pounds, 3 to 5 inches
                        \item After that, per year: 4.5 to 6.5 pounds, 2.5 to 3.5 inches
                        \item Smaller appetites, lower calorie needs relative to infants
                    \end{itemize}
                \item Young children need to eat frequent, small meals with nutrient-rich foods
                    \begin{itemize}
                        \item Toddlers are very active, but have small stomach, eat less at mealtimes
                        \item Ages 2 to 3 need 1,000 to 1,400 calories daily
                            \begin{itemize}
                                \item Protein-rich foods such as lean meats, eggs, poultry, dairy, beans, fruits, vegetables, whole grains
                            \end{itemize}
                    \end{itemize}
                \item Be mindful of portion size
                    \begin{itemize}
                        \item Use child-size cups and plates
                        \item Tailor portions to child's needs
                    \end{itemize}
                \item By 15 months, self-feeding: using cup, spoon
                \item Avoid choking hazards: hot dogs, nuts, whole graps, round candy, popcorn, raisins, raw carrots should not be given to children younger than four
                    \begin{itemize}
                        \item Always eat when sitting up, never while riding in car
                    \end{itemize}
                \item Food choices at day care should be monitored
                \item Young children have special nutrient needs
                    \begin{itemize}
                        \item Calcium
                            \begin{itemize}
                                \item 1 to 3 years: 700 mg/day
                                \item 4 to 8 years: 1000 mg/day
                                \item 8 oz milk, fortified soy drink/orange juice = 300 mg
                            \end{itemize}
                        \item Iron deficiency can lead to developmental delays
                            \begin{itemize}
                                \item Cause: too much milk, other iron-poor foods
                                \item 1 to 3 years: 7 mg/day
                                \item 4 to 8 years: 10 mg/day
                                \item Provide lean meats, iron-fortified grains (cereal)
                            \end{itemize}
                    \end{itemize}
                \item Vitamin D: for healthy bones
                    \begin{itemize}
                        \item 1 to 8 years: 600 IU/day
                    \end{itemize}
                \item Fiber: for bowel regularity
                    \begin{itemize}
                        \item 1 to 3 years: 19g/day
                        \item 4 to 8 years: 25g/day
                        \item Whole grains, fruits, vegetables
                    \end{itemize}
                \item Fluids
                    \begin{itemize}
                        \item 1 to 3 years: 4 cups/day
                        \item 3 to 5 years: 5 cups/day
                        \item Water, milk, juice; also in fruits, vegetables
                            \begin{itemize}
                                \item No more than 4-6 ounces 100 percent fruit juice per day
                            \end{itemize}
                    \end{itemize}
                \item \textbf{Picky eating} and \textbf{food jags} are common in small children
                    \begin{itemize}
                        \item Division of responsibility in feeding:
                            \begin{itemize}
                                \item Parents control type of food offered, when, and where
                                \item Children control whether to eat and how much
                            \end{itemize}
                        \item Parents should serve as good role models, eat varied diet
                        \item \textbf{Food jags} (favoring some foods to exclusion of others) are usually temporary
                        \item Young children following vegetarian diet need adequate Vitamin D, calcium, iron, zinc, and Vitamin B12
                    \end{itemize}
            \end{itemize}
            
        \subsection{What Are the Nutritional Needs and Issues of School-Aged Children?}
            \begin{itemize}
                \item Quality of diet impacts growth of \textbf{school-aged children} (ages 6 to 11)
                    \begin{itemize}
                        \item Parents/caregivers should encourage and model healthy habits
                        \item Eating patterns may be affected by developmental disabilities
                            \begin{itemize}
                                \item Example: autistic children may fixate on certain foods
                            \end{itemize}
                    \end{itemize}
                \item \textbf{Childhood overweight and obesity}
                \item High obesity rates in school-aged children
                    \begin{itemize}
                        \item More than doubled in children, tripled in adolescents over past 30 years
                        \item Due to many factors: too many calories, too little physical activity
                            \begin{itemize}
                                \item Excess calories from sugary drinks, sports drinks, high-fat foods, larger portions
                                \item Less physical activity due to increased "screen" time, less physical education at school
                            \end{itemize}
                        \item Contributes to type 2 diabetes
                    \end{itemize}
                \item Daily food plans for kids help guide food choices
                    \begin{itemize}
                        \item Plans available at \href{https://www.myplate.gov}{https://www.myplate.gov}
                        \item Key messages:
                            \begin{itemize}
                                \item Eat foods from every food group every day
                                \item Choose healthier foods from each group
                                \item Make the right choices for you
                                \item Take healthy eating one step at a time
                                \item Use healthy fats
                                \item Be physically active on a regular basis
                            \end{itemize}
                    \end{itemize}
                \item The importance of breakfast
                    \begin{itemize}
                        \item A nutritious morning meal is very important
                        \item Aids mental function, academic performance, school attendance rates, psychosocial function, mood
                        \item May be associated with healthier body weight
                        \item Optimally, children should eat breakfast at home before school
                    \end{itemize}
                \item School lunch contributes to a child's nutritional status
                    \begin{itemize}
                        \item National School Lunch Program (NSLP) provides nutritionally balanced, low cost or free lunches
                            \begin{itemize}
                                \item Standards set in \textbf{Dietary Guidelines for Americans}
                            \end{itemize}
                        \item For some children, school lunch is healthiest meal of the day
                        \item U.S. Department of Agriculture (USDA) donates food to program
                            \begin{itemize}
                                \item Reimburses for breakfast and lunch meals if schools meet meal guidelines
                            \end{itemize}
                    \end{itemize}
            \end{itemize}

        \subsection{Examining the Evidence: Nutrition, Behavior, and Developmental Disabilities}
            \begin{itemize}
                \item Attention deficit \textbf{hyperactivity disorder (ADHD)}
                    \begin{itemize}
                        \item Also called \textbf{Attention deficit disorder (ADD)}
                        \item Avoidance of certain foods may help
                            \begin{itemize}
                                \item Sugar often blamed, but evidence lacking
                            \end{itemize}
                        \item Nutritional advice from registered dietitian maty help counter effects of medication, disruptive mealtimes
                    \end{itemize}
                \item \textbf{Autism spectrum disorder (ASD)}
                    \begin{itemize}
                        \item May involve GI tract, immune system, but link unclear
                        \item Possible diet strategies: restrict glutens, casein (milk protein), food allergens; increase vitamins/minerals
                    \end{itemize}
            \end{itemize}

        \subsection{Tips for Helping Children Eat Healthfully from myplate.gov}
            \begin{itemize}
                \item \textbf{Cut back on kids' sweet treats}
                    \begin{enumerate}
                        \item Serve small portions
                        \item Offer healthy drinks
                        \item Use the check-out lane that does not display candy
                        \item Choose not to offer sweets as rewards
                        \item Make fruit the everyday dessert
                        \item Encourage kids to invent new snacks
                        \item Play detective in the cereal aisle
                        \item Make treats "treats", not everyday foods
                        \item If kids don't eat their meal, they don't need sweet "extras"
                    \end{enumerate}
                \item \textbf{Be a healthy role model for children}
                    \begin{enumerate}
                        \item Show by example
                        \item Go food shopping together
                        \item Get creative in the kitchen
                        \item Offer the same foods for everyone
                        \item Reward with attention, not food
                        \item Focus on each other at the table
                        \item Listen to your child
                        \item Limit screen time
                        \item Encourage physical activity
                        \item Be a good food role model
                    \end{enumerate}
            \end{itemize}

        \subsection{What Are the Nutritional Needs and Issues of Adolescents?}
            \begin{itemize}
                \item \textbf{Adolescence:} stage of life between start of puberty and adulthood
                    \begin{itemize}
                        \item Many hormonal, physical, and emotional changes
                    \end{itemize}
                \item Peer pressure and other factors influence teen eating behaviors
                    \begin{itemize}
                        \item Peer influence, defiance of authority may prompt different diets, negative habits
                        \item Busy schedule may influence food choices, increase snacking
                    \end{itemize}
                \item Adolescents need calcium and vitamin D for bone growth
                    \begin{itemize}
                        \item Bone growth occurs in \textbf{epiphyseal plate}
                        \item Low calcium, vitamin D intake can cause low \textbf{peak bone mass}, increased fracture risk
                            \begin{itemize}
                                \item Soft drinks and diet sodas displace milk in diet
                            \end{itemize}
                    \end{itemize}
                \item Teenage girls need more iron
                    \begin{itemize}
                        \item Needed for muscle growth, blood volume
                        \item Girls have special need due to blood loss in menstruation
                            \begin{itemize}
                                \item Inadequate iron intake is common
                            \end{itemize}
                    \end{itemize}
                \item Adolescents: at risk for disordered eating
                    \begin{itemize}
                        \item Anorexia nervosa, bulimia, binging, and other behaviors typically emerge during adolescence
                        \item Disordered eating has emotional and physical consequences
                            \begin{itemize}
                                \item Nutrient deficiencies can affect energy level and health
                            \end{itemize}
                        \item Adolescents struggling with body image should seek help from mental health specialist
                    \end{itemize}
            \end{itemize}

        \subsection{What Are the Nutritional Needs of Older Adults?}
            \begin{itemize}
                \item \textbf{Life expectancy} has increased
                    \begin{itemize}
                        \item In 1900, 47 years
                        \item Today, ages 65 and older are fastest growing segment of population
                        \item Advances in medical research, health care, public health policy have increased life span
                        \item Number of older adults in United States will increase dramatically over next several decades
                    \end{itemize}
                \item Older adults need fewer calories, not less nutrition
                    \begin{itemize}
                        \item Metabolic rate declines with age, reducing calorie needs
                        \item Continued intake of nutrients required to build cells, repair tissues, reduce risk of chronic disease
                    \end{itemize}
                \item Older adults need to get adequate fiber and fluids
                    \begin{itemize}
                        \item Fiber reduces risk of diverticulosis, heart disease, other chronic illnesses
                        \item Fiber and fluids help prevent constipation
                        \item Declining thirst mechanism insreases risk of dehydration
                    \end{itemize}
                \item Older adults should monitor their micronutrients
                    \begin{itemize}
                        \item Preformed \textbf{vitamin A:} overconsumption may increase risk of osteoporosis and fractures
                        \item \textbf{Vitamin D:} ability to convert from sunlight (and to absorb and convert to active form in intestines and kidneys) declines with age
                            \begin{itemize}
                                \item Daily need increases from 600 IU/day to 800 IU/day for those aged 70 and older
                            \end{itemize}
                        \item \textbf{Vitamin B12:} Many over age 50 can't absorb natural form because stomach produces less HCl
                            \begin{itemize}
                                \item Synthetic form in fortified foods and supplements should be added to diet
                            \end{itemize}
                    \end{itemize}
            \end{itemize}

        \subsection{Older Adults' Nutritional Needs and Issues}
            \begin{itemize}
                \item \textbf{Iron:} deficiency uncommon but may result from lack of iron-rich diet, chronic malabsorption, intestinal blood loss, kidney disease, cancer, arthritis
                \item \textbf{Zinc:} needed for healthy immune system, ability to taste
                \item \textbf{Calcium:} absorption declines with age
                    \begin{itemize}
                        \item Women over 50 need 1,200 mg/day
                        \item Men need 1,000 mg/day until age 70, then 1,200 mg/day after
                    \end{itemize}
                \item \textbf{Sodium:} reduce intake to <= 1,500 mg/day at age 50
            \end{itemize}

        \subsection{What Additional Challenges Do Older Adults Face?}
            \begin{itemize}
                \item Eating right for health and to prevent and manage chronic disease
                    \begin{itemize}
                        \item Heart disease and stroke
                        \item Type 2 Diabetes
                        \item \textbf{Arthritis}
                        \item \textbf{Alzheimer's disease (AD)/dementia}
                        \item \textbf{Cancer}
                    \end{itemize}
                \item Compounds, such as fiber, vitamins, and minerals, in whole and lightly processed foods help prevent age-related diseases (cancer, heart disease, and more)
                    \begin{itemize}
                        \item Antioxidants help protect body from free radicals, may reduce risk of cognitive problems (Alzheimer's)
                    \end{itemize}
                \item Most older Americans not eating healthies diet
                    \begin{itemize}
                        \item Too much sodium, saturated fat, calories
                        \item Inadequate servings from various food groups (dairy, fruits, vegetables, whole grains)
                    \end{itemize}
            \end{itemize}

        \subsection{What Additional Challenges May Older Adults Face?}
            \begin{itemize}
                \item Heart disease and stroke
                    \begin{itemize}
                        \item Many older adults have diabetes and hypertension, which add to risk for heart attack, stroke
                        \item Cardiovascular disease more common with age
                            \begin{itemize}
                                \item Most common: coronary heart disease, number one cause of death in United States
                            \end{itemize}
                    \end{itemize}
                \item Type 2 diabetes
                    \begin{itemize}
                        \item Ability to maintain glucose blood level diminishes with age, can result in diabetes
                        \item Most people with this illness are overweight
                    \end{itemize}
                \item Hypertension
                    \begin{itemize}
                        \item Controlled with medication, weight loss/physical activity, limiting alcohol/sodium
                    \end{itemize}
                \item Arthritis: painful inflammation in joints
                    \begin{itemize}
                        \item Osteoarthritis most common, may be eased by supplements glucosamine and chondroitin
                        \item Rheumatoid arthritis: omega-3 fats in seafood may help
                    \end{itemize}
                \item Alzheimer's disease: form of dementia
                    \begin{itemize}
                        \item Healthy diet, physical activity, social engagement may help reduce risk
                    \end{itemize}
                \item Economic and emotional conditions can affect nutritional health.
                    \begin{itemize}
                        \item \textbf{Food insecurity:} limited access to adequate, nutritious food
                            \begin{itemize}
                                \item Older Americans Act (1965) provides support and services for ages 60 and older, including nutrition education and \textbf{congregate meals}
                            \end{itemize}
                        \item Depression and grief affect nutrition and health
                        \item Alcohol abuse can add to depression, impair judgment and coordination, lead to accidents
                            \begin{itemize}
                                \item Alcohol may interact negatively with medication
                            \end{itemize}
                    \end{itemize}
                \item Staying physically active
                    \begin{itemize}
                        \item A necessity, not an option
                        \item Many benefits:
                            \begin{itemize}
                                \item Lowers risk of chronic disease
                                \item Helps maintain healthy bones, muscles, joints
                                \item Reduces anxiety, stress, depression
                                \item Improves sleep, flexibility, range of motion
                                \item Can help postpone cognitive decline
                                \item Promotes independent living
                            \end{itemize}
                        \item Suggested exercise per week for adults over age 65 (2008 Physical \textbf{Activity Guidelines for Americans}):
                            \begin{itemize}
                                \item 150 minutes moderate intensity or 75 minutes vigorous intensity, or combination of the two
                                \item Muscle strengthening activities two or more days
                                \item If unable to meet these goals, be as physically active as possible
                                \item Avoid sitting for long periods
                            \end{itemize}
                    \end{itemize}
            \end{itemize}
        
        \subsection{Fighting Cancer with a Healthy Lifestyle}
            \begin{itemize}
                \item \textbf{Cancer} includes 100+ diseases characterized by uncontrolled growth and spread of abnormal cells
                    \begin{itemize}
                        \item Half of all men and one-third of women in United States will develop cancer during lifetime
                        \item Most common
                            \begin{itemize}
                                \item Men: prostate cancer
                                \item Women: breast cancer
                            \end{itemize}
                    \end{itemize}
                \item Carcinogens are thought to cause most cancers
                    \begin{itemize}
                        \item About 5 percent hereditary, rest caused by damage to DNA by carcinogens:
                            \begin{itemize}
                                \item \textbf{Tobacco:} primary cause of lung cancer, mainly among smokers but also from secondhand smoke
                                \item \textbf{Alcohol:} moderate to heavy consumption associated with head and neck, breast, colorectal, esophageal, and liver cancers
                                \item \textbf{Radiation:} overexposure to UV radiation (sunlight) and other forms (X-rays)
                                \item \textbf{Industrial chemicals:} certain metals (nickel), pesticides, and compounds (benzene)
                                \item \textbf{Cancer-causing agents in foods and beverages}
                            \end{itemize}
                    \end{itemize}
                \item You can reduce your risk for cancer with a healthy diet
                    \begin{itemize}
                        \item \textbf{Phytonutrients} (lycopene)
                        \item \textbf{Antioxidants} (carotenoids, selenium)
                        \item \textbf{Retinoids} (vitamin A), vitamin D, folate
                        \item \textbf{Omega-3 fatty acids} (in fish, some oils)
                        \item \textbf{Fiber}
                            \begin{itemize}
                                \item Helps dilute, shed waste products in intestinal tract
                                \item Feeds healthy bacteria in colon, creating by-product that may help fight cancer
                            \end{itemize}
                    \end{itemize}
                \item Avoid foods and beverages that may increase your risk for cancer
                    \begin{itemize}
                        \item Diet high in red and/or processed meats
                            \begin{itemize}
                                \item Nitrites in processed meats can react with amino acids to form cancer-promoting compounds (nitrosamines, nitroamides)
                            \end{itemize}
                        \item Alcohol, consumed in excess
                        \item High salt consumption
                        \item Excess body weight
                            \begin{itemize}
                                \item Contributes to as many as 1 in 5 cancer-related deaths
                            \end{itemize}
                    \end{itemize}
            \end{itemize}

        \subsection{Nutrition in the Real World: Drug, Food, and Drug-Herb Interactions}
            \begin{itemize}
                \item Drugs, food, and herbs can interact in negative, unhealthy ways
                    \begin{itemize}
                        \item Foods can delay or increase absorption of drugs
                            \begin{itemize}
                                \item Example: Calcium binds with tetracycline (antibiotic), decreasing absorption
                            \end{itemize}
                        \item Drugs can interfere with metabolism of substances in foods, leading to dangerously high levels in blood
                            \begin{itemize}
                                \item Example: tyramine in cheese, smoked fish, yogurt
                            \end{itemize}
                        \item Herbal remedies may be unsafe with medication
                            \begin{itemize}
                                \item Example: \textbf{Ginkgo biloga} interferes with blood clotting, should not be taken with blood thinners Coumadin or aspirin
                            \end{itemize}
                    \end{itemize}
            \end{itemize}





\end{document}